\documentclass[12pt,letterpaper]{article}

\linespread{1}

\title{Open Source Licensing}
\author{
    Trevor Bramwell\\
    Professor Carlos Jensen\\
    CS391: Ethics in Computer Science -- MWF 1300\\
}
\date{\today}

\usepackage{hyperref}

\begin{document}
\maketitle

\begin{abstract}
This paper explains the expression ``Free as in Freedom, not as in
beer.'' Defines What open source is and is not in terms of intellectual 
property. The rights Open Source developers can retain and can not retain 
when release their code under open source license. And a description and
definition of key terms of the most common OSI License.
\end{abstract}

\section{Free as in Freedom, Not as in Beer}
When people hear 'free' they generally think of something that costs
nothing: ``It's free!'' they might even exclaim to their friends. But
when refering to free software, people are not talking about the `free'
others may think they are. This free is the `free' in `freedom' not
the `free' in `free beer'.

This free in free software means the user has been granted freedom.
Freedom to edit, modify, redistribute, copy, run, and study the source
code of any program that is free software. It does not mean it has to be
freely given away. People have every right to charge for the laybor they
have invested in creating this software and marketing it to others.

\section{Free Software vs. Open Software}

\pagebreak
\include{biblioagraphy}

\end{document}
