\documentclass[12pt,letterpaper]{article}


\title{Open Source Licensing}
\author{
    Trevor Bramwell\\
    Professor Carlos Jensen\\
    CS391: Ethics in Computer Science -- MWF 1300\\
}
\date{\today}

\usepackage{hyperref}
\usepackage{fullpage}
\linespread{1}

\renewcommand{\abstractname}{Overview}

\begin{document}
\maketitle

\begin{abstract}
This paper explains attemps to provide an introduction to open source
licensing, the different freedom licenses can provide, and list some key
terms and their definitions that open source licenses might contain.  It
also will explain the expression ``Free as in Freedom, not as in
beer.'', define what open source as it relates to intellectual property,
examine what rights Open Source developers can and cannot retain when
release code, and define what it means for a license to be GPL
compliant.
\end{abstract}


\section{Free as in Freedom, Not as in Beer}
When people hear 'free' they generally think of something that costs
nothing. ``It's free,'' they might exclaim to their friends. But
when refering to free software, people are not talking about the `free'
others may think they are. This free is the `free' in `freedom' not
the `free' in `free beer'. It is a question of liberty, not payment.

Users of free software are given several freedoms. They have the freedom
to edit, modify, redistribute, copy, run, and study the source code of
the software. Athough free software is
generally given away for free, the free software definition does not
require the software creator to sell their product, or give it away for
free. Rather it gives the creator the freedom to choose wheather or not
they want to charge for their product. This freedom labeled as the
second freedom in 'The Free Software Definition'[1].

These are a summary of the four freedoms granted by the definition:
    \begin{enumerate}
    % Start counter at 0 because that is how the freedoms are numbered
    \setcounter{enumi}{-1}
        \item Freedom to run the program.
        \item Freedom to study and modify the source code.
        \item Freedom to redistribute copies.
        \item Freedom to distribute your modified copies.
    \end{enumerate}

The most important freedom of these four is the freedom to distribute
your modified copies. It is this freedom

It should be noted that all of these freedoms are 'freedoms' they are
not 'restictions'. Individual creators have the right to choose not to
distribute their modified copies. Although the extent of this freedom is
generally decided by the license the creator has released his or her
code under.

\subsection{Free Software vs. Open Source}


\section{Intellectual Property as it relates to Open Source}
The free and open source movement all began with a printer.


\section{Retainable and Non-Retainable Developer Rights for OSI Licenses}


\section{Key Terms}
\subsection{Copyleft}
The key difference between Copyright and Copyleft lies in who is given
the rights. Under Copyright the rights extend to authors, whereas under
Copyleft the rights extend to every user, since every user is seen as a
potential author.


\section{GPL Compliance}

\include{biblioagraphy}

\end{document}
