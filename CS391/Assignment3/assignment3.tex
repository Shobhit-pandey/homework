%----
% Author: Trevor Bramwell
% Date: Mon Jun  4 21:35:04 PDT 2012
%----

\documentclass[12pt,letterpaper]{article}

\title{The Internets}
\author{
    Trevor Bramwell\\
    Professor Carlos Jensen\\
    CS391: Ethics in Computer Science -- MWF 1300\\
}
\date{\today}

\usepackage{url}
\usepackage{hyperref}

\hypersetup{
    colorlinks,
    citecolor=blue,
    urlcolor=black,
}

\usepackage{fullpage}
\linespread{1}

\begin{document}
\maketitle

\section{Introduction}
I have grown up in the interned age. It was in  middle school when my
family got our first computer that had a modem as well. It wasn't until
about freshman year of High School that we upgraded our service to DSL,
and then later Cable. Because of the limited speed of our Internet, and 

Having grown up in the Internet age and experiencing the transition from
56k modems to DSL, then Cable, and now Fiber Internet,  

The Internet. I have used the Internet all my life. For two days I went
without the Internet. It wasn't hard. I'm sure for others it might have
been, but I am used to it. Not that I generally don't use the Internet,
but because I spend a lot of time offline, doing other things. I read,
play guitar, dance (boy do I DANCE!), but most of all I don't use the
Internet all that much because of what it represents. Nope. Quit trying
to be philosophical and just write. The Internet. Bleh. Ummmmmm. It
wasn't hard to not use the Internet because most of the time I spend
using the Internet is a waste of time. Recently I deleted my Facebook
account, and I found that I caught myself typing \url{http://www.Facebook.com}, more
often than not, when I don't even have one. At random times. If I'm
bored. Facebook. Someone asks/tells me to look at a picture, Facebook.
Email notification that I have a new message. Facebook. Facebook,
Facebook, Facebook. Our culture is addicted to it. It's a disease. Well,
not as much of a disease as Angry Birds marketing is, but you get the
point.

 Why?...Why not? What is so pressing about
Facebook that American's need one?

The only reason my friends have a Facebook is because my friends have a
Facebook. This sounds tautological, because it is. Social media has
always been seen as this great connector, or enabler of community, but
most of the time it is wasted  with status updates of breakups, baby
facts, or cat pictures. None of the discussion revolves around anything
important, or if it does it is mainly conversation in ALL
CAPS!!!1!!!.

11:45 PM. May 15th, 2012. Outside GameStop in Corvallis, Oregon. While hundreds of gamers are lined up
in the cold to get their hands on the midnight release of Diablo 3 for
PC, I stood in line as one of the few awaiting a copy of Max Payne 3 for
Xbox 360. I have been waiting for this game since about four months ago
when I saw an announcement in GameInformer magazine. 


\section{No Internet}
8:30am - Woke up to have a long gaming weekend. Went to look up Max
Payne 3 achievement guide online, but couldn't.

The first morning I didn't have Internet I wanted to lookup a guide for
max payne 3 achievements. I couldn't do this though. The night before I
knew when I woke up in the morning I would forget, so I disconnected my
Internet card on my Xbox and PC.i This was my first challenge, and I
failed. Two hours after getting up and realizing my Internet was turned
off, I mulled the thought over in my head.  I gave in and turned my PC
Internet back on. I went and downloaded the guide from the website and
turned my Internet right back off. This is supposed to be for for a full
48 hours, so I wouldn't turn my Internet back on until 10am the next.


Before the dance that night my girlfriend asked me to look up a recipe
for her. I responded that I couldn't because I couldn't use the Internet
for two days. She remembered and looked up the recipe herself. It was
only a minor inconvenience.


I was at a dance last weekend and a friend asked me to if I was
going to a friend's party that night. I said "What party"? They looked
at me funny. The one on Facebook, they said. Didn't you get the invite?
I had to laugh. No, I said. I don't have a Facebook anymore. Their
immediate reaction: "Why?".





\subsection{Difficulties}
\subsection{How disconnect made me feel}
\subsection{Reflection on experience and feelings}


\section{Internet}
\subsection{Weekday}
During a weekday I get up and go to school. Sometimes in the morning
before I go downstairs to make lunch I will play a few rounds of a
Multi-player, or Single-player game on Xbox. Then I go wait for the bus,
streaming music while I wait. On the Bus I generally keep streaming
music until I get off and walk all the way to work. At which time I
turn on my computer, login and startup IRC, and Email., and various
other websites for internal use at the OSL. I am constantly online while
I am at work either looking at email, or on IRC. During the day I have
class, so I generally leave and walk to class. Sometimes I will stream
music on my way to class. While in class I am generally completely
unplugged, unless it is a CS class where laptops are allowed.  After
class I stream music sometimes on my way back to work.  Where I plug
back in and am on IRC and email.

After work I'll walk to the bus and stream music all the way home.
(~20min). Once I am home I generally play video games, or work on my
homework. But as soon as I get home I generally make dinner,. If I am
making something simple I won't use the Internet. If it is something new
or I am being creative I will look up an recipe online. After dinner I
go to my room and play video games, or watch TV shows on Netflix until
bed.

Sometimes in the morning I wake up and, because my phone is my alarm,
check my email/Reddit/Slashdot/news.


\subsection{Non-Weekday}

On a non weekday , If I didn't stay up all day on Friday night then I
will wake up early to play some video games most of the day. Generally
depending on what game I am playing most of this will be offline. Even
though I am not playing multiplayer though, I will still be signed into
Xbox live. Most of my day will be spent playing video games. The other
day I spent about 4ish hours trying to get a game setup. I had just
downloaded the Humble Indie bundle 5. I wanted to play Limbo on Linux, but
the deb package wouldn't work, so I downloaded steam and installed it
through wine. This whole process was not easy and took me longer than it
should of because of the way wine works. I had to reference the Internet
several times to find the information I needed. I would not have of been
able to make it work without access to the Internet.

It is very obvious how reliant we are upon the Internet. There was a
satiric blog post written recently about icons on the iPhone that are
unrecognizable because we have become so detracted from technology
because everything is now online. For example on the iPhone the icon to
make a phone call is a dial-tone phone (chorded) (Back when people still
used landlines.)  The authors point was that most people don't use these
phones anymore and ironically the icon is for the device they are using
(the iPhone).


\section{Conclusion}

\section{Questionnaire}
\subsection{What devices do you use to access the Internet?}
The devices our household use to access the Internet are phones, game
consoles, laptops, desktops, and tablets. Daniel mainly uses the Tablet
or his iPhone. Jaymee mainly uses her tablet and iPhone. Daniel and
Jaymee both use the Wii to watch Netflix. Jacob uses his
laptop to watch Hulu. Ashlee uses her laptop and kindle fire. Courtney uses her
laptop. I use my laptop, phone, desktop and Xbox.

\subsection{What resources do you access?}
The majority of sites our household access are these:
Facebook, Reddit, Hulu, Netflix, Pintrest, Reddit, Pandora, craigslist,
eBay, wikipedia, amazon, oregonstate.edu.

\subsection{For what purpose?}
Most of the Internet usage at our house is for entertainment. A large
portion of it is probably taken up by either Netflix or Hulu streaming or
Torrenting of TV Show or Movies. After that comes gaming, as I am
constantly on the Internet playing multiplayer games. Then there is
social stuff. Facebook probably taking the cake, followed by Pintrest,
and then Reddit. After that would be Financial. Daniel spends a lot of
his time on Craigslist or eBay looking for great deals, or garage sales.

\subsection{For how long a period of time?}
Since most of the usage is taken up by entertainment or streaming, I'd
say generally between an hour to three hours. When it comes to gaming I
generally play from half an hour to around four hours.

\subsection{Bandwidth Division}
Most of the bandwidth is used for family purposes, entertainment,
gaming, social, etc. Rarely is anyone using it for 'work' with the
exception of Daniel using it for craigslist and eBay. I would say the
discretionary part happens generally later at night if at all in our
household. Of the people who live here, four of the five people are
couples.


\end{document}
