%----
% Author: Trevor Bramwell
% Date: Mon Jun  4 21:35:04 PDT 2012
%----

\documentclass[12pt,letterpaper]{article}

\title{A Minor Inconvinence}
\author{
    Trevor Bramwell\\
    Professor Carlos Jensen\\
    CS391: Ethics in Computer Science -- MWF 1300\\
}
\date{\today}

\usepackage{url}
\usepackage{hyperref}

\hypersetup{
    colorlinks,
    citecolor=blue,
    urlcolor=black,
}

\usepackage{fullpage}
\usepackage{indentfirst}
\linespread{1}

\begin{document}
\maketitle

I grew up in the interned age. In 1998, my family got our first
computer. It came with a 56kb modem. It allowed my parents to recieve
email and my brothers and I to play games with our friends. Most of our
time on the computer was spent offline though, as dialing out for
Internet connectivity was tedious and tied up the phoneline. Every once
in a while, if my parents said it was okay because they weren't
expecting any calls, I could use the internet. Most of my time was spent
reading forum posts on Myst related topics.

After my brother picked up a book from the library on programming in
BASIC, I took more of an interest to computers. I wanted to know how
they worked, what they were made out of, how they communicated. I was
facinated by their ability to connect people not only from across the
street, but from around the world. And it was all possible because of
this silly little language called HTML. 

It wasn't long after I attempted to learn BASIC that I picked up a book
on HTML from the library. It was much easier, more concise, and just
plain made sense to my growing middle school brain. Written words on a
computer created remarkable visualizations. I was hooked. I had to learn
more.

When I first came to know and love the Internet, it was near end of the
Dot-com Bubble: A rough and marvelous time for growth on the Internet.
Since then the Interent has grown to connect a very large amount of the
United States population. Growing up with the Internet all of my life, i
was curious to find out just how addicted I had become. So for two days
I went without the Internet. 

To be honest living without the Internet wasn't that hard. Being that I
had grown up with using a 56kb modem the majority of my life. I am
gracious for how far we have really come in terms of the speed at which
we can communicate online. Going back to using a pen and paper to take
notes during class was trivial, since I already did so. Using my phone
to text and call friends was just as trivial.

My plan for the first day I didn't have Internet was to play Max Payne 3
all day. The game had just come out for Xbox 360 and I really enjoyed
the gameplay. The night before I had stayed up working on earning some
of the extra achievements, since I already had beaten the game.
Following a guide online was the easiest solution to figuring out how to
gain the achievements, and where to find some hidden items. In my wisdom
I had disconnected both my Xbox and computer from the Internet the night
before because I knew I would forget that I wasn't supposed to use the
Internet the next day. At first it was okay; I continued to search
through the levels for some of the collectables on my own, but as I 
continued to play, finding the items on my own was tedious. Two hours
after playing I gave up. Turning my internet back on, on my computer, I
downloaded the guide for the levels I had left to complete and then
turned my Internet immediately back off. 

Only two hours after not going without the Internet, I had already
failed and had to turn it back on.

Throughout the rest of the day I continued to have minor cravings,
mainly brought on by bordem: I wanted to browse Reddit, or read posts on
HackerNews. Sometimes it was because of habit. I'm used to googling
information from my phone. When someone mentions a word I don't know,
I'll look it up, or if someone mentions a person or place I have never
heard of, I'll check Wikipedia. Dealing with that craving was the only
real challenge when going without Internet.

Later in the day I went to a dance at school.
while mingling with some friends, one of them asked if I was going to
another friend's party after the dance. I asked "What party?" and they
gave me a funny look. "The one on Facebook," they responded, "Didn't you
get the invite?" I laughed out loud. Not only was I unable to go online
to see the invitation, but I permantly deleted my Facebook account about
a month ago. Although I didn't have an invitation, I knew I didn't need
one, so I decided to head over there afterwards. I had heard something
about the host of the party moving to a different house, but figured I
would try his old house anyways.

When I got to my friend's old house I could hear music playing from
inside. I was certain the party was going on in there. Walking in the
door I was greeted by a somewhat unfamiliar site: two girls wearing
flannel and jean shorts, and two guys with neon tanktops playing beer
pong. I knew immediately that I was in the wrong place since most of the
people at my friend's parties don't dress in flannel, but I
decided to ask just in case.

"Hey man!" I said in a forte voice to be heard over the loud music, "Is
this Joey's house?"

"No, this is Owen's house. Are you a friend of Owen's?" 

"Nope. I don't think I know any Owens. So you don't know a Joey?" I responded.

"Joey Williams?" the other guy in a neon tanktop replied.

"Uh, sure!" I couldn't actually remember his last name. "Tall, kinda
lanky guy with brown hair?"

"You mean red hair?" the other guy asked.

"No, brown."

"Nah, we don't know him, bro."

"Alright. Take it easy then!" It was obvious I was not in the right
house, and my friend had moved to a different part of town. There was
no way I would be able to figure out where he had moved at that time
without the Internet.

Those were just a few of the situations and experiences I had while not
using the Internet for two days. Following those two days I tracked my
usage of the internet for another two days.


During a weekday I get up and go to school. Sometimes in the morning
before I go downstairs to make lunch I will play a few rounds of a
Multi-player, or Single-player game on Xbox. Then I go wait for the bus,
streaming music while I wait. On the Bus I generally keep streaming
music until I get off and walk all the way to work. At which time I turn
on my computer, login and startup IRC, and Email., and various other
websites for internal use at the OSL. I am constantly online while I am
at work either looking at email, or on IRC. During the day I have class,
so I generally leave and walk to class. Sometimes I will stream music on
my way to class. While in class I am generally completely unplugged,
unless it is a CS class where laptops are allowed.  After class I stream
music sometimes on my way back to work.  Where I plug back in and am on
IRC and email.

After work I'll walk to the bus and stream music all the way home.
(~20min). Once I am home I generally play video games, or work on my
homework. But as soon as I get home I generally make dinner,. If I am
making something simple I won't use the Internet. If it is something new
or I am being creative I will look up an recipe online. After dinner I
go to my room and play video games, or watch TV shows on Netflix until
bed.

Sometimes in the morning I wake up and, because my phone is my alarm,
check my email/Reddit/Slashdot/news.


On a non weekday , If I didn't stay up all day on Friday night then I
will wake up early to play some video games most of the day. Generally
depending on what game I am playing most of this will be offline. Even
though I am not playing multiplayer though, I will still be signed into
Xbox live. Most of my day will be spent playing video games. The other
day I spent about 4ish hours trying to get a game setup. I had just
downloaded the Humble Indie bundle 5. I wanted to play Limbo on Linux,
but the deb package wouldn't work, so I downloaded steam and installed
it through wine. This whole process was not easy and took me longer than
it should of because of the way wine works. I had to reference the
Internet several times to find the information I needed. I would not
have of been able to make it work without access to the Internet.

It is very obvious how reliant we are upon the Internet. There was a
satiric blog post written recently about icons on the iPhone that are
unrecognizable because we have become so detracted from technology
because everything is now online. For example on the iPhone the icon to
make a phone call is a dial-tone phone (chorded) (Back when people still
used landlines.)  The authors point was that most people don't use these
phones anymore and ironically the icon is for the device they are using
(the iPhone).


Having first learned HTML after being put-off by BASIC, I became
fascinated with the internet.

\subsection{Difficulties} \subsection{How disconnect made me feel}
\subsection{Reflection on experience and feelings}

The devices our household use to access the Internet are phones, game
consoles, laptops, desktops, and tablets. Daniel mainly uses the Tablet
or his iPhone. Jaymee mainly uses her tablet and iPhone. Daniel and
Jaymee both use the Wii to watch Netflix. Jacob uses his laptop to watch
Hulu. Ashlee uses her laptop and kindle fire. Courtney uses her laptop.
I use my laptop, phone, desktop and Xbox.

The majority of sites our household access are these: Facebook, Reddit,
Hulu, Netflix, Pintrest, Reddit, Pandora, craigslist, eBay, wikipedia,
amazon, oregonstate.edu.

Most of the Internet usage at our house is for entertainment. A large
portion of it is probably taken up by either Netflix or Hulu streaming
or Torrenting of TV Show or Movies. After that comes gaming, as I am
constantly on the Internet playing multiplayer games. Then there is
social stuff. Facebook probably taking the cake, followed by Pintrest,
and then Reddit. After that would be Financial. Daniel spends a lot of
his time on Craigslist or eBay looking for great deals, or garage sales.

Since most of the usage is taken up by entertainment or streaming, I'd
say generally between an hour to three hours. When it comes to gaming I
generally play from half an hour to around four hours.

Most of the bandwidth is used for family purposes, entertainment,
gaming, social, etc. Rarely is anyone using it for 'work' with the
exception of Daniel using it for craigslist and eBay. I would say the
discretionary part happens generally later at night if at all in our
household. Of the people who live here, four of the five people are
couples.


Looking back on not having internet for two days, the experience was not
too earth shattering. Of course had I decided to not use the internet
for two days during the week, instead of the weekend, I would not have
been able to do my work, or do homework very well.


\end{document}
