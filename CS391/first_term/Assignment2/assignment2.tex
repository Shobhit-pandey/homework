\documentclass[12pt,letterpaper]{article}


\title{Open Source Licensing}
\author{
    Trevor Bramwell\\
    Professor Carlos Jensen\\
    CS391: Ethics in Computer Science -- MWF 1300\\
}
\date{\today}

\usepackage{url}
\usepackage{hyperref}

\hypersetup{
    colorlinks,
    citecolor=blue,
    urlcolor=black,
}

\usepackage{fullpage}
\linespread{1}


\renewcommand{\abstractname}{Overview}

\begin{document}
\maketitle

\begin{abstract}
This paper explains attemps to provide an introduction to open source
licensing, the different freedom licenses can provide, and list some key
terms and their definitions that open source licenses might contain.  It
also will explain the expression ``Free as in Freedom, not as in
beer.'', define what open source as it relates to intellectual property,
examine what rights Open Source developers can and cannot retain when
release code, and define what it means for a license to be GPL
compliant.
\end{abstract}


\section*{Free as in Freedom, Not as in Beer}
When people hear 'free' they generally think of something that costs
nothing. ``It's free,'' they might exclaim to their friends. But
when refering to free software, people are not talking about the `free'
others may think they are. This free is the `free' in `freedom' not
the `free' in `free beer'. It is a question of liberty, not payment.

Users of free software are given several freedoms. They have the freedom
to edit, modify, redistribute, copy, run, and study the source code of
the software. Athough free software is
generally given away for free, the free software definition does not
require the software creator to sell their product, or give it away for
free. Rather it gives the creator the freedom to choose wheather or not
they want to charge for their product. This freedom labeled as the
second freedom in 'The Free Software Definition'.

These are a summary of the four freedoms granted by the definition:
    \begin{enumerate}
    % Start counter at 0 because that is how the freedoms are numbered
    \setcounter{enumi}{-1}
        \item Freedom to run the program.
        \item Freedom to study and modify the source code.
        \item Freedom to redistribute copies.
        \item Freedom to distribute your modified copies.
    \end{enumerate}

The most important freedom of these four is the freedom to distribute
your modified copies. It is this freedom

It should be noted that all of these freedoms are 'freedoms' they are
not 'restictions'. Individual creators have the right to choose not to
distribute their modified copies. Although the extent of this freedom is
generally decided by the license the creator has released his or her
code under.

\subsection*{Free Software vs. Open Source}
There really no difference between Free Software and Open Source software.
Generally they even get lumped together and people refer to them as Free and
Open Source Software or FOSS, yet Richard Stallman, leader of the Free Software
Foundation would argue differently. Sallman says in his article `Why Open
Source misses the point of Free Software' that `` `open source software' — and the
one most people seem to think it means — is `You can look at the source
code.','' but this point is unjust, since when most people refer to Open Source
software they are refering to the liberties granted by it. 

There are only a few licenses that explicitly restrict authors from only
viewing their code, and none of the major Open Source license do this.
Stallman's argument for using the term Free Software is based on the assumption
that when people say `Free Software' they will also explain that the `Free'
refers to Freedom, yet when people say `Open Source' they will take no
provisions to explain it's meaning, and people will take it on face value.  
\cite{website:gnu-open-source}


\section*{Retainable and Non-Retainable Developer Rights for OSI Licenses}


\section*{Key Terms}
Because wading through the plethora of open source licenses can be
confusing and feel like reading a foreign language, I have attempted to
clarify a few of the key terms used in open source license, and
discussions about them.

\subsection*{Copyleft}
A copyleft license allows modification and derivitive
works as long at the new work is also realeased under the original
license. The most well known copyleft license is the GPL. 

\subsection*{Permissive}
A permissive license, also commonly referred to as a non-copyleft
license, does not provide any restrictions on how the licensed work can
be combined or redistibuted with other licensed work. Therefore the
licensed work can easily be combined with other free software, open
source software, or even proprietary software. An example of a
permissive license is the MIT License, and a good example of a well
known project that uses the MIT License is Ruby on Rails.

\subsection*{Attribution}
Attribution means that redistributed works and modifications must
attribute, or acknowledge, the creation of the original work to the
original author, and any new authors should be attributed as well if a
derivitive work is created.

\subsection*{Non-Commercial}
A somewhat lesser known term in the licensing of open source software is
the 'non-commercial' clause. These clauses generally state that the
created work can not be used in or for any commercial purposes.
Generally the author means to imply that you can not make any financial
gains off of the software, but you are free to modify it and
redistribute it.


\section*{GPL Compliance}
GPL Compliance means that the code being released under a given license, yet build or linked against GPL'd code, is compliant with the GPL guidelines.

One of the important factors open source projects look at when choosing a license is whether or not the license it GPL Compliant. License compliance is an interesting 


\pagebreak

\nocite{*}

\bibliographystyle{plain}
\bibliography{biblio}
\end{document}

