%%%%%%%
% Author; Trevor Bramwell
% Date: Feb. 29th, 2012
%   Lab 4 writeup for CS372
% TODO: Find good way to include network traces
%%%%%%

\documentclass[12pt,letterpaper]{article}
\usepackage[margin=1in]{geometry}
%\usepackage{alltt}
%\renewcommand{\ttdefault}{txtt}
\usepackage{chngcntr}
\counterwithout{subsection}{section}

\renewcommand{\theenumi}{\alph{enumi}.}
\renewcommand{\labelenumi}{\theenumi}

\usepackage{color}
\newcommand{\red}[1]{\textcolor{red}{#1}}

\usepackage{amsmath}

% Preamble
\title{CS372 Assignment 4}
\author{
    Trevor Bramwell \\
    CS372 - Introduction to Computer Networks \\
    Assignment 4
}
\date{\today}

\begin{document}
\maketitle

\paragraph{P9} Show that the maximum efficiency of pure ALOHA is $1/(2e)$.
\emph{Note}: This problem is easy if you have completed the problem above!

\paragraph{P10} Consider two nodes, A and B, that use the slotted ALOHA 
protocol to contend for a channel. Suppose node A has more data to transmit 
than node B, and node A's retransmission probability $p_{A}$ is greater
than node B's retransmission probability, $p_{B}$.
\begin{enumerate}
\item Provide a formula for node A's average throughput. What is the total
    efficiency of the protocol with these two nodes?
\item If $p_{A} = 2p_{B}$, is node A's average throughput twice as large
    as that of node B? Why or why not? If not, how can you choose $p_{A}$ and 
$p_{B}$ to make that happen?
\item In general, suppose there are $N$ nodes, among which node A has
    retransmission probability $2p$ and all other nodes have retransmission
    probability $p$. Provide expressions to compute the average throughputs of
    node A and of any other node.
\end{enumerate}

\paragraph{P13}
Consider a broadcast channel with $N$ nodes and a transmission rate of $R$ bps.
Supose the braodcast channel uses polling (with an additonal polling node) for
multiple access. Suppose the amount of time from when a node completes 
transmission until the subsequent node is permitted to transmit (that is, the
polling delay) is $d_{p}$. Suppose that within a polling round, a given node is
allowed to transmit at most $Q$ bits. What is the maximum throughput of the
broadcast channel?

\paragraph{P14}
Consider three LANs interconnected by two routers, as shown in Figure 5.38.
\begin{enumerate}
\item Assign IP addresses to all the interfaces. For Subnet 1 use addresses
    of the form 192.168.1.xxx; for Subnet 2 use address of the form 
    192.168.2.xxx; and for Subnet 3 use addresses of the form 192.168.3.xxx.
\item Assign MAC addresses to all the adapters
\item Consider sending an IP datagram from Host E to Host b. Suppose all of
    the ARP tables are up to date. Enumerate all the steps, as done for the
    single-router example in Section 5.4.2.
\item Repeat (c), now assuming that the ARP table in the sending host is empty
    (and the other tables are up to date).
\end{enumerate}
\paragraph{P19}
Explain why a minimum frame size is required for Ethernet. For example, 10Base
Ethernet imposes a minimum frame size constraint of 64 bytes. (If you have done
the previous problem, you might have relized the reason). Now suppose that the
distance between two ends of an Ethernet LAN is $d$. Can you derive a formula
to find the minimum frame size needed for an Ethernet packet? Based on your
reasoning, what is the minimum required packet size for an Ethernet that spans
2 kilometers?

\paragraph{P24}
In this problem you will derive the efficiency of a CSMA/CD-like multiple
access protocol. In this protocol, time is slotted adn all adapters are 
synchronized to the slots. Unlick slotted ALOHA, however, the lenght of a
slot (in seconds) is much less than a frame time (the time to transmit a frame)
. Let S be the lenght of a slot. Suppose all frames are of constant length 
$L = kRS$, where $R$ is the transmission rate of the channel and $k$ is a
large integer. Suppose there are $N$ nodes, each with an infinite number of
frames to send. We also assume that $d_{prop} < S$, so that all nodes can
detect a collision before the end of a slot time. The protocol is as follows:
\begin{itemize}
\item If, for a given slot, no nod has possesion of the channel, all nodes
    contend for the channel; in particular, each node transmits in the slot
    with probability $p$. If exactly one node transmits in the slot, that node
    takes possession of the channel for the subsequent $k-1$ slots and
    transmits its entire frame.
\item If some node has possession of the channel, all other nodes refrain from
    transmitting until the node that possesses the channel has finished
    transmitting its frame. Once this node has transmitted its frame, all nodes
    contend for the channel.
\end{itemize}
Note that the channel alternates between two states: the productive state, 
which lasts exactly $k$ slots, and the nonproductive state, which lasts for
a random number of slots. Clearly, the channel efficiency is the ratio of
$k/(k+x)$, where $x$ is the expected number of consecutive unproductive slots.
\begin{enumerate}
\item For fixed $N$ and $p$, determine the efficiency of this protocol.
\item For fixed $N$, determine the $p$ that maximizes the efficiency.
\item Using the $p$ (which is a function of $N$) found in (b), determine
    the efficiency as $N$ approches infinity.
\end{enumerate}

\paragraph{P25}
Suppose two nodes, A and B, are attached to opposite ends of an 800 m cable, 
and that they each have one frame of 1,500 bits (including all headers and
preambles) to send to each other. Both nodes attempt to transmit at time $t=0$.
Suppose there are four repeaters between A and B, each inserting at 20-bit
delay. Assume the transmission rate is 100 Mbps, and CSMA/CD with backoff
intervals of multiples of 512 bits is used. After the first collision, A draws
$K=0$ and B draws $K=1$ in the exponential backoff protocol. Ignore the jam
signal and the 96-bit time delay.
\begin{enumerate}
\item What is the one-way propagation delay (including repeater delays) between
    A and B in seconds? Assume that the signal propagation speed is $2*10^8$
    m/sec.
\item At what time (in seconds) is A's packet completely delivered at B?
\item Now suppose that only A has a packet to send and that the repeaters are
    replaced with switches. Suppose that each switch has a 20-bit processing
    delay in addition to a store-and-forward delay. At what time, in seconds,
    is A's packet delivered at B?
\end{enumerate}

\paragraph{P27}
Consider Figure 5.38\ in problem P14. Provide MAC addresses and IP addresses
for the interfaces at Host A, both routers, and Host F. Suppose Host A sends a
datagram to Host F. Give the source and destination MAC addresses in the fram 
encapsulating this IP datagram as the frame is transmitted (i) from A to the
left routher, (ii) from the left router to the right router, (iii) from the
right router to F. Also give the source and destination IP addresses in the IP
datagram encapsulated within the frame at each of these points in time.

\end{document}
