%%%%%%%
% Author; Trevor Bramwell
% Date: Mar. 14th, 2012
%   Lab 5 writeup for CS372
% TODO: Find good way to include network traces
%%%%%%

\documentclass[12pt,letterpaper]{article}
\usepackage[margin=1in]{geometry}
%\usepackage{alltt}
%\renewcommand{\ttdefault}{txtt}
\usepackage{listings}
\lstset{
    breaklines=true,
    basicstyle=\footnotesize,
    escapechar=\$
}
\usepackage{chngcntr}
\counterwithout{subsection}{section}

\renewcommand{\b}[1]{\textbf{#1}}

\newcommand{\q}{\textbf{Question} }
\newcommand{\ans}{\textbf{Answer} }

% Preamble
\title{Wireshark Lab 5: Ethernet and ARP}
\author{
    Trevor Bramwell \\
    CS372 - Introduction to Computer Networks \\
    Professor Bechir Hamdaoui
}
\date{\today}

\begin{document}
\maketitle

\section{Capturing and analyzing Ethernet frames}

\begin{enumerate}
\item \q What is the 48-bit Ethernet address of your computer?

\ans 00:22:75:af:71:6b

\item \q What is the 48-bit destination address in the Ethernet frame?
    Is this the Ethernet address of gaia.cs.umass.edu? (Hint: the answer
    is \emph{no}). What device has this as its Ethernet address? [Note:
    this is an important question, and one that students sometimes get wrong.
    Re-read pages 468-469\ in the text and make sure you understand the answer
    here.]

\ans 00:16:cb:c1:5f:50. This is the Ethernet address of my external router.

\item \q Give the hexadecimal value for the two-byte Frame type field. What do
    the bits(s) whose value is 1 mean within the flag field?

\ans 0x0800 means that this frame is of the IP Protocol type.

\item \q How many bytes from the very start of the Ethernet frame does the
    ASCII ``G'' in ``GET'' appear in the Ethernet frame?

\ans 53 bytes from the start.

\item \q What is the hexadecimal value of the CRC field in this Ethernet
    frame?

\ans 0x0d 0x0a 0x0d 0x0a

\pagebreak
\lstinputlisting{packets/get_billofrights.pkt}
\pagebreak

\item \q What is the value of the Ethernet source address? Is this the address
    of your computer, or of gaia.cs.umass.edu (Hint: the answer is \emph{no}).
    What device has this as its Ethernet address?

\ans 00:16:cb:c1:5f:50. It is the address of neither. It the address of my
    extenal router.

\item \q What is the destination address in the Ethernet frame? Is this the
    Ethernet address of your computer?

\ans 00:22:75:af:71:6b. It is the Ethernet address of my wireless usb adapter,
    so yes.

\item \q Give the hexadecimal value for the two-byte Frame type field. What do
    the bit(s) whose value is 1 mean within the flag field?

\ans 0x0800. The bits signify that this is an IP Protocol frame.

\item \q How many bytes from the very start of the Ethernet frame does the
    ASCII ``O'' in ``OK'' (i.e., the HTTP response code) appear in the
    Ethernet frame?

\ans 66 bytes from the start.

\item \q What is the hexadecimal value of the CRC field in this Ethernet frame
    ?

\ans 0x3e 0x0a 0x0d 0x3c

\pagebreak
\lstinputlisting{packets/http_ok.pkt}
\pagebreak


\section{The Address Resolution Protocol}
\item \q Write down the contents of your computer's ARP cache. What is the
    meaning of each column value?

\lstinputlisting{packets/arp}

\begin{description}
    \item[Address] A named reference to the MAC address of the network router.
    \item[HWtype] The type of hardware address.
    \item[HWaddress] The actual hardware address.
    \item[Flags] Flags attached to the address. C means Cached.
    \item[Mask] What this address could be hidden as.
    \item[Iface] The interface on which this address resides.
\end{description}

\item \q What are the hexadecimal values for the source and destination
    addresses in the Ethernet frame containing the ARP request message?

\ans Source: 00:22:75:af:71:6b. Destination: ff:ff:ff:ff:ff:ff.

\item \q Give the hexadecimal value for the two-byte Ethernet Frame type
    field. What do the bit(s) whose value is 1 mean within the flag field?

\ans 0x0806. It specifics that this packet is using the ARP protocol.

\item \q Download the ARP specification.
    \begin{enumerate}
        \item \q How many bytes from the very beginning of the Ethernet frame
            does the ARP \emph{opcode} field begin?

        \ans 21 Bytes after the beginning.

        \item \q What is the value of the \emph{opcode} field within the ARP-
            payload part of the Ethernet frame in which an ARP request is
            made?

        \ans 0x0001 (Request).

        \item \q Does the ARP message contain the IP address of the sender?

        \ans Yes. It is 10.0.1.20.

        \item \q Where in the ARP request does the ``question'' appear -- the
            Ethernet address of the machine whose corresponding IP address
            is being queried?

        \ans Starting at byte 33 and going till byte 39. It comes just after
            the sender Ethernet address and IP address is listed.
    \end{enumerate}

\pagebreak
\lstinputlisting{packets/arp_1.pkt}
\pagebreak

\item Now find the ARP reply that was sent in response to the ARP request.
    \begin{enumerate}
        \item \q How many bytes from the very beginning of the Ethernet frame
            does the ARP \emph{opcode} field begin?

        \ans 21 bytes after the beginning of the Ethernet frame.

        \item \q What is the value of the \emph{opcode} field within the ARP-
            payload part of the Ethernet frame in which an ARP request is
            made?

        \ans 0x0002 (Reply).

        \item \q Where in the ARP message does the ``answer'' to the earlier
            ARP request appear -- the IP address of the machine having the
            Ethernet address of the machine whose corresponding IP address
            is being queried?

        \ans 7 bytes after the \emph{opcode}. The 6 bytes after the \emph{
            opcode} is used for the Ethernet address of the queried machine.
    \end{enumerate}

\item \q What are the hexadecimal values for the source and destination
    addresses in the Ethernet frame containing the ARP reply message?

\ans Source: 00:16:cb:c1:5f:50. Destination: 00:22:75:af:71:6b.

\item \q Why is there no ARP reply (sent in response to the ARP request in packet
    6) in the packet trace?

\ans There is no ARP reply sent because the Ethernet address in the request
    does not match the local machines hardware address.
\end{enumerate}

\pagebreak
\lstinputlisting{packets/arp_2.pkt}
\pagebreak

\section{Extra Credit}
\begin{enumerate}
    \item \q What would happen if, when you manually added an entry, you
        entered the correct IP address, but the wrong Ethernet address
        for that remote interface?
    
    \ans It disables that interface. All outbound requests go nowhere.

    \item \q What is the default amount of time that an entry remains in your
        ARP cache before being removed?

    \ans 60 seconds. It can be found in \emph{/proc/sys/net/ipv4/neigh/wlan0/
        gc\_stale\_time}.

\end{enumerate}


\end{document}
