%%%%%%%
% Author; Trevor Bramwell
% Date: Sat May  5 16:35:13 PDT 2012
%   Lab 3 writeup for CS372
%%%%%%

\documentclass[12pt,letterpaper]{article}
\usepackage[margin=1in]{geometry}

\usepackage{listings}
\lstset{
    breaklines=true,
    basicstyle=\footnotesize,
    escapechar=\$
}
\usepackage{chngcntr}

\usepackage{enumitem}

\renewcommand{\labelenumii}{\arabic{enumii})}

% Custom Commands
\renewcommand{\b}[1]{\textbf{#1}}
\newcommand{\q}{\textbf{Question} }
\newcommand{\ans}{\textbf{Answer} }

% Preamble
\title{Wireshark Lab: TCP}
\author{
    Trevor Bramwell \\
    CS372 - Introduction to Computer Networks \\
    Professor Bechir Hamdaoui
}
\date{\today}

\begin{document}
\maketitle

\setcounter{section}{1}
\section{A first look at the captured trace}
\begin{enumerate}
\item \q What is the IP address and TCP port number used by the client computer
(source) that is transferring the file to gaia.cs.umass.edu? 

\ans 10.0.1.23 port 45679

\item \q What is the IP address of gaia.cs.umass.edu? On what port number is it
sending and receiving TCP segments for this connection?

\ans 128.119.245.12 port 80

\end{enumerate}

\lstinputlisting{packets/2_1.pkt}

\pagebreak
\section{TCP Basics}
% Resume allows the numbering to continue from where it left off.
\begin{enumerate}[resume]
\item \q What is the sequence number of the TCP SYN segment that is used to
initiate the TCP connection between the client computer and gaia.cs.umass.edu?

What is it in the segment that identifies the segment as a SYN segment?  

\ans
\begin{enumerate}
    \item 0
    \item The `Sequence number' field of the TCP Protocol, and the set `Syn'
        flag.
\end{enumerate}

\lstinputlisting{packets/3_4.pkt}

\item \q What is the sequence number of the SYNACK segment sent by
gaia.cs.umass.edu to the client computer in reply to the SYN?  What is the
value of the ACKnowledgement field in the SYNACK segment?  How did
gaia.cs.umass.edu determine that value? What is it in the segment that
identifies the segment as a SYNACK segment? 

\ans
\begin{enumerate}
    \item 0
    \item 1
    \item It is the first ACK back.
    \item The Flags field is set to 0x12 (SYN, ACK)
\end{enumerate}

\lstinputlisting{packets/3_5.pkt}

\item \q What is the sequence number of the TCP segment containing the HTTP
POST command?

\ans 1

\lstinputlisting{packets/3_6.pkt}

\item \q Consider the TCP segment containing the HTTP POST as the first segment
in the TCP connection. What are the sequence numbers of the first six segments
in the TCP connection (including the segment containing the HTTP POST)?  At
what time was each segment sent?  When was the ACK for each segment received?
Given the difference between when each TCP segment was sent, and when its
acknowledgement was received, what is the RTT value for each of the six
segments?  What is the EstimatedRTT value (see page 249 in text) after the
receipt of each ACK?  Assume that the value of the EstimatedRTT is equal to the
measured RTT for the first segment, and then is computed using the EstimatedRTT
equation on page 249 for all subsequent segments.  Note: Wireshark has a nice
feature that allows you to plot the RTT for each of the TCP segments sent.
Select a TCP segment in the “listing of captured packets” window that is being
sent from the client to the gaia.cs.umass.edu server.  Then select:
Statistics->TCP Stream Graph- >Round Trip Time Graph.  

\ans
\begin{enumerate}
    \item 1, 823, 2271, 3719, 5167, 6615
    \item 16:57:14.048263, 16:57:14.048296, 16:57:14.048309, 16:57:14.048320,
        16:57:14.048371, 16:57:14.048415
    \item 16:57:14.470306, 16:57:14.470451, 16:57:14.470921, 16:57:14.471671,
        16:57:14.880228, 16:57:14.880971
    \item 
\end{enumerate}

\item \q What is the length of each of the first six TCP segments?

\ans Each of them are 1448 bytes 

\item \q What is the minimum amount of available buffer space advertised at the
received for the entire trace?  Does the lack of receiver buffer space ever
throttle the sender?  

\item \q Are there any retransmitted segments in the trace file? What did you
check for (in the trace) in order to answer this question?  

\item \q How much data does the receiver typically acknowledge in an ACK?  Can
you identify cases where the receiver is ACKing every other received segment
(see Table 3.2 on page 257 in the text).  

\item \q What is the throughput (bytes transferred per unit time) for the TCP
connection?  Explain how you calculated this value.  
\end{enumerate}


\section{TCP congestion control in action}
\begin{enumerate}[resume]
\item \q Use the Time-Sequence-Graph(Stevens) plotting tool to view the
sequence number versus time plot of segments being sent from the client to the
gaia.cs.umass.edu server.  Can you identify where TCP’s slowstart phase begins
and ends, and where congestion avoidance takes over?  Comment on ways in which
the measured data differs from the idealized behavior of TCP that we’ve studied
in the text. 

\end{enumerate}

\end{document}
