\title{Wireshark Lab 3}
\author{
    Trevor Bramwell \\
    CS372 - Introduction to Computer Networks \\
    Wireshark Lab 3: TCP
}
\date{\today}

\documentclass[12pt]{article}

\begin{document}
\maketitle

\section{A first look at the capture trace}

\subsection{}
See 

\subsection{}
What is the IP address of gaia.cs.umass.edu? On what port number is it sending 
and receiving TCP segments for this connection?

10.0.1.25:43535

\tiny
\input{2_1.out}
\normalsize

\subsection{}
What is the IP address and TCP port number used by your client computer 
(source) to transfer the file to gaia.cs.umass.edu?

128.119.245.12:80

\section{TCP Basics}
\subsection{4}
\textbf{Question:}What is the sequence number of the TCP SYN segment that is used to initiate the
TCP connection between the client computer and gaia.cs.umass.edu? What is it
in the segment that identifies the segment as a SYN segment?

\textbf{Answer:}The sequence number is 0. The flags are set to 0x02 [0000 0000 0010].

\subsection{5}
What is the sequence number of the SYNACK segment sent by gaia.cs.umass.edu
to the client computer in reply to the SYN? What is the value of the
ACKnowledgement field in the SYNACK segment? How did gaia.cs.umass.edu
determine that value? What is it in the segment that identifies the segment \\
as a SYNACK segment?

The sequence number of the SYNACK segment is 0. The ACKnowledgement field is \\
set to 1. gaia.cs.umass.edu determined that value from it recieving a packet.\\
The flags 0x12 identify the segment as a SYNACK segment.

\subsection{6}
What is the sequence number of the TCP segment containing the HTTP POST
command? Note that in order to find the POST command, you’ll need to dig into
the packet content field at the bottom of the Wireshark window, looking for a
segment with a “POST” within its DATA field.

\subsection{7}
Consider the TCP segment containing the HTTP POST as the first segment in the
TCP connection. What are the sequence numbers of the first six segments in the
TCP connection (including the segment containing the HTTP POST)? At what
time was each segment sent? When was the ACK for each segment received?
Given the difference between when each TCP segment was sent, and when its
acknowledgement was received, what is the RTT value for each of the six
segments? What is the EstimatedRTT value (see page 249 in text) after the
receipt of each ACK? Assume that the value of the EstimatedRTT is equal to
the measured RTT for the first segment, and then is computed using the
EstimatedRTT equation on page 249 for all subsequent segments.
Note: Wireshark has a nice feature that allows you to plot the RTT for
each of the TCP segments sent. Select a TCP segment in the “listing of
captured packets” window that is being sent from the client to the
gaia.cs.umass.edu server. Then select: Statistics->TCP Stream Graph-
>Round Trip Time Graph.

\subsection{8}
What is the length of each of the first six TCP segments?

\subsection{9}
What is the minimum amount of available buffer space advertised at the received
for the entire trace? Does the lack of receiver buffer space ever throttle the
sender?

\subsection{10}
Are there any retransmitted segments in the trace file? What did you check for (in
the trace) in order to answer this question?

\subsection{11}
How much data does the receiver typically acknowledge in an ACK? Can you
identify cases where the receiver is ACKing every other received segment (see
Table 3.2 on page 257 in the text).

\subsection{12}
What is the throughput (bytes transferred per unit time) for the TCP connection?
Explain how you calculated this value.

\section{TCP congestion control in action}
\subsection{13} 
Use the Time-Sequence-Graph(Stevens) plotting tool to view the sequence
number versus time plot of segments being sent from the client to the
gaia.cs.umass.edu server. Can you identify where TCP’s slowstart phase begins
and ends, and where congestion avoidance takes over? Comment on ways in
which the measured data differs from the idealized behavior of TCP that we’ve
studied in the text.

\subsection{14}
Answer each of two questions above for the trace that you have gathered when
you transferred a file from your computer to gaia.cs.umass.edu

\end{document}
