%%%%%%%
% Author; Trevor Bramwell
% Date: Mon Apr  9 20:35:22 PDT 2012
%   Lab 1 writeup for CS372
%%%%%%

\documentclass[12pt,letterpaper]{article}
\usepackage[margin=1in]{geometry}

\usepackage{listings}
\lstset{
    breaklines=true,
    basicstyle=\footnotesize,
    escapechar=\$
}
\usepackage{chngcntr}
%\counterwithout{subsection}{section}

% Custom Commands
\renewcommand{\b}[1]{\textbf{#1}}
\newcommand{\q}{\textbf{Question} }
\newcommand{\ans}{\textbf{Answer} }

% Preamble
\title{Wireshark Lab 1: Getting Started}
\author{
    Trevor Bramwell \\
    CS372 - Introduction to Computer Networks \\
    Professor Bechir Hamdaoui
}
\date{\today}

\begin{document}
\maketitle

\begin{enumerate}
\item \q List the different protocols that appear in the protocol column in 
    the unfiltered 
    packet-listing window in step 7 above. 

\ans 
\begin{itemize}
    \item DNS
    \item HTTP
    \item TCP
\end{itemize}

\item \q How long did it take from when the HTTP GET message was sent until
    the HTTP OK reply was received? (By default, the value of the Time column
    in the packetlisting window is the amount of time, in seconds, since
    Wireshark tracing began. To display the Time field in time-of-day format,
    select the Wireshark View pull down menu, then select Time Display Format,
    then select Time-of-day.)

\ans 0.2893670 milliseconds

\item \q What is the Internet address of the gaia.cs.umass.edu (also known \\
as wwwnet.cs.umass.edu)?  What is the Internet address of your computer? 

\ans Address of gaia.cs.umass.edu: 128.119.245.12 \\
     Address of my computer: 10.0.1.20

\item \q Print the two HTTP messages displayed in step 9 above. To do so,
    select Print from the Wireshark File command menu, and select ``Selected
    Packet Only`` and ``Print as displayed`` and then click OK.

\pagebreak
\lstinputlisting{packets/get.pkt}
\pagebreak
\lstinputlisting{packets/ok.pkt}
%\pagebreak

\end{enumerate}
\end{document}
