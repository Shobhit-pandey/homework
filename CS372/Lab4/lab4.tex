%%%%%%%
% Author; Trevor Bramwell
% Date: Wed May 23 22:02:23 PDT 2012
%   Lab 4 writeup for CS372
%%%%%%

\documentclass[12pt,letterpaper]{article}
\usepackage[margin=1in]{geometry}

\usepackage{listings}
\lstset{
    breaklines=true,
    basicstyle=\footnotesize,
    escapechar=\$
}
\usepackage{chngcntr}

\usepackage{enumitem}

\renewcommand{\labelenumii}{\arabic{enumii})}

% Custom Commands
\renewcommand{\b}[1]{\textbf{#1}}
\newcommand{\q}{\textbf{Question} }
\newcommand{\ans}{\textbf{Answer} }

% Preamble
\title{Wireshark Lab: IP}
\author{
    Trevor Bramwell \\
    CS372 - Introduction to Computer Networks \\
    Professor Bechir Hamdaoui
}
\date{\today}

\begin{document}
\maketitle

\setcounter{section}{1}
\section{A look at the capture trace}
\begin{enumerate}
% 1
\item \q Select the first ICMP Echo Request message sent by your
computer, and expand the Internet Protocol part of the packet in the packet
details window. What is the IP address of your computer?

\ans 192.168.1.100
\end{enumerate}

\lstinputlisting{packets/2_1.pkt}
\pagebreak

\begin{enumerate}[resume]
% 2
\item \q Within the IP packet header, what is the value in the
upper layer protocol field?

\ans ICMP (1)

% 3
\item \q How many bytes are in the IP header? 
How many bytes are in the payload \emph{of the IP datagram}?
Explain how you determined the number of payload bytes.

\ans The IP header contains 20 bytes, the payload of the IP
datagram contains 64 bytes.
The payload bytes can be determined by selecting the payload
header (Internet Control Message Protocol) and counting the number of bytes
that are selected in the raw packet window.

% 4
\item \q Has this IP datagram been fragmented? Explain how you 
determined whether or not the datagram has been fragmented.

\ans No. The fragment offset is set to 0.

% 5
\item \q Which fields in the IP datagram \emph{always} change from
one datagram to the next within this series of ICMP messages sent by your
computer? 

\ans Identification, Header Checksum, Time to Live, Source and
Destination Port.

% 6
\item \q Which fields stay constant? Which of the fields \emph{must} stay constant? Which fields
must change? Why?

\ans Source and Destination stay constant. These much stay
constant while Time to Live changes because we are only interested in the route
between our host and gaia.cs.umass.edu.

% 7
\item \q Describe the pattern you see in the values in the
Identification field of the IP datagram

\ans It increases by 1 every time.

\pagebreak
% 8
\item \q What is the value in the Identification field and the TTL
field?

\ans Identification: 0x0169 (361) and TTL: 64
\end{enumerate}

\lstinputlisting{packets/2_8.pkt}

\begin{enumerate}[resume]
% 9
\item \q Do these values remain unchanged for all of the ICMP
TTL-exceeded replies sent to your computer by the nearest (first hop) router?
Why?

\ans The TTL does not change from 64, but each Identification
number does.
\end{enumerate}

\pagebreak
\subsection*{Fragmentation}
\begin{enumerate}[resume]
% 10
\item \q
Find the first ICMP Echo Request message that was sent by your computer after
you changed the \emph{Packet Size} in \emph{pingplotter} to be 2000. Has that message been
fragmented across more than one IP datagram?

\ans Yes it has as indicated by the [2 IPv4 Fragments ...] item.
\end{enumerate}

\lstinputlisting{packets/2_10.pkt}

\pagebreak
\begin{enumerate}[resume]
% 11
\item \q
Print out the first fragment of the fragmented IP datagram. What information in
the IP header indicates that the datagram been fragmented? What information in
the IP header indicates whether this is the first fragment versus a latter fragment?
How long is this IP datagram?

\ans The flags have been set to 0x01 (More Fragments). This is
the first fragment because the fragment offset is set to 0. This IP datagram
is 1480 bytes
\end{enumerate}

\lstinputlisting{packets/2_10-11.pkt}

\pagebreak
\begin{enumerate}[resume]
% 12
\item \q
Print out the second fragment of the fragmented IP datagram. What information in
the IP header indicates that this is not the first datagram fragment? Are the more
fragments? How can you tell?

\ans This is not the first datagram fragment because the fragment
offset is set to 1480. There are no more fragments because the flags are set to
0x00. If there were more fragments the flags would be set to 0x01 as in the
first fragment.
\end{enumerate}

\lstinputlisting{packets/2_12.pkt}

\pagebreak
\begin{enumerate}[resume]
% 13
\item \q
What fields change in the IP header between the first and second fragment?

\ans Total Length, Flags, Identification, and Header Checksum.

% 14
\item \q
How many fragments were created from the original datagram?

\ans 3
\end{enumerate}

\lstinputlisting{packets/2_13_1.pkt}
\lstinputlisting{packets/2_13_2.pkt}
\lstinputlisting{packets/2_13_3.pkt}

\begin{enumerate}[resume]
% 15
\item \q
What fields change in the IP header among the fragments?

\ans Total Length, Flags, Identification, and Header Checksum.
\end{enumerate}
\end{document}
