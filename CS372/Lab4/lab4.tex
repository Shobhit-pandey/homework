%%%%%%%
% Author; Trevor Bramwell
% Date: Feb. 29th, 2012
%   Lab 4 writeup for CS372
% TODO: Find good way to include network traces
%%%%%%

\documentclass[12pt,letterpaper]{article}
\usepackage[margin=1in]{geometry}
\usepackage{chngcntr}
\counterwithout{subsection}{section}

% Preamble
\title{Wireshark Lab 4}
\author{
    Trevor Bramwell \\
    CS372 - Introduction to Computer Networks \\
    Wireshark Lab 3: TCP
}
\date{\today}

\begin{document}
\maketitle

\setcounter{section}{1}
\section{A look at the capture trace}
\subsection{}
\paragraph{Question} Select the first ICMP Echo Request message sent by your
computer, and expand the Internet Protocol part of the packet in the packet
details window. What is the IP address of your computer?
\paragraph{Answer} 192.168.1.100

\subsection{}
\paragraph{Question} Within the IP packet header, what is the value in the
upper layer protocol field?
\paragraph{Answer} UDP (17)

\subsection{}
\paragraph{Question} How many bytes are in the IP header? 
\paragraph{Answer} 20
\paragraph{Question} How many bytes are in the payload of the IP datagram?
\paragraph{Answer} 38
\paragraph{Question} Explain how you determined the number of payload bytes.
\paragraph{Answer}

\subsection{}
\paragraph{Question} Has this IP datagram been fragmented? Explain how you 
determined whether or not the datagram has been fragmented.
\paragraph{Answer}

\subsection{}
\paragraph{Question}
\paragraph{Answer}

\subsection{}
\paragraph{Question}
\paragraph{Answer}

\subsection{}
\paragraph{Question}
\paragraph{Answer}

\subsection{}
\paragraph{Question}
\paragraph{Answer}

\subsection{}
\paragraph{Question}
\paragraph{Answer}

\subsection{}
\paragraph{Question}
\paragraph{Answer}

\subsection{}
\paragraph{Question}
\paragraph{Answer}

\subsection{}
\paragraph{Question}
\paragraph{Answer}

\subsection{}
\paragraph{Question}
\paragraph{Answer}

\tiny
%\input{2_1.out}
\normalsize
\end{document}
