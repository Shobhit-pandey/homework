\documentclass[letterpaper,10pt]{article}

\usepackage{graphicx}                                        

\usepackage{amssymb}                                         
\usepackage{amsmath}                                         
\usepackage{amsthm}                                          

\usepackage{alltt}                                           
\usepackage{float}
\usepackage{color}

\usepackage{url}

\usepackage{balance}
\usepackage[TABBOTCAP, tight]{subfigure}
\usepackage{enumitem}

\usepackage{pstricks, pst-node}

\usepackage{geometry}
\geometry{textheight=10in, textwidth=7.5in}

% random comment

\newcommand{\cred}[1]{{\color{red}#1}}
\newcommand{\cblue}[1]{{\color{blue}#1}}

\usepackage{hyperref}

\usepackage{textcomp}
\usepackage{listings}

\def\name{Trevor Bramwell}

%% The following metadata will show up in the PDF properties
\hypersetup{
  colorlinks = true,
  urlcolor = black,
  pdfauthor = {\name},
  pdfkeywords = {cs311 ``operating systems'' homework},
  pdftitle = {CS 311 Homework 1},
  pdfsubject = {CS 311 Homework 1},
  pdfpagemode = UseNone
}

\parindent = 0.0 in
\parskip = 0.2 in

\title{CS311 - FA13: Homework 1}
\date{\today}
\author{\name}

\begin{document}
\maketitle

\begin{enumerate}
\item Describe at least 2 ways of transferring files from a remote server to a local machine.
    \begin{enumerate}
        \item \texttt{\$ scp user@remotehost:filepath localpath}
        \item \texttt{\$ \{wget,curl -L\} URL}
    \end{enumerate}

\item What are revision control systems? Why are they useful? Explain how to create a subversion or git repository on os-class (and create it, while you're at it).

A `revision control system' is a program from keeping track of revisions
to files. They are useful in software development for keeping track of
the history of a project and keeping code between developers in sync.

    \begin{enumerate}
        \item \texttt{\$ svnadmin create}
        \item \texttt{\$ git init}
    \end{enumerate}

\item What is the difference between redirecting and piping? Describe each.

Redirecting on the shell is a way to connect stdin, stdout, and stderr
of one command to a file descriptor. For example:
\texttt{\$ echo "hello, world" > newfile} would connect the echo
command's stdout to newfile's stdin. 

Piping is shorthand for connecting fd 1 (stdout) to fd 0 (stdin), and
runs each command as a subprocess.

\item What is make, and how is it useful?

Parphrase from the manpage: ``Make is a utility to maintain groups of
programs.''. In short, Make is a build tool for managing program
dependencies. When not being used for the C language Make is generally
used to hold a collection of related shell commands, like how this class
is currently using it to build LaTeX documents.

\item Describe, in detail, the syntax of a make file.

A makefile is made up of 5 elements:
    \begin{enumerate}
        \item target rules
        \item inference rules
        \item macros
        \item empty lines
        \item comments
    \end{enumerate}

Since empty lines and comments are self exlanitory, I will skip over
them.

Macros

Target Rules

Inference Rules


\item Give a find command that will run the file command on every regular file (not directories!) within the current filesystem subtree.

\texttt{\$ find . -type f -exec file `\{\}' \textbackslash;}

\end{enumerate}

\end{document}
