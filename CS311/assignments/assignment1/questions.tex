\documentclass[letterpaper,10pt]{article}

\usepackage{graphicx}                                        

\usepackage{amssymb}                                         
\usepackage{amsmath}                                         
\usepackage{amsthm}                                          

\usepackage{alltt}                                           
\usepackage{float}
\usepackage{color}

\usepackage{url}

\usepackage{balance}
\usepackage[TABBOTCAP, tight]{subfigure}
\usepackage{enumitem}

\usepackage{pstricks, pst-node}

\usepackage{geometry}
\geometry{textheight=10in, textwidth=7.5in}

% random comment

\newcommand{\cred}[1]{{\color{red}#1}}
\newcommand{\cblue}[1]{{\color{blue}#1}}

\usepackage{hyperref}

\usepackage{textcomp}
\usepackage{listings}

\def\name{Trevor Bramwell}

%% The following metadata will show up in the PDF properties
\hypersetup{
  colorlinks = true,
  urlcolor = black,
  pdfauthor = {\name},
  pdfkeywords = {cs311 ``operating systems'' homework},
  pdftitle = {CS 311 Homework 1},
  pdfsubject = {CS 311 Homework 1},
  pdfpagemode = UseNone
}

\parindent = 0.0 in
\parskip = 0.2 in

\title{CS311 - FA13: Homework 1}
\date{\today}
\author{\name}

\begin{document}
\maketitle

\begin{enumerate}
\item Describe at least 2 ways of transferring files from a remote
      server to a local machine.
    \begin{enumerate}
        \item \texttt{\$ scp user@remotehost:filepath localpath}
        \item \texttt{\$ \{wget,curl -L\} URL}
    \end{enumerate}

\item What are revision control systems? Why are they useful? Explain
      how to create a subversion or git repository on os-class (and create
      it, while you're at it).

A `revision control system' is a program from keeping track of revisions
to files. They are useful in software development for keeping track of
the history of a project and keeping code between developers in sync.

    \begin{enumerate}
        \item \texttt{\$ svnadmin create}
        \item \texttt{\$ git init}
    \end{enumerate}

\item What is the difference between redirecting and piping? Describe each.

Redirecting allows you to redirect a command's input or output to a
file. Since stdin, stdout, and stderr are also files,
they are include in what you can redirect to.

For example:
    \texttt{\$ echo "hello, world" > newfile}
would redirect echo's stdout to newfile. newfile would now contain
``hello, world''.

Piping connects the stdout of a command to the stdin of a second
command by way of a pipe, before any redirection happens. Piping sets up
a link between two commands so the output of the first can easily be
chained to the input of the second.

\item What is make, and how is it useful?

Make is a language and a program for handling build dependencies.
It is useful for compiling only the parts of a program that have
changed. For example: say you have a program that is dependend upon
functions from multiple files of `C' code. If one of those files
changes, then you will have to recompile every each one of the files
along with your program. 

Make resolves this by only recompiling the single file that changed, and
building any other dependencies that it relies on.

Make can also be used as a simple interface for a collection of
commands; similar to bash aliases.

\item Describe, in detail, the syntax of a make file.

A makefile is made up of 2 major elements: Rules and Macros. They can
also contain comments and empty lines.

A makefile generally has a list of macros at the top, followed by any
number of rules.

Macros are used to easily reference things like CFLAGS and LDFLAGS, and
the list of objects to be compiled.

Each rule consists of a target, some optional dependencies, and a list
of 1 or more commands to execute in order. If make sees a dependency is
out of date when running a target, it will run the command to rebuild
that dependency. 

\item Give a find command that will run the file command on every
      regular file (not directories!) within the current filesystem
      subtree.

\texttt{\$ find . -type f -exec file `\{\}' \textbackslash;}

\end{enumerate}

\end{document}
