\documentclass[letterpaper,10pt]{article}

\usepackage{graphicx}                                        

\usepackage{amssymb}                                         
\usepackage{amsmath}                                         
\usepackage{amsthm}                                          

\usepackage{alltt}                                           
\usepackage{float}
\usepackage{color}

\usepackage{url}

\usepackage{balance}
\usepackage[TABBOTCAP, tight]{subfigure}
\usepackage{enumitem}

\usepackage{pstricks, pst-node}

\usepackage{geometry}
\geometry{textheight=10in, textwidth=7.5in}

\usepackage{hyperref}

\usepackage{textcomp}
\usepackage{listings}
\usepackage{verbatim}

\title{CS311 - FA13: Archive}
\date{\today}
\author{Trevor Bramwell}

% Setup the PDF Metadata
\hypersetup{
  colorlinks = true,
  urlcolor = black,
  pdfauthor = {Trevor Bramwell},
  pdfkeywords = {},
  pdftitle = {CS311 - FA13: Archive},
  pdfsubject = {},
  pdfpagemode = UseNone
}

\parindent = 0.0 in
\parskip = 0.2 in

\begin{document}
\maketitle

\section{Design}
    This assignment is to create a simple implementation of the UNIX
    utility \emph{ar}. The utility will support the operations on archive files
    of appending, creating, extracting, listing and deleting. It will
    also provide a flag for verbose output and if time permits, a way to
    add modified files, after a delay, from a directory.

    Since this \emph{ar} revolves around the previously mentioned
    operations, I will be structuring the code in the same manner.
    Depending on the flag passed in, the program will call the
    corresponding function to perform the operation. A set of utility
    functions will be created as well to reduce code duplication, and
    improve code in a more self-documenting way.

    When I start writing my program, I will begin with the user
    interface, that is the command line. Having the interface created
    first will make it easier for me to focus on the functionality of
    the program.

    I will then walk though each operation that can happen document the
    steps it needs to take, i.e. open archive file, write file, etc. to
    flesh out what the utility functions might be. I will then focus on
    the single operation of \emph{listing}, and use a pre-created
    archive file so as to reduce the number of concerns. From there I
    will itertively enable each flag until they all work and the program
    works as intended.

\subsection{Versions}

Up to 0.5 GNU ar will be used for testing.

\begin{itemize}
    \item 0.0 Tests (all flags, all flags w/verbose)
    \item 0.1 List (-t)
    \item 0.2 Extract (-x)
    \item 0.3 Append (-q)
    \item 0.4 Quick Append (-A)
    \item 0.5 Create Archive when using (-q, -A)
    \item 0.6 Delete (-d)
    \item 1.0 Verbose (-v) - Change all printf statements to only execute when flag
          passed.
    \item 1.1 Timeout (-w)
\end{itemize}

\subsection{Deviation}
    After getting through appending and table, I realized verbosity was
    a simple if check away, and added it. Instead of moving onto extract
    and delete right way, I stuck with append and table until they were
    complete and free of bugs. I did not put aside enough time to
    complete extract and delete.

\section{Questions}
\begin{description}
  \item  What do you think the main point of this assignment is?

         The main point of this assignment is to get us reaquanted with
         the C programming language, and to learn UNIX system calls.

  \item  How did you ensure your solution was correct? Testing details, for
         instance. This section should be very thorough.

         Originally I planned to check my program by using the python
         unittest library and subprocess modules. I could tell from the
         initial work on this that it would turn into a project on it's
         own and abandoned it in favor of manual testing.
         
         The majority of my problems came from the even/odd padding in
         the ar program. It was a while before I realized I had been
         modding the size by 3 instead of 2 to check if the size was
         even or odd.

         Most of my manual testing involved having a base archive
         pre-created with `ar -cq' and then running my program off of
         that.

  \item  What did you learn?
        
         I learned that I am not as strong of a C programmer as I
         though. A majority of challenges I encountered dealted with
         converting between types. I became familiar with sscanf and
         sprintf, but I still do not feel entirely comfortable with
         converting types in C.

         I also learned how to use open, write, read, getopt, and became
         more familiar with C.
\end{description}

\newpage

\section{Work Log}

\verbatiminput{work-log.txt}

\end{document}
