\documentclass[letterpaper,10pt]{article}

\usepackage{graphicx}                                        

\usepackage{amssymb}                                         
\usepackage{amsmath}                                         
\usepackage{amsthm}                                          

\usepackage{alltt}                                           
\usepackage{float}
\usepackage{color}

\usepackage{url}

\usepackage{balance}
\usepackage[TABBOTCAP, tight]{subfigure}
\usepackage{enumitem}

\usepackage{pstricks, pst-node}

\usepackage{geometry}
\geometry{textheight=10in, textwidth=7.5in}

\usepackage{hyperref}

\usepackage{textcomp}
\usepackage{listings}

\title{CS311 - FA13: Archive}
\date{\today}
\author{Trevor Bramwell}

% Setup the PDF Metadata
\hypersetup{
  colorlinks = true,
  urlcolor = black,
  pdfauthor = {\name},
  pdfkeywords = {},
  pdftitle = {\title},
  pdfsubject = {},
  pdfpagemode = UseNone
}

\parindent = 0.0 in
\parskip = 0.2 in

\begin{document}
\maketitle

\section{Design}
    a design for your system

    Revolves around reading and writing to a single archive file. Takes in
    multiple files when adding.

    Has the Operations:
        Append
        Extract
        List
        Remove

    Remove is an Extract w/Delete, Extract should take a flag to also remove
    the file.

    Append takes the directory, or a list of files. Version that takes a
    directory will pass in all regular files in the directory to the version
    that takes a list.
        
    fuctions:
        open_archive
        to_archive

    Creating:
        open a new file (argv[1]), 

\subsection{Deviation}
    places your implementation deviated from this design

\section{Challenge}
    any challenges you overcame in completing this assignment

\section{Questions}
\begin{description}
  \item  What do you think the main point of this assignment is?
  \item  How did you ensure your solution was correct? Testing details, for
         instance. This section should be very thorough.
  \item  What did you learn?
\end{description}

\section{Work Log}
    a work log, detailing what you did when -- this can fairly easily be
    created if you are using some form of revision control

\end{document}
