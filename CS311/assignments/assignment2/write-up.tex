\documentclass[letterpaper,10pt]{article}

\usepackage{graphicx}                                        

\usepackage{amssymb}                                         
\usepackage{amsmath}                                         
\usepackage{amsthm}                                          

\usepackage{alltt}                                           
\usepackage{float}
\usepackage{color}

\usepackage{url}

\usepackage{balance}
\usepackage[TABBOTCAP, tight]{subfigure}
\usepackage{enumitem}

\usepackage{pstricks, pst-node}

\usepackage{geometry}
\geometry{textheight=10in, textwidth=7.5in}

\usepackage{hyperref}

\usepackage{textcomp}
\usepackage{listings}

\title{CS311 - FA13: Archive}
\date{\today}
\author{Trevor Bramwell}

% Setup the PDF Metadata
\hypersetup{
  colorlinks = true,
  urlcolor = black,
  pdfauthor = {\name},
  pdfkeywords = {},
  pdftitle = {\title},
  pdfsubject = {},
  pdfpagemode = UseNone
}

\parindent = 0.0 in
\parskip = 0.2 in

\begin{document}
\maketitle

\section{Design}
    This assignment is to create a simple implementation of the UNIX
    utility \emph{ar}. The utility will support the operations on archive files
    of appending, creating, extracting, listing and deleting. It will
    also provide a flag for verbose output and if time permits, a way to
    add modified files, after a delay, from a directory.

    Since this \emph{ar} revolves around the previously mentioned
    operations, I will be structuring the code in the same manner.
    Depending on the flag passed in, the program will call the
    corresponding function to perform the operation. A set of utility
    functions will be created as well to reduce code duplication, and
    improve code in a more self-documenting way.

    When I start writing my program, I will begin with the user
    interface, that is the command line. Having the interface created
    first will make it easier for me to focus on the functionality of
    the program.

    I will then walk though each operation that can happen document the
    steps it needs to take, i.e. open archive file, write file, etc. to
    flesh out what the utility functions might be. I will then focus on
    the single operation of \emph{listing}, and use a pre-created
    archive file so as to reduce the number of concerns. From there I
    will itertively enable each flag until they all work and the program
    works as intended.

\subsection{Notes}
    Remove is an Extract w/Delete, Extract should take a flag to also remove
    the file.

    Append takes the directory, or a list of files. Version that takes a
    directory will pass in all regular files in the directory to the version
    that takes a list.
        
\subsection{Deviation}
    places your implementation deviated from this design

\section{Challenge}
    any challenges you overcame in completing this assignment

\section{Questions}
\begin{description}
  \item  What do you think the main point of this assignment is?

         The main point of this assignment is to get us reaquanted with
         the C programming language, and to learn UNIX system calls.

  \item  How did you ensure your solution was correct? Testing details, for
         instance. This section should be very thorough.

         I ensured my program was correct by writing a suite of
         interface tests. For example: I test for the listing
         functionality by having a pre-determined list of files in an
         archive, running my \emph{ar}, and ensuring the returned list
         matches my pre-determined one.

  \item  What did you learn?
\end{description}

\section{Work Log}
    a work log, detailing what you did when -- this can fairly easily be
    created if you are using some form of revision control

\end{document}
