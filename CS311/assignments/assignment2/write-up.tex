\documentclass[letterpaper,10pt]{article}

\usepackage{graphicx}                                        

\usepackage{amssymb}                                         
\usepackage{amsmath}                                         
\usepackage{amsthm}                                          

\usepackage{alltt}                                           
\usepackage{float}
\usepackage{color}

\usepackage{url}

\usepackage{balance}
\usepackage[TABBOTCAP, tight]{subfigure}
\usepackage{enumitem}

\usepackage{pstricks, pst-node}

\usepackage{geometry}
\geometry{textheight=10in, textwidth=7.5in}

\usepackage{hyperref}

\usepackage{textcomp}
\usepackage{listings}

\title{CS311 - FA13: Archive}
\date{\today}
\author{Trevor Bramwell}

% Setup the PDF Metadata
\hypersetup{
  colorlinks = true,
  urlcolor = black,
  pdfauthor = {\name},
  pdfkeywords = {},
  pdftitle = {\title},
  pdfsubject = {},
  pdfpagemode = UseNone
}

\parindent = 0.0 in
\parskip = 0.2 in

\begin{document}
\maketitle


\section{Design}
    The following is the plan of execution I will take when tackling this
    problem.

    myar revolves around the following operations:
    \begin{itemize}
        \item Append
        \item Extract
        \item List
        \item Remove
    \end{itemize}

    

\subsection{Versions}

Up to 0.x GNU ar will be used for testing.

    0.0 Tests (all flags, all flags w/verbose)
    0.1 List (-t)
    0.x Extract (-x)
    0.x Append (-q)
    0.x Quick Append (-A)
    0.x Create Archive when using (-q, -A)
    0.x Delete (-d)
    1.0 Verbose (-v) - Change all printf statements to only execute when flag
                       passed.
    1.1 Timeout (-w)
        
\subsection{Notes}
    Remove is an Extract w/Delete, Extract should take a flag to also remove
    the file.

    Append takes the directory, or a list of files. Version that takes a
    directory will pass in all regular files in the directory to the version
    that takes a list.

    fuctions:
        open_archive
        to_archive

\subsection{Deviation}
    places your implementation deviated from this design

\section{Challenge}
    any challenges you overcame in completing this assignment

\section{Questions}
\begin{description}
  \item  What do you think the main point of this assignment is?
         
         To get reaquainted with C and to become familiar with system
         calls.

  \item  How did you ensure your solution was correct? Testing details, for
         instance. This section should be very thorough.
  \item  What did you learn?
\end{description}

\section{Work Log}
    a work log, detailing what you did when -- this can fairly easily be
    created if you are using some form of revision control

\newpage

\section*{Instructions}
Write a C program called myar.c. This program will illustrate the use of file
I/O on UNIX by maintaining a UNIX archive library, in the standard archive
format.

For this assignment, the following is the syntax your program must support:

myar [options] archive [member...]

where archive is the name of the archive file to be used, and options is one of
the following options (all are silent unless verbose is specified):

-v  iff specified with other options, print a verbose version of the output
-q  quickly append named files to archive    
-x  extract named files      
-t  print a concise table of contents of the archive     
-d  delete named files from archive      
-A  quickly append all ``regular'' files in the current directory   (except the
    archive itself)

-w  Extra credit: for a given timeout, add all modified files to the archive.
    (except the archive itself)

You must use getopt to parse the command line options (and only getopt -- do
not use getopt_long).

The archive file maintained must use exactly the standard format defined in
/usr/inc1ude/ar.h, and in fact may be tested with archives created with the ar
command. The options listed above are compatible with the options having the
same name in the ar command. -A is a new option not in the usual ar command.

Notes:

    For the -q command myar should create an archive file if it doesn't exist,
    using permissions ``666''. For the other commands myar reports an error if
    the archive does not exist, or is in the wrong format.

    You will have to use the system calls stat and utime to properly deal with
    extracting and restoring the proper timestamps. Since the archive format
    only allows one timestamp, store the mtime and use it to restore both the
    atime and mtime. Permissions should also be restored to the original value,
    subject to umask limitation.

    The -q and -A commands do not check to see if a file by the chosen name
    already exists. It simply appends the files to the end of the archive.

    The -x and -d commands operate on the first file matched in the archive,
    without checking for further matches.

    In the case of the -d option, you will have to build a new archive file to
    recover the space. Do this by unlinking the original file after it is
    opened, and creating a new archive with the original name.

    You are required to handle multiple file names as arguments.

    Since file I/O is expensive, do not make more than one pass through the
    archive file, an issue especially relevant to the multiple delete case.

    Compilation will be done with icc and use (at minimum) the following flags:
    '-std=c99 -Wall'.

    For the -w flag, the command will take as long as specified by the timeout
    argument. You should print out a status message upon adding a new file only
    if -v is specified. This may result in many different copies of the same
    file in the archive.

    For extra credit, any time a file is added that already exists, remove the
    old copy from the archive, but ONLY if it is not the same. If identical, do
    not add the new file. Size is NOT a valid indication of isomorphism. You
    will need to use a more advanced technique.

\subsection*{Writeup Instructions}

Your write-up should include (at minimum) the following:

    a design for your system, as well as places your implementation deviated
    from this design

    a work log, detailing what you did when -- this can fairly easily be
    created if you are using some form of revision control

    any challenges you overcame in completing this assignment

    answers to the following questions:
        What do you think the main point of this assignment is?

        How did you ensure your solution was correct? Testing details, for
        instance. This section should be very thorough.

        What did you learn?

\subsection*{Deliverable}

Things to turn in:

    C source code
    write-up, as a tex document
    any support files necessary to compile your tex document or source
    a makefile to build both source code and your tex file

You are not allowed to use Lyx to create the document. You must create the
document "by hand", instead.

\end{document}
