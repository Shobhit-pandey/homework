\documentclass[letterpaper,10pt]{article}

\usepackage{alltt}                                           
\usepackage{float}
\usepackage{color}

\usepackage{url}

\usepackage{balance}
\usepackage{enumitem}

\usepackage{pstricks, pst-node}

\usepackage{geometry}
\geometry{textheight=10in, textwidth=7.5in}

\usepackage{hyperref}

\usepackage{textcomp}
\usepackage{listings}
\usepackage{verbatim}

\title{CS311 - FA13: Happy Primes}
\date{\today}
\author{Trevor Bramwell}

% Setup the PDF Metadata
\hypersetup{
  colorlinks = true,
  urlcolor = black,
  pdfauthor = {Trevor Bramwell},
  pdfkeywords = {},
  pdftitle = {CS311 - FA13: Happy Primes},
  pdfsubject = {},
  pdfpagemode = UseNone
}

\parindent = 0.0 in
\parskip = 0.2 in

\begin{document}
\maketitle

\section{Design}
The first step I will take in this assignment will be to convert my
sieve of eratosthenes to use a bitmap instead of an array. I will then
refactor the main loop to be a single function which will become the
thread function. I will then write functions for dealing with a bitmap,
specifically set\_bit, and clear\_bit.

I will be reusing as much code from the first implementation of the
sieve as possible, and will also try to factor out as much shared
functionality as possible to a shared library (ex. compute happiness).

After each process or thread has computed the primailty of their
section, they will increment a integer. Once that integer reaches the
number of threads or process, the bitmap will be copied, and they will
be allowed to continue onto computing happiness. This synchronization is
needed due to happy numbers using the same bitmap.

\subsection{Signals \& Implicit Sharing}
Because a bitmap backed is a single number, it would be extremely wasteful to
lock the number. Instead, I will setup \emph{MAX\_INT (or MAX\_LONG)/num
threads} , partitions across the number. Two locked partions will be
needed for working on the section of the map they represent. This
includes the extra two partions at 0, and MAX\_INT (or MAX\_LONG). Each
\emph{num threads} section will have a condition variable to signal
sections adjacent.

\subsection{Processes \& Shared Memory}
The same approach will be used in this part, but with the exception of
mutexs and condition variables being replaced by semaphors of size 2.
The implicit memory, or global variable, will be replaced by the bitmap
being stored in the shared memory.

\subsection{Happy Numbers}
Before moving onto \emph{Happy Numbers} I will check to make sure I have
all the primes with multiple numbers of threads and processes. Once I am
confident with my output of primes, I will move on to checking for happy
numbers. Checking for number happiness will be the same as generating
primes, with the exception that if a number is not happy, it is marked
as not prime. Since there is a distinct transition point between prime
generation and happiness, a copy of the bitmap will be made and saved
for later output.


\subsection{Deviation}


\section{Questions}
\begin{description}
  \item  What do you think the main point of this assignment is?

         The main point of this assignment is for us to gain experience
         with race conditions, use pthreads, shared memory, and
         synchronization constructs. Basically have first hand
         experience writing concurrent code.

  \item  How did you ensure your solution was correct? Testing details, for
         instance. This section should be very thorough.

  \item  What did you learn?

\end{description}

\newpage

\section{Work Log}

%\verbatiminput{work-log.txt}

\end{document}
