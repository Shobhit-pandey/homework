\documentclass[letterpaper,10pt]{article}

\usepackage{alltt}                                           
\usepackage{float}
\usepackage{color}

\usepackage{url}

\usepackage{balance}
\usepackage{enumitem}

\usepackage{pstricks, pst-node}

\usepackage{geometry}
\geometry{textheight=10in, textwidth=7.5in}

\usepackage{hyperref}

\usepackage{textcomp}
\usepackage{listings}
\usepackage{verbatim}

\title{CS311 - FA13: Perfect Numbers}
\date{\today}
\author{Trevor Bramwell}

% Setup the PDF Metadata
\hypersetup{
  colorlinks = true,
  urlcolor = black,
  pdfauthor = {Trevor Bramwell},
  pdfkeywords = {},
  pdftitle = {CS311 - FA13: Perfect Numbers},
  pdfsubject = {},
  pdfpagemode = UseNone
}

\parindent = 0.0 in
\parskip = 0.2 in

\begin{document}
\maketitle

\section{Design}

This project is designed around three interdepended applications:
compute, report, and manage. All use internet sockets to communicate,
with the end goal of finding perfect numbers. Distributed.

Manage manages compute processes, and hands out ranges of numbers
relative to the amount of work each compute process is capable of.
Report requests data from manage, and also is used to stop manage,
which in turn stops all compute clients.

For the managing of ranges needing to be computed, and distribution, I
borrow heavily from The Linux Programming Interface, section 59.11
(Stream Sockets). This section shows how to handling passing ranges from
a server to clients, and update a local reference to the next range to
compute.

\subsection{XML Protocol}
Because this assignment requires that we parse XML by hand, I have
created a 'perfect' protocol for communication between the compute
processes and manage.

After each compute process establishes a connection with manage, they
send their performance data. In response, manage sends an initial point
for compute to start at. This request and reponse is handled by the
following XML protocol:

NOTE: Each request and response begins with the string:
\begin{verbatim}
<?xml version="1.0" encoding="UTF-8"?>
\end{verbatim}

Request:
\begin{verbatim}
<perf>235234</perf>
\end{verbatim}

Reponse:
\begin{verbatim}
<range>1</range>
\end{verbatim}

A subsequent request to manage by any client will return 235235.

The other protocol involved is between manage and report.

Request:
\begin{verbatim}
<report>True</report>
\end{verbatim}

Response:
\begin{verbatim}
<perfect>
  <number>1</number>
  <number>6</number>
  ...
  <number>N</number>
</perfect>
<clients>
  <client>
    <ip>352.643.234.8</ip>
    <port>723423</port>
    <perf>235234</perf>
  </client>
</clients>
\end{verbatim}

\section{Deviation}

\section{Questions}
\begin{description}
  \item  What do you think the main point of this assignment is?

  \item  How did you ensure your solution was correct? Testing details, for
         instance. This section should be very thorough.

  \item  What did you learn?

\end{description}

\newpage

%\section{Work Log}

%\verbatiminput{work-log.txt}

\end{document}
