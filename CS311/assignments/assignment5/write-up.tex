\documentclass[letterpaper,10pt]{article}

\usepackage{alltt}                                           
\usepackage{float}
\usepackage{color}

\usepackage{url}

\usepackage{balance}
\usepackage{enumitem}

\usepackage{pstricks, pst-node}

\usepackage{geometry}
\geometry{textheight=10in, textwidth=7.5in}

\usepackage{hyperref}

\usepackage{textcomp}
\usepackage{listings}
\usepackage{verbatim}

\title{CS311 - FA13: Perfect Numbers}
\date{\today}
\author{Trevor Bramwell}

% Setup the PDF Metadata
\hypersetup{
  colorlinks = true,
  urlcolor = black,
  pdfauthor = {Trevor Bramwell},
  pdfkeywords = {},
  pdftitle = {CS311 - FA13: Perfect Numbers},
  pdfsubject = {},
  pdfpagemode = UseNone
}

\parindent = 0.0 in
\parskip = 0.2 in

\begin{document}
\maketitle

\section{Design}

This project is designed around three interdepended applications:
compute, report, and manage. All use internet sockets to communicate,
with the end goal of finding perfect numbers. Distributed.

Manage manages compute processes, and hands out ranges of numbers
relative to the amount of work each compute process is capable of.
Report requests data from manage, and also is used to stop manage,
which in turn stops all compute clients.

For the managing of ranges needing to be computed, and distribution, I
borrow heavily from The Linux Programming Interface, section 59.11
(Stream Sockets). This section shows how to handling passing ranges from
a server to clients, and update a local reference to the next range to
compute.

Since manage is also in python, I will be borrowing from the Python
Module of the Week website's section on select.

Report will be a simple python script that connects via sockets, and
sends XML message for specific pieces of data. The '-k' will send an XML
message that tells manage to kill itself.

\subsection{XML Protocol}
Because this assignment requires that we parse XML by hand, I have
created a 'perfect' protocol for communication between the compute
processes and manage.

\subsubsection*{Compute}
After each compute process establishes a connection with manage, they
send their performance data. In response, manage sends an initial point
for compute to start at. This request and reponse is handled by the
following XML protocol:

NOTE: Each request and response begins with the string:
\begin{verbatim}
<?xml version="1.0" encoding="UTF-8"?>
\end{verbatim}

Request:
\begin{verbatim}
<perf>235234</perf>
\end{verbatim}

Reponse:
\begin{verbatim}
<range>1</range>
\end{verbatim}

A subsequent request to manage by any client would return 235235, as
that will be the next available range.

When a client finds perfect numbers it will send them to the server, and
the server will not send a response.

Request:
\begin{verbatim}
<perfect>
  <number>1</number>
  ...
  <number>8286</number>
</perfect>
\end{verbatim}

\subsubsection*{Report}
The other protocol involved is between manage and report. There are only
two requests report sends. One for the numbers, and one for shutting
down the server.

Request:
\begin{verbatim}
<report />
\end{verbatim}

Response:
\begin{verbatim}
<perfect>
  <number>1</number>
  <number>6</number>
  ...
  <number>N</number>
</perfect>
<clients>
  <client>
    <ip>352.643.234.8</ip>
    <port>723423</port>
    <perf>235234</perf>
  </client>
  ...
</clients>
\end{verbatim}

Request:
\begin{verbatim}
<quit />
\end{verbatim}

\section{Deviation}
Since this protocol isn't actually XML compliant, I forwent the strict
adhearence to the xml header, and just wrapped each request and response
in:
\begin{verbatim}
<xml>
...
</xml>
\end{verbatim}

\section{Questions}
\begin{description}
  \item  What do you think the main point of this assignment is?

         The main point of this assignment was to get us to learn how to
         use AF\_INET sockets, select, and message queues.

  \item  How did you ensure your solution was correct? Testing details, for
         instance. This section should be very thorough.

         My testing was done primarily through trial and error. Almost
         all of it was done on my local machine, using AF\_INET sockets.
         This allowed me to iterrate quickly on issues and resolve bugs
         when they cropped up.

  \item  What did you learn?

         I learned about using sockets, perfect numbers, calculating
         IOPS, and regular expression groups.

         Calls to send and recv, are not guaranteed to send/recv the
         full amount requested. These calls can also exit immediately if
         the socket is set to non-blocking. This caused major issues for
         me when trying to send 'kill' signals to all the compute
         processes.

         Also, it is very challenging to debug asynchronus network code. This
         is evident from using select. Because a client or server can
         die at any time, you have to think about the order of actions
         when reading and writing to sockets.

\end{description}

\newpage

\section{Work Log}

\verbatiminput{work-log.txt}

\end{document}
