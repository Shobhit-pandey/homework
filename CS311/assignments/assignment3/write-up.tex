\documentclass[letterpaper,10pt]{article}

\usepackage{alltt}                                           
\usepackage{float}
\usepackage{color}

\usepackage{url}

\usepackage{balance}
\usepackage{enumitem}

\usepackage{pstricks, pst-node}

\usepackage{geometry}
\geometry{textheight=10in, textwidth=7.5in}

\usepackage{hyperref}

\usepackage{textcomp}
\usepackage{listings}
\usepackage{verbatim}

\title{CS311 - FA13: Uniqify}
\date{\today}
\author{Trevor Bramwell}

% Setup the PDF Metadata
\hypersetup{
  colorlinks = true,
  urlcolor = black,
  pdfauthor = {Trevor Bramwell},
  pdfkeywords = {},
  pdftitle = {CS311 - FA13: Uniqify},
  pdfsubject = {},
  pdfpagemode = UseNone
}

\parindent = 0.0 in
\parskip = 0.2 in

\begin{document}
\maketitle

\section{Design}

To implement the given system, I will first setup a 3D array to hold
pipes, and use this in each of the processes. After that I will setup
the parser to get data, round robin style, into each pipe. I will then
fork off N processes, and close all unneeded file descriptors. Each
process will then exec sort and return it's input to the pipe. The rest
of the program will be merging and suppressing the output from the
pipes, and waiting for the sort processes to finish.

\subsection{Deviation}

The first thing I found out as I got going was that a 3D array was too
confusing for me to handle. I split it up into two separate arrays for
holding the pipes. The second issue I had was setting up the parser
before forking the sort processes. I solved this by forking at three
points in my program so that the execution would be easy to follow. I
did not plan on it, but later decided to refactor the code for
readibility. This also helped for understanding the program logic, as
there was less text to read for each fork.

\section{Questions}
\begin{description}
  \item  What do you think the main point of this assignment is?

         The main point of this assignment was to teach us how to
         register signals, setup pipes, and handle multiple processes.

  \item  How did you ensure your solution was correct? Testing details, for
         instance. This section should be very thorough.

         I tested my solution by first creating a file that held 6
         copies of war and peace. I then used this for the basis of my
         tests, using different numbers of pipes and almost always
         specifying an output file.

         To ensure the output and input streams were correct, I tried
         all permutations of each. I was satisfied when they all output
         to the correct places.

  \item  What did you learn?

         Sometimes the problems you think will the be simplest to solve,
         are the hardest. When going about the assignment I put off
         supressing to the last minute because I thought it would be
         pretty straight forward. Little did I know it would be what I
         spent the majority of my time on, and I would still not get it
         right.

         Having a working solution is more important than having a
         correct solution. Had I ignored the fact that supression wasn't
         working correctly, and would take more time than I had, I could
         have turned this assignment in on time, and not have lost as
         many points as I probably will.
\end{description}

\newpage

\section{Work Log}

\verbatiminput{work-log.txt}

\end{document}
