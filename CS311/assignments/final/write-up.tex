\documentclass[letterpaper,10pt]{article}

\usepackage{alltt}                                           
\usepackage{float}
\usepackage{color}

\usepackage{url}

\usepackage{balance}
\usepackage{enumitem}

\usepackage{pstricks, pst-node}

\usepackage{geometry}
\geometry{textheight=10in, textwidth=7.5in}

\usepackage{hyperref}

\usepackage{textcomp}
\usepackage{listings}
\usepackage{verbatim}

\title{CS311 - FA13: Final}
\date{\today}
\author{Trevor Bramwell}

% Setup the PDF Metadata
\hypersetup{
  colorlinks = true,
  urlcolor = black,
  pdfauthor = {Trevor Bramwell},
  pdfkeywords = {},
  pdftitle = {CS311 - FA13: Final},
  pdfsubject = {},
  pdfpagemode = UseNone
}

\parindent = 0.0 in
\parskip = 0.2 in

\begin{document}
\maketitle

\section{Overview}

This paper compares and contrasts Stream Sockets, Anonymous Pipes, and
Multiprocessing, between the Windows and POSIX APIs. For each section I
will provide a sample piece of code for each interface, using as many
API functions as possible. I will first give an overview of what the
example does, provide the example, then explain how the APIs differ
within each example. When these interfaces are placed side by side, this
should allow the reader to easily see the similarites and differences
between them.

\section{Sockets}

The first API I will be comparing is Sockets. In Windows these are
referred to as \emph{WinSock}. WinSock has the same commands for
creating and accepting connections as POSIX sockets, with the addition
of 'closesocket'. The difference is in Window's use of macros over
\emph{file descriptors} (fds).

Client Socket Example: \url{http://msdn.microsoft.com/en-us/library/windows/desktop/ms737591\%28v=vs.85%29.aspx}

%%%%%%
\begin{comment}
%%%%%%

\subsection{socket}
Windows:
SOCKET WSAAPI socket(
  \_In\_  int af,
  \_In\_  int type,
  \_In\_  int protocol
);

POSIX:
int socket(int domain, int type, int protocol);

\subsection{bind}
Windows:
int bind(
  \_In\_  SOCKET s,
  \_In\_  const struct sockaddr *name,
  \_In\_  int namelen
);

POSIX:
int bind(int socket, const struct sockaddr *address,
       socklen\_t address\_len);

\subsection{listen}
Windows:
int listen(
  \_In\_  SOCKET s,
  \_In\_  int backlog
);

POSIX:
int listen(int socket, int backlog);

\subsection{connect}
Windows:
int connect(
  \_In\_  SOCKET s,
  \_In\_  const struct sockaddr *name,
  \_In\_  int namelen
);

POSIX:
int connect(int socket, const struct sockaddr *address,
       socklen\_t address\_len);

\subsection{accept}
Windows:
SOCKET accept(
  \_In\_     SOCKET s,
  \_Out\_    struct sockaddr *addr,
  \_Inout\_  int *addrlen
);

POSIX:
int accept(int socket, struct sockaddr *restrict address,
       socklen\_t *restrict address\_len);

\subsection{shutdown}
Windows:
int shutdown(
  \_In\_  SOCKET s,
  \_In\_  int how
);

POSIX:
int shutdown(int socket, int how);

\subsection{close/closesocket}
Widows:
int closesocket(
  \_In\_  SOCKET s
);

POSIX:
int close(int fildes);

\subsection{send}
Windows:
int send(
  \_In\_  SOCKET s,
  \_In\_  const char *buf,
  \_In\_  int len,
  \_In\_  int flags
);

POSIX:
ssize\_t send(int socket, const void *buffer, size\_t length, int flags);

\subsection{recv}
Windows:
int recv(
  \_In\_   SOCKET s,
  \_Out\_  char *buf,
  \_In\_   int len,
  \_In\_   int flags
);

POSIX:
ssize\_t recv(int socket, void *buffer, size\_t length, int flags);

%%%%%%
\end{comment}
%%%%%%

\section{Anonymous Pipes}

Pipes Example: \url{http://msdn.microsoft.com/en-us/library/windows/desktop/ms682499(v=vs.85).aspx}

\section{Multiprocessing}

Multiprocessing Example: \url{https://msdn.microsoft.com/en-us/library/windows/desktop/ms682512(v=vs.85).aspx}

\newpage

\section{References}

\subsection{Sockets}
\url{http://msdn.microsoft.com/en-us/library/windows/desktop/bb530741(v=vs.85).aspx}

Windows API Reference: \url{msdn.microsoft.com/en-us/library/windows/desktop/ms741394(v=vs.85).aspx}
POSIX References: \url{http://pubs.opengroup.org/onlinepubs/9699919799/toc.htm}

\subsection{Pipes}
\url{http://msdn.microsoft.com/en-us/library/windows/desktop/aa365780(v=vs.85).aspx}

\url{http://msdn.microsoft.com/en-us/library/windows/desktop/aa365590(v=vs.85).aspx}

\subsection{Multiprocessing}
\url{http://msdn.microsoft.com/en-us/library/windows/desktop/ms684841(v=vs.85).aspx}

\end{document}
