\documentclass[letterpaper,10pt]{article}

\usepackage{alltt}                                           
\usepackage{float}
\usepackage{color}

\usepackage{url}

\usepackage{balance}
\usepackage{enumitem}

\usepackage{pstricks, pst-node}

\usepackage{geometry}
\geometry{textheight=10in, textwidth=7.5in}

\usepackage{hyperref}

\usepackage{textcomp}
\usepackage{listings}
\usepackage{verbatim}

\title{CS311 - FA13: Final}
\date{\today}
\author{Trevor Bramwell}

% Setup the PDF Metadata
\hypersetup{
  colorlinks = true,
  urlcolor = black,
  pdfauthor = {Trevor Bramwell},
  pdfkeywords = {},
  pdftitle = {CS311 - FA13: Final},
  pdfsubject = {},
  pdfpagemode = UseNone
}

% Setup listings
\lstset{
    language=C,
    numbers=left,
    basicstyle=\small,
    stepnumber=1,
    frame=single,
    showstringspaces=false,
    showtabs=false,
    stringstyle=\ttfamily,
    captionpos=b,
    backgroundcolor=\color{white},
    title=\lstname
}

\parindent = 0.0 in
\parskip = 0.2 in

\begin{document}
\maketitle

\section{Overview}

This paper compares and contrasts Stream Sockets, Anonymous Pipes, and
Multiprocessing, between the Windows and POSIX APIs. For each section I
will provide a sample piece of code for each interface, using as many
API functions as possible. I will first give an overview of what the
example does, provide the example, then explain how the APIs differ
within each example. When these interfaces are placed side by side, this
should allow the reader to easily see the similarites and differences
between them.

All POSIX examples come from the Linux Man Pages v3.54, all Windows
examples come from the Microsoft Developer Network (MSDN) website.
References to example are provided at the end of this document.

\section{Stream Sockets}

The first API I will be comparing is Stream Sockets. In Windows these are
referred to as \emph{WinSock}. WinSock has the same commands for
creating and accepting connections as POSIX sockets, with the addition
of 'closesocket'. The difference is in Window's use of macros over
\emph{file descriptors} (fds).

\lstinputlisting{posix_sockets_server.c}
\lstinputlisting{win32_sockets_server.c}

From the examples you can see that both use a \emph{struct addrinfo},
make the same calls to \emph{getaddrinfo}, and free the struct with
\emph{freeaddrinfo}.

Both calls to sockets take the same arguments, with the difference that
Windows returns a SOCKET type instead of an file descriptor. This is due
to Windows using file handles, instead of descriptors.

Windows defines some extra symbolic constants for sockets, like
SOCKET_ERROR, which is similar to the return value of `-1' in POSIX.

Instead of using \emph{recv} and \emph{send} the POSIX example uses
\emph{read} and \emph{write} because they are the same call when no
flags are passed. In Windows, because it uses file handles, the
\emph{send} and \emph{recv} commands take SOCKET as their first argument
instead of an integer.

\section{Anonymous Pipes}

The second API I will be comparing is Pipes. Pipes work similarly in
POSIX in Windows. They have a designated read and write end, and errors
occur when reading or writing to the wrong end. The ends are treated as
files, so the file read and write commands work on them.The major
difference is with their initialization.

Here is a program using POSIX pipes that creates a pipe and a child
process, and passes command line arguments through the pipe, which the
child in turn write to standard out.

The Windows program is different in which it have the child open a
file, and write the file pipe, which the parent reads and then writes to
standard out.

\lstinputlisting{posix_pipes.c}
\lstinputlisting{win32_pipes.c}

On Windows pipes are created using two designated file HANDLEs and a
SECURITY_ATTRIBUTES, which determines whether or not children inherit
file HANDLES. Wereas in POSIX a pipe is created using an integer array
of size 2.

\section{Multiprocessing}

\lstinputlisting{posix_procs.c}
\lstinputlisting{win32_procs.c}


\newpage

\section{References}

\subsection{Sockets}
\url{http://msdn.microsoft.com/en-us/library/windows/desktop/bb530741(v=vs.85).aspx}

Windows API Reference: \url{msdn.microsoft.com/en-us/library/windows/desktop/ms741394(v=vs.85).aspx}
POSIX References: \url{http://pubs.opengroup.org/onlinepubs/9699919799/toc.htm}

Client Socket Example: \url{http://msdn.microsoft.com/en-us/library/windows/desktop/ms737591(v=vs.85).aspx}
Server Socket Example: \url{http://msdn.microsoft.com/en-us/library/windows/desktop/ms737593(v=vs.85).aspx}

\subsection{Pipes}
\url{http://msdn.microsoft.com/en-us/library/windows/desktop/aa365780(v=vs.85).aspx}

\url{http://msdn.microsoft.com/en-us/library/windows/desktop/aa365590(v=vs.85).aspx}

Pipes Example: \url{http://msdn.microsoft.com/en-us/library/windows/desktop/ms682499(v=vs.85).aspx}

\subsection{Multiprocessing}
\url{http://msdn.microsoft.com/en-us/library/windows/desktop/ms684841(v=vs.85).aspx}

Multiprocessing Example: \url{https://msdn.microsoft.com/en-us/library/windows/desktop/ms682512(v=vs.85).aspx}

\end{document}
