\documentclass[letterpaper,10pt]{article}

\usepackage{alltt}                                           
\usepackage{float}
\usepackage{color}

\usepackage{url}

\usepackage{balance}
\usepackage{enumitem}

\usepackage{pstricks, pst-node}

\usepackage{geometry}
\geometry{textheight=10in, textwidth=7.5in}

\usepackage{hyperref}

\usepackage{textcomp}
\usepackage{listings}
\usepackage{verbatim}

\title{CS311 - FA13: Final}
\date{\today}
\author{Trevor Bramwell}

% Setup the PDF Metadata
\hypersetup{
  colorlinks = true,
  urlcolor = black,
  pdfauthor = {Trevor Bramwell},
  pdfkeywords = {},
  pdftitle = {CS311 - FA13: Final},
  pdfsubject = {},
  pdfpagemode = UseNone
}

% Setup listings
\lstset{
    language=C,
    numbers=left,
    basicstyle=\small,
    stepnumber=1,
    frame=single,
    showstringspaces=false,
    showtabs=false,
    stringstyle=\ttfamily,
    captionpos=b,
    backgroundcolor=\color{white},
    title=\lstname
}

\parindent = 0.0 in
\parskip = 0.2 in

\begin{document}
\maketitle

\section{Overview}

This paper compares and contrasts Stream Sockets, Anonymous Pipes, and
Multiprocessing, between the Windows and POSIX APIs. For each section I
will provide a sample piece of code for each interface, using as many
API functions as possible. I will first give an overview of what the
example does, provide the example, then explain how the APIs differ
within each example. When these interfaces are placed side by side, this
should allow the reader to easily see the similarites and differences
between them.

All POSIX examples come from the Linux Man Pages v3.54, all Windows
examples come from the Microsoft Developer Network (MSDN) website.
References to example are provided at the end of this document.

\section{Sockets}

The first API I will be comparing is Sockets. In Windows these are
referred to as \emph{WinSock}. WinSock has the same commands for
creating and accepting connections as POSIX sockets, with the addition
of 'closesocket'. The difference is in Window's use of macros over
\emph{file descriptors} (fds).

\lstinputlisting{posix_sockets_server.c}
\lstinputlisting{win32_sockets_server.c}

\section{Anonymous Pipes}

\lstinputlisting{posix_pipes.c}
\lstinputlisting{win32_pipes.c}

\section{Multiprocessing}

\lstinputlisting{posix_procs.c}
\lstinputlisting{win32_procs.c}


\newpage

\section{References}

\subsection{Sockets}
\url{http://msdn.microsoft.com/en-us/library/windows/desktop/bb530741(v=vs.85).aspx}

Windows API Reference: \url{msdn.microsoft.com/en-us/library/windows/desktop/ms741394(v=vs.85).aspx}
POSIX References: \url{http://pubs.opengroup.org/onlinepubs/9699919799/toc.htm}

Client Socket Example: \url{http://msdn.microsoft.com/en-us/library/windows/desktop/ms737591(v=vs.85).aspx}
Server Socket Example: \url{http://msdn.microsoft.com/en-us/library/windows/desktop/ms737593(v=vs.85).aspx}

\subsection{Pipes}
\url{http://msdn.microsoft.com/en-us/library/windows/desktop/aa365780(v=vs.85).aspx}

\url{http://msdn.microsoft.com/en-us/library/windows/desktop/aa365590(v=vs.85).aspx}

Pipes Example: \url{http://msdn.microsoft.com/en-us/library/windows/desktop/ms682499(v=vs.85).aspx}

\subsection{Multiprocessing}
\url{http://msdn.microsoft.com/en-us/library/windows/desktop/ms684841(v=vs.85).aspx}

Multiprocessing Example: \url{https://msdn.microsoft.com/en-us/library/windows/desktop/ms682512(v=vs.85).aspx}

\end{document}
