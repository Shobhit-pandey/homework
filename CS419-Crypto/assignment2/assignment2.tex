\documentclass[12pt]{article}

%% minimal example of some latex stuff

\usepackage{xspace,graphicx,amsmath,amssymb,xcolor}

\newcommand{\pct}{\mathbin{\%}}

\newcommand{\K}{\mathcal{K}}
\newcommand{\M}{\mathcal{M}}
\newcommand{\C}{\mathcal{C}}
\newcommand{\Z}{\mathbb{Z}}

\newcommand{\Enc}{\text{\sf Enc}}
\newcommand{\Dec}{\text{\sf Dec}}
\newcommand{\KeyGen}{\text{\sf KeyGen}}

% fancy script L
\usepackage[mathscr]{euscript}
\renewcommand{\L}{\ensuremath{\mathscr{L}}\xspace}
\newcommand{\lib}[1]{\ensuremath{\L_{\textsf{#1}}}\xspace}

\newcommand{\myterm}[1]{\ensuremath{\text{#1}}\xspace}
\newcommand{\bias}{\myterm{bias}}
\newcommand{\link}{\diamond}
\newcommand{\subname}[1]{\ensuremath{\textsc{#1}}\xspace}


%%% code boxes

\newcommand{\fcodebox}[1]{%
    \framebox{\codebox{#1}}%
}
\newcommand{\hlcodebox}[1]{%
    \fcolorbox{black}{myyellow}{\codebox{#1}}%
}


\usepackage{varwidth}

\newcommand{\codebox}[1]{%
        \begin{varwidth}{\linewidth}%
        \begin{tabbing}%
            ~~~\=\quad\=\quad\=\quad\=\kill
            #1
        \end{tabbing}%
        \end{varwidth}%
}


\definecolor{myyellow}{HTML}{F5F589}

\newcommand{\mathhighlight}[1]{\basehighlight{$#1$}}
\newcommand{\highlightline}[1]{%\raisebox{0pt}[-\fboxsep][-\fboxsep]{
    \hspace*{-\fboxsep}\basehighlight{#1}%
%}
}
\newcommand{\basehighlight}[1]{\colorbox{myyellow}{#1}}


\title{Cryptography: HW2}
\author{Trevor Bramwell}
\date{
    Professor Mike Rosulek\\
    January 23, 2015
}
%%%%

\begin{document}
\maketitle


\paragraph{1} Negligible Functions
\\\\
Unless otherwise stated, all the following functions are negligible.

    \[
        \frac{1}{2^{n/2}} =  \frac{1}{2^{1/2}} * \frac{1}{2^{n}}
    \]
    We know $\frac{1}{2^{n}}$ is negligible. Since this takes the form
    of $p(x)f(x)$ where $p(x)$ is a polynomial and $f(x)$ is negligible,
    $p(x)f(x)$ is negligible as well.

    \[
        \frac{1}{2^{\log{(n^2)}}} < \frac{1}{2^{log(n)}} < \frac{1}{n^c}
    \]
    Becuase ($\log{n}$) grows at a quicker rate than the constant $c$,
    $\frac{1}{2^{\log{(n)}}}$ will asymptotically approach zero faster
    than $\frac{1}{n^c}$.

    \[
        \frac{1}{n^{\log{(n)}}} < \frac{1}{n^c}
    \]

    \[
        \frac{1}{2^{(\log{n})^2}} < \frac{1}{2^{(\log{n})}} < \frac{1}{n^c}
    \]


    \[
        \frac{1}{(\log{n})^2} > \frac{1}{\log{n}} > \frac{1}{n^c}
        \\
        \text{ - Not Negligible.}
    \]

    \[
        \left(\frac{1}{n^{1/n}} = \frac{1}{n^{-n}} = n^{n}\right) >
        \frac{1}{n^c}
        \\
        \text{ - Not Negligible.}
    \]

    \[
        \left(\frac{1}{\sqrt{n}} = \frac{1}{n^{1/2}}\right) >= \frac{1}{n^c}
        \\
        \text{ - Not Negligible.}
    \]

    \[
        \frac{1}{2^{\sqrt{n}}} < \frac{1}{2^n} < \frac{1}{n^c}
    \]

    \[
        \frac{1}{n^{\sqrt{n}}} < \frac{1}{n^{\log{n}}} < \frac{1}{n^c}
    \]

\paragraph{2} Distinguisher for the injective PRG: $G$.

\begin{itemize}
    \item[\textbf{a}] $\bias(A, \lib{prg-real}^G, \lib{prg-rand}^G) =$
        $\frac{1}{2^{n+\ell}}$\\\\
        This distinguisher is no better than a random guess, except that
        it randomly guesses all possible outputs. 
    \item[\textbf{b}] This does not contradict $G$ being a PRG. The
        security property of $\lib{prg-real}^G$ and $\lib{prg-rand}^G$
        still holds.
    \item[\textbf{c}] When the bias of $G$ is not injective, the bias
        has the possibility of increasing a non-negligible amount, thus
        rendering $\lib{prg-real}$ insecure.
\end{itemize}

\paragraph{3} Secure Length-Tripling PRG\\

(a)
\[
    A \link 
    \fcodebox{
        \underline{$\subname{query}()$:} \\
        \> $s \gets \{0,1\}^{2n}$ \\
        \> \highlightline{$x := \subname{G}(s_{left})$} \\
        \> \highlightline{$y := \subname{G}(s_{right})$} \\
        \> \highlightline{return $x \oplus y$}
    }
\]

(b)
\[
    A \link 
    \fcodebox{
        \underline{$\subname{query}()$:} \\
        \> \highlightline{$s_{left} \gets \{0,1\}^{n}$} \\
        \> \highlightline{$s_{right} \gets \{0,1\}^{n}$} \\
        \> $x := \subname{G}(s_{left})$ \\
        \> $y := \subname{G}(s_{right})$ \\
        \> return $x \oplus y$
    }
\]

(c)
\[
    A \link 
    \fcodebox{
        \underline{$\subname{query}()$:} \\
        \> $s_{left} \gets \{0,1\}^{n}$ \\
        \> $x := \subname{G}(s_{left})$ \\
        \> \highlightline{$s_{right} \gets \{0,1\}^{n}$} \\
        \> $y := \subname{G}(s_{right})$ \\
        \> return $x \oplus y$
    }
\]

(d)
\[
    A \link 
    \fcodebox{
        \underline{$\subname{query}()$:} \\
        \> \highlightline{$x := \subname{query'}()$} \\
        \> $s_{right} \gets \{0,1\}^{n}$ \\
        \> $y := \subname{G}(s_{right})$ \\
        \> return $x \oplus y$
    }
    \link 
    \hlcodebox{
        \underline{$\subname{query'}()$:} \\
        \> $k \gets \{0,1\}^n$ \\
        \> return $G(k)$
    }
\]

(e)
\[
    A \link 
    \fcodebox{
        \underline{$\subname{query}()$:} \\
        \> $x := \subname{query'}()$ \\
        \> $s_{right} \gets \{0,1\}^{n}$ \\
        \> $y := \subname{G}(s_{right})$ \\
        \> return $x \oplus y$
    }
    \link 
    \hlcodebox{
        \underline{$\subname{query'}()$:} \\
        \> $k \gets \{0,1\}^{3n}$ \\
        \> return $k$
    }
\]

(f)
\[
    A \link 
    \fcodebox{
        \underline{$\subname{query}()$:} \\
        \> \highlightline{$x \gets \{0,1\}^{3n}$} \\
        \> $s_{right} \gets \{0,1\}^{n}$ \\
        \> $y := \subname{G}(s_{right})$ \\
        \> return $x \oplus y$
    }
\]

(g)
\[
    A \link 
    \fcodebox{
        \underline{$\subname{query}()$:} \\
        \> $x \gets \{0,1\}^{3n}$ \\
        \> \highlightline{$ y := \subname{query'}()$} \\
        \> return $x \oplus y$
    }
    \link 
    \hlcodebox{
        \underline{$\subname{query'}()$:} \\
        \> $k \gets \{0,1\}^n$ \\
        \> return $G(k)$
    }
\]

(h)
\[
    A \link 
    \fcodebox{
        \underline{$\subname{query}()$:} \\
        \> $x \gets \{0,1\}^{3n}$ \\
        \> $ y := \subname{query'}()$ \\
        \> return $x \oplus y$
    }
    \link 
    \hlcodebox{
        \underline{$\subname{query'}()$:} \\
        \> $k \gets \{0,1\}^{3n}$ \\
        \> return $k$
    }
\]

(i)
\[
    A \link 
    \fcodebox{
        \underline{$\subname{query}()$:} \\
        \> $x \gets \{0,1\}^{3n}$ \\
        \> \highlightline{$y \gets \{0,1\}^{3n}$} \\
        \> return $x \oplus y$
    }
\]

(j)
\[
    A \link 
    \fcodebox{
        \underline{$\subname{query}()$:} \\
        \> \highlightline{$z \gets \{0,1\}^{3n}$} \\
        \> \highlightline{return $z$}
    }
\]

\noindent
Justification of steps:
\begin{itemize}
    \addtolength{\leftskip}{24pt}
    \item[(a)] This is just $A \link \lib{prg-real}^G$ with the deatils of
        $H$ filled in.
    \item[(a)$\Rightarrow$(b)] We have split $s$ into it's $left$ and
        $right$ parts. No effect on the program.
    \item[(c)$\Rightarrow$(d)] Factored out $s_{left}$ and first call to
        $G$ in it's own subroutine $QUERY'()$. No effect on the program.
    \item[(e)] We swap the call to the length-tripling PRG $G$. Assuming
        $G$ is a secure PRG, we can replace it's call with the
        generation and return of $3n$ random-bits.
    \item[(e)$\Rightarrow$(f)] Inlining of $QUERY'()$. No effect on the
        program.
    \item[(g)$\Rightarrow$(i)] The process of (d) through (f) is
        repeated for $y$.
    \item[(j)] Because we know that XOR produces a random distribution
        at uniform, it is simple to state that this process is the same
        as generating $3n$ random-bits $\square$
\end{itemize}

\end{document}
