\documentclass[12pt]{article}

%% minimal example of some latex stuff

\usepackage{xspace,graphicx,amsmath,amssymb,xcolor}

\newcommand{\pct}{\mathbin{\%}}

\newcommand{\K}{\mathcal{K}}
\newcommand{\M}{\mathcal{M}}
\newcommand{\C}{\mathcal{C}}
\newcommand{\Z}{\mathbb{Z}}

\newcommand{\Enc}{\text{\sf Enc}}
\newcommand{\Dec}{\text{\sf Dec}}
\newcommand{\KeyGen}{\text{\sf KeyGen}}

% fancy script L
\usepackage[mathscr]{euscript}
\renewcommand{\L}{\ensuremath{\mathscr{L}}\xspace}
\newcommand{\lib}[1]{\ensuremath{\L_{\textsf{#1}}}\xspace}

\newcommand{\myterm}[1]{\ensuremath{\text{#1}}\xspace}
\newcommand{\bias}{\myterm{bias}}
\newcommand{\link}{\diamond}
\newcommand{\subname}[1]{\ensuremath{\textsc{#1}}\xspace}


%%% code boxes

\newcommand{\fcodebox}[1]{%
    \framebox{\codebox{#1}}%
}
\newcommand{\hlcodebox}[1]{%
    \fcolorbox{black}{myyellow}{\codebox{#1}}%
}


\usepackage{varwidth}

\newcommand{\codebox}[1]{%
        \begin{varwidth}{\linewidth}%
        \begin{tabbing}%
            ~~~\=\quad\=\quad\=\quad\=\kill
            #1
        \end{tabbing}%
        \end{varwidth}%
}


\definecolor{myyellow}{HTML}{F5F589}

\newcommand{\mathhighlight}[1]{\basehighlight{$#1$}}
\newcommand{\highlightline}[1]{%\raisebox{0pt}[-\fboxsep][-\fboxsep]{
    \hspace*{-\fboxsep}\basehighlight{#1}%
%}
}
\newcommand{\basehighlight}[1]{\colorbox{myyellow}{#1}}


\title{Cryptography: HW4}
\author{Trevor Bramwell}
\date{
    Professor Mike Rosulek\\
    February 5, 2015
}
%%%%

\begin{document}
\maketitle


\paragraph{1a} Consider a variant of the CPA definition for CBC
encryption in which an adversary is able to see the next IV before
choosing a plaintext.

\begin{center}
\[
    \framebox{
        \codebox{
            $k \gets \Sigma.$KeyGen \\
            \\
            \underline{$\subname{CHALLENGE}(m_L, m_R)$:} \\
            \> return null if $|m_L| \neq |m_R|$ \\
            \> $c_{0}c_{1}{\cdots}c_{\ell} := \Sigma.\subname{Enc}(k, m_L)$ \\
            \> return $c_{0}c_{1}{\cdots}c_{\ell}$\\
            \\
            \underline{$\subname{VEC}$:} \\
            \> return $iv_{next}$
        }
    }
    \qquad
    \framebox{
        \codebox{
            $k \gets \Sigma.$KeyGen \\
            \\
            \underline{$\subname{CHALLENGE}(m_L, m_R)$:} \\
            \> return null if $|m_L| \neq |m_R|$ \\
            \> $c_{0}c_{1}{\cdots}c_{\ell} := \Sigma.\subname{Enc}(k, m_R)$ \\
            \> return $c_{0}c_{1}{\cdots}c_{\ell}$\\
            \\
            \underline{$\subname{VEC}$:} \\
            \> return $iv_{next}$
        }
    }
\]
\end{center}

\paragraph{1b} This modification breaks CPA security for CBC encryption.
This can be seen by the following distinguisher $A$.

\[
    \fcodebox{
        \underline{$\subname{A}$:} \\
        \> $iv_1 \gets \subname{VEC}$ \\
        \> $iv_{0}ca_{1}{\cdots}ca_{\ell} := \subname{CHALLENGE}(m_{1}m_{2}{\cdots}m_{\ell})$ \\
        \> $iv_{1}cb_{1}{\cdots}cb_{\ell} :=
        \subname{CHALLENGE}((iv_{0}{\oplus}iv_{1}{\oplus}m_{1})m_{2}{\cdots}m_{\ell})$ \\
        \> return $ca_{1-\ell} \stackrel{?}{=} cb_{1-\ell}$
    }
\]

\noindent
Assume $k$ does not change for each call to $\subname{CHALLENGE}$. In
the first call to $\subname{CHALLENGE}$, $F_k$ is called with
$(iv_{0}{\oplus}m_1)$. In
the second call to $\subname{CHALLENGE}$, we know the $iv$ ahead of
time. This allows us to replace it with the previously used $iv_{0}$.
\[
    iv_{0}{\oplus}m_{1} = iv_{1} \oplus (iv_{0}{\oplus}iv_{1}{\oplus}m_{1})
\]
This results in the same ciphertext being output twice. The bias for A
is then:

\begin{align*}
    \bias(A,\lib{cpa-real}^{CBC},\lib{cpa-rand}^{CBC}) &=
    |\Pr[A \link \lib{cpa-real}^{CBC} \mbox{ outputs 1}] - \Pr[A \link \lib{cpa-rand}^{CBC} \mbox{ outputs 1}]| \\
    &= | 1 - \frac{1}{2^{n}} | \\
    &= \text{Non-negligible}
\end{align*}

\paragraph{2} Consider a variant of CPA\$ security in which the
adversary is allowed to {\it choose} but not {\it re-use} IVs. We know
"standard" CBC encryption (with a random IV) is CPA\$-secure.

(a)
\[
    \fcodebox{
        $k \gets \{0, 1\}^{2n}$ \\
        $iv \gets $Adversary Choice \\ \\
        \underline{$\subname{CHALLENGE}(m_{1}{\cdots}m_{\ell})$:} \\
        \> $iv' \gets F(k_{left}, iv)$ \\
        \> $c := $ CBC encryption of $m_{1}{\cdots}m_{\ell}$ \\
        \>\> under key $k_{right}$ \\
        \>\> using initialization vector $iv'$ \\
        \> return $c_{1}{\cdots}c_{\ell}$ \\
    }
\]

(b)
\[
    \fcodebox{
        \highlightline{$k_{left} \gets \{0, 1\}^{n}$} \\
        \highlightline{$k_{right} \gets \{0, 1\}^{n}$} \\
        $iv \gets $Adversary Choice \\ \\
        \underline{$\subname{CHALLENGE}(m_{1}{\cdots}m_{\ell})$:} \\
        \> $iv' \gets F(k_{left}, iv)$ \\
        \> $c := $ CBC encryption of $m_{1}{\cdots}m_{\ell}$ \\
        \>\> under key $k_{right}$ \\
        \>\> using initialization vector $iv'$ \\
        \> return $c_{1}{\cdots}c_{\ell}$ \\
    }
\]

(c)
\[
    \fcodebox{
        $k_{right} \gets \{0, 1\}^{n}$ \\
        $iv \gets $Adversary Choice \\ \\
        \underline{$\subname{CHALLENGE}(m_{1}{\cdots}m_{\ell})$:} \\
        \> $iv' \gets $\highlightline{$\subname{QUERY}(iv)$} \\
        \> $c := $ CBC encryption of $m_{1}{\cdots}m_{\ell}$ \\
        \>\> under key $k_{right}$ \\
        \>\> using initialization vector $iv'$ \\
        \> return $c_{1}{\cdots}c_{\ell}$ \\
    }
    \link
    \hlcodebox{
        $k_{left} \gets \{0, 1\}^{n}$ \\ \\
        \underline{$\subname{QUERY}(iv)$:} \\
        \> return $F(k_{left}, iv)$
    }
\]

(d)
\[
    \fcodebox{
        $k_{right} \gets \{0, 1\}^{n}$ \\
        $iv \gets $Adversary Choice \\ \\
        \underline{$\subname{CHALLENGE}(m_{1}{\cdots}m_{\ell})$:} \\
        \> $iv' \gets $\highlightline{$\subname{QUERY}(iv)$} \\
        \> $c := $ CBC encryption of $m_{1}{\cdots}m_{\ell}$ \\
        \>\> under key $k_{right}$ \\
        \>\> using initialization vector $iv'$ \\
        \> return $c_{1}{\cdots}c_{\ell}$ \\
    }
    \link
    \hlcodebox{
        $T := empty$ \\ \\
        \underline{$\subname{QUERY}(iv)$:} \\
        \> if $T[iv]$ undefined: \\
        \>\> $T[iv] \gets \{0, 1\}^n$ \\
        \> return $T[iv]$
    }
\]

(e)
\[
    \fcodebox{
        $k_{right} \gets \{0, 1\}^{n}$ \\
        $iv \gets $Adversary Choice \\ \\
        \underline{$\subname{CHALLENGE}(m_{1}{\cdots}m_{\ell})$:} \\
        \> $iv' \gets \subname{QUERY}(iv)$ \\
        \> $c := $ CBC encryption of $m_{1}{\cdots}m_{\ell}$ \\
        \>\> under key $k_{right}$ \\
        \>\> using initialization vector $iv'$ \\
        \> return $c_{1}{\cdots}c_{\ell}$ \\
    }
    \link
    \fcodebox{
        $T := empty$ \\ \\
        \underline{$\subname{QUERY}(iv)$:} \\
        \> if $T[iv]$ undefined: \\
        \>\> \highlightline{$T[iv] := \subname{SAMP}()$} \\
        \> return $T[iv]$
    }
    \link
    \hlcodebox{
        \underline{$\subname{SAMP}()$:} \\
        \> $s \gets \{0, 1\}^n$ \\
        \> return $s$
    }
\]

(f)
\[
    \fcodebox{
        $k_{right} \gets \{0, 1\}^{n}$ \\
        $iv \gets $Adversary Choice \\ \\
        \underline{$\subname{CHALLENGE}(m_{1}{\cdots}m_{\ell})$:} \\
        \> $iv' \gets \subname{QUERY}(iv)$ \\
        \> $c := $ CBC encryption of $m_{1}{\cdots}m_{\ell}$ \\
        \>\> under key $k_{right}$ \\
        \>\> using initialization vector $iv'$ \\
        \> return $c_{1}{\cdots}c_{\ell}$ \\
    }
    \link
    \fcodebox{
        $T := empty$ \\ \\
        \underline{$\subname{QUERY}(iv)$:} \\
        \> if $T[iv]$ undefined: \\
        \>\> $T[iv] := \subname{SAMP}()$ \\
        \> return $T[iv]$
    }
    \link
    \fcodebox{
        \highlightline{$S := \emptyset$} \\ \\
        \underline{$\subname{SAMP}()$:} \\
        \> $s \gets \{0, 1\}^n $\highlightline{$\; \backslash \; S$} \\
        \> \highlightline{$S := S \cup \{s\}$} \\
        \> return $s$
    }
\]

(g)
\[
    \fcodebox{
        $k_{right} \gets \{0, 1\}^{n}$ \\
        $iv \gets $Adversary Choice \\ \\
        \underline{$\subname{CHALLENGE}(m_{1}{\cdots}m_{\ell})$:} \\
        \> $iv' \gets \subname{QUERY}(iv)$ \\
        \> $c := $ CBC encryption of $m_{1}{\cdots}m_{\ell}$ \\
        \>\> under key $k_{right}$ \\
        \>\> using initialization vector $iv'$ \\
        \> return $c_{1}{\cdots}c_{\ell}$ \\
    }
    \link
    \hlcodebox{
        \underline{$\subname{QUERY}(iv)$:} \\
        \> $t := \subname{SAMP}()$ \\
        \> return $t$
    }
    \link
    \fcodebox{
        $S := \emptyset$ \\ \\
        \underline{$\subname{SAMP}()$:} \\
        \> $s \gets \{0, 1\}^n \; \backslash \; S$ \\
        \> $S := S \cup \{s\}$ \\
        \> return $s$
    }
\]

(h)
\[
    \fcodebox{
        $k_{right} \gets \{0, 1\}^{n}$ \\ \\
        \underline{$\subname{CHALLENGE}(m_{1}{\cdots}m_{\ell})$:} \\
        \> $iv' := $\highlightline{$\subname{SAMP}()$} \\
        \> $c := $ CBC encryption of $m_{1}{\cdots}m_{\ell}$ \\
        \>\> under key $k_{right}$ \\
        \>\> using initialization vector $iv'$ \\
        \> return $c_{1}{\cdots}c_{\ell}$ \\
    }
    \link
    \fcodebox{
        $S := \emptyset$ \\ \\
        \underline{$\subname{SAMP}()$:} \\
        \> $s \gets \{0, 1\}^n \; \backslash \; S$ \\
        \> $S := S \cup \{s\}$ \\
        \> return $s$
    }
\]

(i)
\[
    \fcodebox{
        $k_{right} \gets \{0, 1\}^{n}$ \\ \\
        \underline{$\subname{CHALLENGE}(m_{1}{\cdots}m_{\ell})$:} \\
        \> $iv' := \subname{SAMP}()$ \\
        \> $c := $ CBC encryption of $m_{1}{\cdots}m_{\ell}$ \\
        \>\> under key $k_{right}$ \\
        \>\> using initialization vector $iv'$ \\
        \> return $c_{1}{\cdots}c_{\ell}$ \\
    }
    \link
    \hlcodebox{
        \underline{$\subname{SAMP}()$:} \\
        \> $s \gets \{0, 1\}^n$ \\
        \> return $s$
    }
\]

(j)
\[
    \fcodebox{
        $k_{right} \gets \{0, 1\}^{n}$ \\ \\
        \underline{$\subname{CHALLENGE}(m_{1}{\cdots}m_{\ell})$:} \\
        \> \highlightline{$iv' \gets \{0, 1\}^n$} \\
        \> $c := $ CBC encryption of $m_{1}{\cdots}m_{\ell}$ \\
        \>\> under key $k_{right}$ \\
        \>\> using initialization vector $iv'$ \\
        \> return $c_{1}{\cdots}c_{\ell}$ \\
    }
\]\\

\noindent
The CPA\$ security relies on $iv'$s being pseudorandom. Since each $iv'$
is a result from the call to the PRF, the outputs will only be
pseudorandom if the inputs to the PRF are distinct. Because this variant
does not allow for $iv$s to be repeated, they are guaranteed to be
distinct. \\

\noindent
Justification of steps:
\begin{itemize}
    \addtolength{\leftskip}{24pt}
    \item[(a)] This is the inlining of the encryption scheme to the
        CPA\$ security variant.
    \item[(a)$\Rightarrow$(b)] $k$ is split into it's two halves. No
        affect on the program output distribution.
    \item[(b)$\Rightarrow$(c)] The PRF call is factored out into it's
        own function along with $k_{left}$. No affect on the program.
    \item[(c)$\Rightarrow$(d)] $\lib{prf-real}^F$ replaced with
        $\lib{prf-rand}^F$. Program output distribution changes by a
        negligible amount.
    \item[(d)$\Rightarrow$(e)] Factoring out the sampling of random
        blocks. No effect on program.
    \item[(e)$\Rightarrow$(f)] Sampling without replacement instead of
        with replacement. Negligible effect to output distribution.
    \item[(g)] Because $iv$ are not allowed to be reused,
        $\subname{QUERY}$ is only called with a non-defined $T[iv]$.
        Removing the if-statement does not affect the output
        distribution.
    \item[(h)] $iv$ is now never used, so it is removed from the
        definition. $\subname{QUERY}$ merely calls $\subname{SAMP}$, so
        this redundancy is removed.
    \item[(h)$\Rightarrow$(i)] Sampling with replacement instead of
        sample without replacement. Negligible effect on program's
        output distribution.
    \item[(i)$\Rightarrow$(j)] Call to $\subname{SAMP}$ inlined. No
        effect on program.
\end{itemize}

Program (j) now has pseudorandom ciphertexts under chosen plaintext and
{\it chosen-but-never-repeated-IV} attacks.

\paragraph{3} A distinguisher showing $\Sigma^2$ does not have CCA
security is as follows: \\

\[
    \fcodebox{
        \underline{$\subname{A}$:} \\
        \> $k \gets \Sigma^2$.KeyGen \\
        \> $c_1, c_2 = \Sigma^2.\subname{Enc}(k, m \in \{0, 1\}^{2n})$ \\
        \> $m = \subname{Dec}(k, (c_1 \oplus c_2, c_1 \oplus c_2))$ \\
        \> return $m_{left} \stackrel{?}{=} m_{right}$
    }
\]\\

\noindent
This distinguisher relies on the fact that $k$ is used twice to decrypt.
This allows an adversary to pass $(c_1 \oplus c_2)$ twice in 
$\subname{Dec}$. Because $(c_1 \oplus c_2)$ was not an encryption string
generated by $\subname{Enc}$ it will not result in an error within
$\subname{Dec}$. The output from $\subname{Dec}$ will be a message
concatenated with itself, hence $m_{left} = m_{right}$. \\

\noindent
The bias of $A$ is as follows:

\begin{align*}
    \bias(A,\lib{cca-real}^{\Sigma^2},\lib{cca-rand}^{\Sigma^2}) &=
    |\Pr[A \link \lib{cca-real}^{\Sigma^2} \mbox{ outputs 1}] - \Pr[A \link \lib{cca-rand}^{\Sigma^2} \mbox{ outputs 1}]| \\
    &= | 1 - \frac{1}{2^{2n}} | \\
    &= \text{Non-negligible}
\end{align*}

\end{document}
