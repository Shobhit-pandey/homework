\documentclass[12pt]{article}

%% minimal example of some latex stuff

\usepackage{xspace,graphicx,amsmath,amssymb,xcolor}

\newcommand{\pct}{\mathbin{\%}}

\newcommand{\K}{\mathcal{K}}
\newcommand{\M}{\mathcal{M}}
\newcommand{\C}{\mathcal{C}}
\newcommand{\Z}{\mathbb{Z}}

\newcommand{\Enc}{\text{\sf Enc}}
\newcommand{\Dec}{\text{\sf Dec}}
\newcommand{\KeyGen}{\text{\sf KeyGen}}

% fancy script L
\usepackage[mathscr]{euscript}
\renewcommand{\L}{\ensuremath{\mathscr{L}}\xspace}
\newcommand{\lib}[1]{\ensuremath{\L_{\textsf{#1}}}\xspace}

\newcommand{\myterm}[1]{\ensuremath{\text{#1}}\xspace}
\newcommand{\bias}{\myterm{bias}}
\newcommand{\link}{\diamond}
\newcommand{\subname}[1]{\ensuremath{\textsc{#1}}\xspace}


%%% code boxes

\newcommand{\fcodebox}[1]{%
    \framebox{\codebox{#1}}%
}
\newcommand{\hlcodebox}[1]{%
    \fcolorbox{black}{myyellow}{\codebox{#1}}%
}


\usepackage{varwidth}

\newcommand{\codebox}[1]{%
        \begin{varwidth}{\linewidth}%
        \begin{tabbing}%
            ~~~\=\quad\=\quad\=\quad\=\kill
            #1
        \end{tabbing}%
        \end{varwidth}%
}


\definecolor{myyellow}{HTML}{F5F589}

\newcommand{\mathhighlight}[1]{\basehighlight{$#1$}}
\newcommand{\highlightline}[1]{%\raisebox{0pt}[-\fboxsep][-\fboxsep]{
    \hspace*{-\fboxsep}\basehighlight{#1}%
%}
}
\newcommand{\basehighlight}[1]{\colorbox{myyellow}{#1}}


\title{Cryptography: HW4}
\author{Trevor Bramwell}
\date{
    Professor Mike Rosulek\\
    February 5, 2015
}
%%%%

\begin{document}
\maketitle


\paragraph{1a} Consider a variant of the CPA definition for CBC
encryption in which an adversary is able to see the next IV before
choosing a plaintext.

\begin{center}
\[
    \framebox{
        \codebox{
            $k \gets \Sigma.$KeyGen \\
            $iv \gets \{0, 1\}^n$ \\
            \\
            \underline{$\subname{CHALLENGE}(m_L, m_R)$:} \\
            \> return null if $|m_L| \neq |m_R|$ \\
            \> $c_0, c_1, \cdots, c_{\ell} := \Sigma.\subname{Enc}(k, iv, m_L)$ \\
            \> return $c_0, c_1, \cdots, c_{\ell}$\\
            \\
            \underline{$\subname{VEC}$:} \\
            \> return $iv$
        }
    }
    \qquad
    \framebox{
        \codebox{
            $k \gets \Sigma.$KeyGen \\
            $iv \gets \{0, 1\}^n$ \\
            \\
            \underline{$\subname{CHALLENGE}(m_L, m_R)$:} \\
            \> return null if $|m_L| \neq |m_R|$ \\
            \> $c_0, c_1, \cdots, c_{\ell} := \Sigma.\subname{Enc}(k, iv, m_R)$ \\
            \> return $c_0, c_1, \cdots, c_{\ell}$\\
            \\
            \underline{$\subname{VEC}$:} \\
            \> return $iv$
        }
    }
\]
\end{center}

\paragraph{1b} This modification breaks CPA security for CBC encryption.
This can be seen by the following distinguisher $A$.

\[
    \fcodebox{
        \underline{$\subname{A}$:} \\
        \> $iv \gets \subname{VEC}$ \\
        \> $m_L = iv, m_2, m_3, \cdots, m_{\ell}$ \\
        \> $c_0, c_1, c_2, \cdots, c_{\ell} := \subname{CHALLENGE}(m_L, m_L)$ \\
        \> return $c_1 \oplus c_2 \stackrel{?}{=} m_2$
    }
\]

By replacing $m_1$ with $iv$ the first call to $F_k$ is $F(k, 0^n)$.
This results in ...

\paragraph{3} A distinguisher showing $\Sigma^2$ does not have CCA
security is as follows: \\

\[
    \fcodebox{
        \underline{$\subname{A}$:} \\
        \> $k \gets \Sigma^2$.KeyGen \\
        \> $c_1, c_2 = \Sigma^2.\subname{Enc}(k, m \in \{0, 1\}^{2n})$ \\
        \> $m = \subname{Dec}(k, (c_1 \oplus c_2, c_1 \oplus c_2))$ \\
        \> return $m_{left} \stackrel{?}{=} m_{right}$
    }
\]\\

\noindent
This distinguisher relies on the fact that $k$ is used twice to decrypt.
This allows an adversary to pass $(c_1 \oplus c_2)$ twice in 
$\subname{Dec}$. Because $(c_1 \oplus c_2)$ was not an encryption string
generated by $\subname{Enc}$ it will not result in an error within
$\subname{Dec}$. The output from $\subname{Dec}$ will be a message
concatenated with itself, hence $m_{left} = m_{right}$. \\

\noindent
The bias of $A$ is as follows:

\begin{align*}
    \bias(A,\lib{cca-real}^{\Sigma^2},\lib{cca-rand}^{\Sigma^2}) &=
    |\Pr[A \link \lib{cca-real}^{\Sigma^2} \mbox{ outputs 1}] - \Pr[A \link \lib{cca-rand}^{\Sigma^2} \mbox{ outputs 1}]| \\
    &= | 1 - \frac{1}{2^{2n}} | \\
    &= \text{Non-negligible}
\end{align*}

\end{document}
