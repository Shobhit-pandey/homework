\documentclass[12pt]{article}

%% minimal example of some latex stuff

\usepackage{xspace,graphicx,amsmath,amssymb,xcolor}

\newcommand{\pct}{\mathbin{\%}}

\newcommand{\K}{\mathcal{K}}
\newcommand{\M}{\mathcal{M}}
\newcommand{\C}{\mathcal{C}}
\newcommand{\Z}{\mathbb{Z}}

\newcommand{\Enc}{\text{\sf Enc}}
\newcommand{\Dec}{\text{\sf Dec}}
\newcommand{\KeyGen}{\text{\sf KeyGen}}

% fancy script L
\usepackage[mathscr]{euscript}
\renewcommand{\L}{\ensuremath{\mathscr{L}}\xspace}
\newcommand{\lib}[1]{\ensuremath{\L_{\textsf{#1}}}\xspace}

\newcommand{\myterm}[1]{\ensuremath{\text{#1}}\xspace}
\newcommand{\bias}{\myterm{bias}}
\newcommand{\link}{\diamond}
\newcommand{\subname}[1]{\ensuremath{\textsc{#1}}\xspace}


%%% code boxes

\newcommand{\fcodebox}[1]{%
    \framebox{\codebox{#1}}%
}
\newcommand{\hlcodebox}[1]{%
    \fcolorbox{black}{myyellow}{\codebox{#1}}%
}


\usepackage{varwidth}

\newcommand{\codebox}[1]{%
        \begin{varwidth}{\linewidth}%
        \begin{tabbing}%
            ~~~\=\quad\=\quad\=\quad\=\kill
            #1
        \end{tabbing}%
        \end{varwidth}%
}


\definecolor{myyellow}{HTML}{F5F589}

\newcommand{\mathhighlight}[1]{\basehighlight{$#1$}}
\newcommand{\highlightline}[1]{%\raisebox{0pt}[-\fboxsep][-\fboxsep]{
    \hspace*{-\fboxsep}\basehighlight{#1}%
%}
}
\newcommand{\basehighlight}[1]{\colorbox{myyellow}{#1}}


\title{Cryptography: HW3}
\author{Trevor Bramwell}
\date{
    Professor Mike Rosulek\\
    January 30, 2015
}
%%%%

\begin{document}
\maketitle


\paragraph{1} Secure PRF: F' \\
% Need to refactor to use k, r and QUERY twice to be able to swap
% prg-real/rand
% Use CHALLENGE to show CPA$ ?

Given $F'(k, r) = G(F(k,r))$ we will show that it is a secure PRF.\\

\noindent
First we rewrite $F'(k,r)$ as an algorithm. Then after following a
series of steps that do not affect the output of the program, we will
arive at a double-length PRF.\\

(a)
\[
    \fcodebox{
        \underline{$\subname{F'}(k, r)$:} \\
        \> $return \; \subname{G}(\subname{F}(k, r))$
    }
\]

(b)
\[
    \fcodebox{
        \underline{$\subname{F'}(k, r)$:} \\
        \> $s \gets \subname{F}(k, r)$ \\
        \> $return \; \subname{G}(s)$
    }
\]

(c)
\[
    \fcodebox{
        \underline{$\subname{F'}(k, r)$:} \\
        \> $s \gets \{0, 1\}^n$ \\
        \> $return \; \subname{G}(s)$
    }
\]

(d)
\[
    \fcodebox{
        \underline{$\subname{F'}(k, r)$:} \\
        \> $s \gets \{0, 1\}^{n+\ell}$ \\
        \> $return \; s$
    }
\]

\begin{itemize}
    \addtolength{\leftskip}{24pt}
    \item[(a)] This is the formation of $F'$ as a algorithm.
    \item[(a)$\Rightarrow$(b)] $s$ is used to hold the output of
        $F(k, r)$. No effect on program.
    \item[(b)$\Rightarrow$(c)] Because we are given $F$ is a secure PRF,
        meaning it is indistinguishable from randomness,
        we can replace the call to $F(k, r)$ with a random string.
    \item[(c)$\Rightarrow$(d)] By the same logic in (b)$\Rightarrow$(c)
        we can replace $G$.
    \item[(d)] $F'(k, r)$ is a secure PRF.
\end{itemize}


\paragraph{2} Distinguisher for insecure $F'$ \\

\noindent
Given $F'(k, r) = F(k, r) \oplus F(k, \bar{r})$, there exists the
following distinguisher $A$ that can tell with non-negligible
probability the use of $\lib{prg-real}^{F'}$ over $\lib{prg-rand}^{F'}$.


\[
    \fcodebox{
        \underline{$\subname{A}()$:} \\
        \> $k \gets \{0, 1\}^n$ \\
        \> $l := \subname{F'}(k, 0^n)$ \\
        \> $m := \subname{F'}(k, 1^n)$ \\
        \> $return (l = m)$
    }
\]

\noindent
The bias for $A$ is:

\begin{align*}
    \bias(A, \lib{prg-real}^{F'}, \lib{prg-rand}^{F'}) 
             &= | Pr[A \link \lib{prg-real}^{F'} \mbox{ outputs 1}] - Pr[A
\link \lib{prg-rand}^{F'}\mbox{ outputs 1}] | \\
             &= | 1 - \frac{1}{2^n} | \\
             &= \text{Non-negligible amount}
\end{align*}

\paragraph{3} Insecurity of a 2-Round Feistel Network \\

\noindent
Given two distinct strings $L_1$ and $L_2$, the following distinguisher
can be used in a CPA attack against a 2-Round Feistel Network.

\[
    \fcodebox{
        \underline{$\subname{A}()$:} \\
        \> $R \gets \{0, 1\}^n$ \\
        \> $(a_L, a_R) := \subname{F}(L_1, R)$ \\
        \> $(b_L, b_R) := \subname{F}(L_2, R)$ \\
        \> return $a_L \oplus b_L = L_1 \oplus L_2$
    }
\]\\

\noindent
In a 2-round Feistel network, with distinct $f$ round functions, the output of $F(L, R)$ is a 2-tuple:
\[
    (f_1(R) \oplus L, \; f_2(f_1(R) \oplus L) \oplus R.
\]

Calling $F(L, R)$ twice, with a constant $R$, and distinct $L_i$ provides
the unique property of:

\[
    (f_1(R) \oplus L_1) \oplus (f_1(R) \oplus L_2)
\]

which reduces to:

\[
    L_1 \oplus L_2
\]

\paragraph{4} PRP Distinguisher

\[
    \fcodebox{
        \underline{$\subname{A}()$:} \\
        \> $k \gets \subname{KeyGen}$ \\
        \> $(r_1, z_1) := \subname{Enc}(k, 0^n)$ \\
        \> $(r_2, z_2) := \subname{Enc}(k, 0^n)$ \\
        \> return $(r_1 \oplus z_1 \oplus r_2 \oplus z_2) = 0$
    }
\]\\

\noindent
To show that $F$ does \textbf{not} have CPA-security, the above
distinguisher is given. $\subname{Enc}$ returns a 2-tuple: $(r, z)$ with
the property of $r \oplus z = F(k, m)$. Given two calls to
$\subname{Enc}$ the distinguisher is able to check the equality of $F(k, m)$.
$F(k, m)$ will not be equal when using $\lib{prg-rand}$.\\

\noindent
The bias for $A$ is:

\begin{align*}
    \bias(A, \lib{prg-real}^{F}, \lib{prg-rand}^{F}) 
             &= | Pr[A \link \lib{prg-real}^{F} \mbox{ outputs 1}] - Pr[A
\link \lib{prg-rand}^{F}\mbox{ outputs 1}] | \\
             &= | 1 - \frac{1}{2^n} | \\
             &= \text{Non-negligible amount}
\end{align*}


\end{document}
