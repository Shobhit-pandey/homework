\documentclass[12pt]{article}

%% minimal example of some latex stuff

\usepackage{xspace,graphicx,amsmath,amssymb,xcolor}
\usepackage{numprint}

\newcommand{\pct}{\mathbin{\%}}

\newcommand{\K}{\mathcal{K}}
\newcommand{\M}{\mathcal{M}}
\newcommand{\C}{\mathcal{C}}
\newcommand{\Z}{\mathbb{Z}}

\newcommand{\Enc}{\text{\sf Enc}}
\newcommand{\Dec}{\text{\sf Dec}}
\newcommand{\KeyGen}{\text{\sf KeyGen}}

% fancy script L
\usepackage[mathscr]{euscript}
\renewcommand{\L}{\ensuremath{\mathscr{L}}\xspace}
\newcommand{\lib}[1]{\ensuremath{\L_{\textsf{#1}}}\xspace}

\newcommand{\myterm}[1]{\ensuremath{\text{#1}}\xspace}
\newcommand{\bias}{\myterm{bias}}
\newcommand{\link}{\diamond}
\newcommand{\subname}[1]{\ensuremath{\textsc{#1}}\xspace}


%%% code boxes

\newcommand{\fcodebox}[1]{%
    \framebox{\codebox{#1}}%
}
\newcommand{\hlcodebox}[1]{%
    \fcolorbox{black}{myyellow}{\codebox{#1}}%
}


\usepackage{varwidth}

\newcommand{\codebox}[1]{%
        \begin{varwidth}{\linewidth}%
        \begin{tabbing}%
            ~~~\=\quad\=\quad\=\quad\=\kill
            #1
        \end{tabbing}%
        \end{varwidth}%
}


\definecolor{myyellow}{HTML}{F5F589}

\newcommand{\mathhighlight}[1]{\basehighlight{$#1$}}
\newcommand{\highlightline}[1]{%\raisebox{0pt}[-\fboxsep][-\fboxsep]{
    \hspace*{-\fboxsep}\basehighlight{#1}%
%}
}
\newcommand{\basehighlight}[1]{\colorbox{myyellow}{#1}}


\title{Cryptography: HW6}
\author{Trevor Bramwell}
\date{
    Professor Mike Rosulek\\
    March 5, 2015
}
%%%%

\begin{document}
\maketitle


\paragraph{1} 

\paragraph{2}

\paragraph{3a}

Given an RSA modulus $N$ and $\phi{(N)}$, it is possible to factor $N$
easily without computing $d$ or finding a non-trivial square root of
unity.
\\\\
To start with, we know:

\begin{align*}
    N &= pq \\
    \phi{(N)} &= (p - 1)(q - 1)
\end{align*}

Factoring $(p - 1)(q - 1)$ and doing some algebra, we arrive at:

\[
    N - \phi{(N)} + 1 = q + p
\]

This allows us to now use the quadratic equation:

\[
    (x - q)(x - p) = 0
\]

and find roots equal to $q$ and $p$. Factoring we have:

\begin{align*}
    (x - q)(x - p) &= x^2 - qx - px + pq \\
    &= x^2 + (-q - p)x + pq
\end{align*}

This provides us with the quadractic equations constants:

\begin{align*}
    a &= 1 \\
    b &= (-q - p) \\
   -b &= (q + p) \\
    c &= pq = N
\end{align*}

Which when plugged into the quadratic equation, give us:

\[
    p,q = \frac{(q + p) \pm \sqrt{(-q-p)^2-4N}}{2}
\]


\paragraph{3b}

The prime factors ($p$, $q$) of $N$ are:

\begin{verbatim}
p = 
1050633452306161046573480136342977017860065421289419090831011858888328139819275
4157844910388599060640609121542398864123229041436886066123523490551620883141000
2395649542617648887800946357680235133745987328106517234870888931439344639465469
355722657253870347038180634028206997149618094761699250765049161158010333

q =
9352419571367596976939279289180413447699593351180217187280653361389412531812702
8917922827635059951366086790265075822909039161954650657371451399404924083352681
6235957699648688337327410429196529474184107065919242793511757586398184746028719
1342833560356018485176253542816589237713501814110266911435430483265167
\end{verbatim}

\end{document}
