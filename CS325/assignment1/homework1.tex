\documentclass[12pt]{article}

\usepackage{amsmath}
\usepackage{amssymb}

\title{CS325 Winter 2013: HW 1}
\author{
    Daniel Reichert \\
    Trevor Bramwell
}
\date{\today}

\newcommand{\BigO}[1]{\ensuremath{\operatorname{O}\left(#1\right)}}

% Big O: \BigO
% Big Omega: \Omega
% Big Theta: \Theta

% Examples:
%
%   $\BigO{n}$
%   $\Omega(n\log{n})$
%   $\Theta(\log{2n})$


\begin{document}
\maketitle

\section*{0.2 from text book}
Show that, if $c$ is a positive real number, then $g(n)=1+c+c^2+\cdots+c^n$ is:

\begin{enumerate}
\item $\Theta(1) \text{ if } c < 1$
\item $\Theta(n) \text{ if } c = 1$
\item $\Theta(c^n) \text{ if } c > 1$
\end{enumerate}

The moral: in big-$\Theta$ terms, the sum of a geometric series is simply the
first term if the series is
strictly decreasing, the last term if the series is strictly increasing,
or the number of terms if the
series is unchanging.

\section*{0.3(a) from text book}

The Fibonacci numbers $F_0, F_1, F_2$, \ldots, are defined by the rule
\[ F_0 = 0, F_1 = 1, F_n = F_{n - 1} + F_{n - 2}. \]
In this problem we will confirm that this sequence grows exponentially
fast and obtain some
bounds on its growth.
\begin{enumerate}
\item Use induction to prove that $F_n \geq 2^{0.5n}$ for $n \geq 6$.
\end{enumerate}

% 83
\section*{2.3 from text book}
Section 2.2 describes a method for solving recurrence relations which is
based on analyzing the recursion tree and deriving a formula for the
work done at each level.  Another (closely related) method is to expand
out the recurrence a few times, until a pattern emerges. For instance,
let’s start with the familiar $T(n) = 2T(n/2) + \BigO{n}$. Think of
$\BigO{n}$ as being $\leq cn$ for some constant $c$,
so: $T(n) \leq 2T(n/2) + cn$. 
By repeatedly applying this rule, we can bound $T(n)$ in terms of $T(n/2)$,
then $T(n/4)$, then $T(n/8)$, and so on, at each step getting closer to
the value of $T(\cdot)$ we do know,
namely $T(1) = \BigO{1}$.
\begin{align*}
T (n) &\leq 2T (n/2) + cn \\
&\leq 2[2T(n/4) + cn/2] + cn = 4T(n/4) + 2cn \\
&\leq 4[2T(n/8) + cn/4] + 2cn = 8T(n/8) + 3cn \\
&\leq 8[2T(n/16) + cn/8] + 3cn = 16T(n/16) + 4cn \\
&  \vdots
\end{align*}

A pattern is emerging\ldots the general term is 
\[T(n) \leq 2^kT(n/2^k) + kcn.\]

Plugging in $k = \log_2{n}$, we get $T(n) \leq nT(1) + cn\log_2{n} = \BigO{n\log{n}}.$
\begin{enumerate}
\item Do the same thing for the recurrence $T(n) = 3T(n/2) + \BigO{n}$. What
is the general $k$th term
in this case? And what value of $k$ should be plugged in to get the
answer?

\item Now try the recurrence $T(n) = T( n −1 ) + \BigO{1}$, a case which is not
covered by the master theorem. Can you solve this too?
\end{enumerate}

\section*{2.4 from text book}

\section*{2.17 from text book}

\end{document}
