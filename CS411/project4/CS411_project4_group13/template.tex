\documentclass[letterpaper,10pt]{article}

\usepackage{graphicx}                                        
\usepackage{amssymb}                                         
\usepackage{amsmath}                                         
\usepackage{amsthm}                                          

\usepackage{alltt}                                           
\usepackage{float}
\usepackage{color}
\usepackage{url}

\usepackage{balance}
\usepackage[TABBOTCAP, tight]{subfigure}
\usepackage{enumitem}
\usepackage{pstricks, pst-node}

\usepackage{geometry}
\geometry{textheight=8.5in, textwidth=6in}

\newcommand{\cred}[1]{{\color{red}#1}}
\newcommand{\cblue}[1]{{\color{blue}#1}}

\usepackage{hyperref}
\usepackage{geometry}

\def\name{Group 13: Brian Cox, Matthew Okazaki, Trevor Bramwell}

%pull in the necessary preamble matter for pygments output
\input{pygments.tex}
\parindent = 0.0 in
\parskip = 0.1 in

%% The following metadata will show up in the PDF properties
\hypersetup{
  colorlinks = true,
  urlcolor = black,
  pdfauthor = {\name},
  pdfkeywords = {cs411 ``operating systems'' Slob},
  pdftitle = {CS 411 Project 4: Replacing Slab with Best-fit Slob} 
  pdfsubject = {CS 411 Project 4},
  pdfpagemode = UseNone
}
\begin{document}

\title{CS411 Project 4}


\date{Trevor Bramwell, Brian Cox, Matthew Okazaki }
\maketitle
\begin{abstract}
This document describes the design, implementation, testing, and obstacles
faced in implementing best-fit slob for the Linux Kernel. These challenges
were faced by Group 13 of the Fall 2012 CS411 class with the assistance
of Squidly 13. This group consists of Brian Cox, Matthew Okazaki,
Trevor Bramwell, and Squidly. 
\end{abstract}

\section*{Design }

At the beginning of this project we already understood several things
about the project and about the kernel:
\begin{itemize}
\item The slob was located in the mm directory 
\item The slab, located in the same directory as the slob, was the current
memory management system 
\item The slob currently implemented first-fit and we wanted best-fit 
\item We needed a way to show that the best-fit slob used more memory, was
less efficient, than first-fit slob.
\end{itemize}
Using the same general concept that everything is referenced in the
kconfig, we used the config menu to disable the slab and enable the
slob. this had minor tediousness to it, but we understood what to
do. Reading through Tannenbaum we found descriptions of best-fit,
worst-fit, last-fit, and first-fit. Comparing first-fit to any of
the others showed there was only a minor difference between them,
all versions of memory management, except first-fit, must iterate
through all the pages. With implementing best-fit, we need to change
the slob so it didn\textquoteright{}t allocate as soon as it found
a page it fit in, and we needed to find free space as close to the
object\textquoteright{}s size as possible. Lastly, we had to set up
a pair of system calls and a program. The system calls would tell
us how much space was being used and how much was free, while the
program would take the information from the system calls and find
the percentage of free space and used space. 


\section*{Implementation }

We implemented the project by changing the slob\_alloc function in
the slob.c file. The function would already loop through the link
list in order to find the first page that could fit and allocate the
memory. We modified this in order to loop through all the pages in
the linked list and find the page that would best fit the unit. We
also added system calls within the project in order to retrieve information
about the memory allocation. Because of this, we ended up modifying
several other files and creating two system call functions in slob.c.
The other files that were modified were the unistd.h, the syscall.h,
and the syscall\_table\_32.s. These files were necessary in order
to correctly implement the system call. 


\section*{Testing }

Our testing consisted of making sure the kernel ran and using the
system calls and program we made to check the memory. Implementing
two system calls, sys\_get\_slob\_amt\_free and sys\_get\_slob\_amt\_claimed,
these two system calls would return the amount of free space and the
amount of total space. The program would make the call to the two
system calls and process the different values to display the percentage
free, percentage used, and total volume. 


\section*{Difficulties }

while working on the project we were uncertain if we needed to find
the best fitting page or the best fitting block. After a short burst
of research we found that only worrying about the page would suffice.
Following that, the changes to the kernel would not boot. Thanks to
the large number of hours Trevor put into it, he managed to isolate
the problem and the kernel was up and running. Lastly the syscalls
used for getting the memory information gave extra issues.

\section*{Code}
%input the pygmentized output of report_code.c, using a (hopefully) unique name
%this file only exists at compile time. Feel free to change that.
\input{__report_code.c.tex}
\end{document}
