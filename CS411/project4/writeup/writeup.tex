\documentclass[letterpaper,10pt,titlepage]{article}

\usepackage{graphicx}                                        
\usepackage{amssymb}                                         
\usepackage{amsmath}                                         
\usepackage{amsthm}                                          

\usepackage{alltt}                                           
\usepackage{float}
\usepackage{color}
\usepackage{url}

\usepackage{balance}
\usepackage[TABBOTCAP, tight]{subfigure}
\usepackage{enumitem}
\usepackage{pstricks, pst-node}

\usepackage{geometry}
\geometry{textheight=8.5in, textwidth=6in}

\newcommand{\cred}[1]{{\color{red}#1}}
\newcommand{\cblue}[1]{{\color{blue}#1}}

\usepackage{hyperref}
\usepackage{geometry}

\def\name{Group 13: Brian Cox, Matthew Okazaki, Trevor Bramwell}

%pull in the necessary preamble matter for pygments output
%\input{pygments.tex}
\parindent = 0.0 in
\parskip = 0.1 in

%% The following metadata will show up in the PDF properties
\hypersetup{
  colorlinks = true,
  urlcolor = black,
  pdfauthor = {\name},
  pdfkeywords = {cs411 ``operating systems'' SLOB Best Fit},
  pdftitle = {CS 411 Project 4:SLOB Best Fit} 
  pdfsubject = {CS 411 Project 4},
  pdfpagemode = UseNone
}

\def\name{Trevor Bramwell}
\title{CS411: Individual Writeup - Group 13\\
    Project 4: SLOB Best Fit}
\author{\name}

\begin{document}
\maketitle

%What do you think the main point of this assignment is?
\section*{Introduction}

The main point of this assignment was to gain an understanding of
memory management in the Linux kernel and memory algorithms.

This was achieved by having our group write a Best-Fit algorithm
for the SLOB memory manager. The Best-Fit algorithm in the SLOB performs
very badly compared to the First-Fit algorithm, yet is easy to
implement.

%How did you personally approach the problem?
%Design decisions, algorithm, etc.
\section*{Approach}

I approached the problem by first browsing the source code of
\emph{mm/slob.c} and \emph{mm/slab.c} to get a feeling for how the Linux
memory managers were setup. Second, I took an agile approach to solving the
problem by first making sure I could boot into a SLOB kernel. Third, I
worked on setting up the system calls, so as to first collect the
performance measurements from the First-Fit algorithm. Fourth, I went
about creating a program to collect this information from the system
calls and output in a readable manner. Lastly, I implemented the
Best-Fit algorithm in the SLOB, and recorded it's performance.

\section*{Design}

Designing the Best-Fit algorithm was quite easy. In essence it was
finding a minimum. Designing the testing program was even easier as I
already had an idea of what information I was going to provide to user
space. I assumed I had that information, and worked accordingly, only
finally testing it after I had the system calls in place. 


% How did you ensure your solution was correct?
% Testing details, for instance.
\section*{Testing}

To ensure that the Best-Fit algorithm was working as expected, all that
was required was to make sure the Kernel booted. A kernel can't get very
far at all without being able to get and store data in memory.
Thus, after I had implemented Best-Fit, and reached Init in the startup
process, I knew my implementation was correct. 

For testing the fragmentation, I began with assumptions. I knew the
Best-Fit algorithm produced more fragmentation, and I assumed the
First-Fit algorithm fragmentation had to be pretty low, or else no one
would use it. Both these assumptions were correct as I ended up getting
around 10\% fragmentation from First-Fit and approximately 90\% from
Best-Fit.

\section*{Learning}


A large time on this project was spent getting just SLOB to boot. After
trying several configurations over two days of compiling, I finally got
a Kernel with SLOB enabled to boot. The problem was not very clear,
though went away once I 'reset' my files by copying in an original
version of \emph{mm/slob.c} from the 3.0.4 Kernel.

This project gave me a much greater respect for Kernel configuration. I
also learned a great deal about memory management in the Kernel, and
feel more comfortable working on things in the Kernel.


\end{document}
