\documentclass[letterpaper,10pt,titlepage]{article}

\usepackage{graphicx}                                        
\usepackage{amssymb}                                         
\usepackage{amsmath}                                         
\usepackage{amsthm}                                          

\usepackage{alltt}                                           
\usepackage{float}
\usepackage{color}
\usepackage{url}

\usepackage{balance}
\usepackage[TABBOTCAP, tight]{subfigure}
\usepackage{enumitem}
\usepackage{pstricks, pst-node}

\usepackage{geometry}
\geometry{textheight=8.5in, textwidth=6in}

\newcommand{\cred}[1]{{\color{red}#1}}
\newcommand{\cblue}[1]{{\color{blue}#1}}

\usepackage{hyperref}
\usepackage{geometry}

\def\name{Group 13: Brian Cox, Matthew Okazaki, Trevor Bramwell}

%pull in the necessary preamble matter for pygments output
%\input{pygments.tex}
\parindent = 0.0 in
\parskip = 0.1 in

%% The following metadata will show up in the PDF properties
\hypersetup{
  colorlinks = true,
  urlcolor = black,
  pdfauthor = {\name},
  pdfkeywords = {cs411 ``operating systems'' Ram Disk Driver},
  pdftitle = {CS 411 Project 3:Encrypted Ram Disk Driver} 
  pdfsubject = {CS 411 Project 3},
  pdfpagemode = UseNone
}

\def\name{Trevor Bramwell}
\title{CS411: Individual Writeup - Group 13\\
    Project 3: Encrypted Ram Disk Driver}
\author{\name}

\begin{document}
\maketitle

%What do you think the main point of this assignment is?
\section*{Introduction}

The main point of this assignment was to gain an understanding of
Linux device drivers, and to understand how to work with the Linux
Crypto API.

This was achieved by having our group write a Encrypted RAM Disk 
Driver. This driver works by presenting the user with a disk that is
stored completely in RAM and is encrypted when written to and
decrypted when read from.

%How did you personally approach the problem?
%Design decisions, algorithm, etc.
\section*{Approach}

My approach to the problem was to first review the assignment. Second, I
for some external resources. This lead me to find the final project for
a previous CS411 class which contained a wealth of more information. I
informed my group of the discovery, and we followed the new instructions
to read \emph{Linux Device Drivers 3rd Edition} (LDD3).

\section*{Design}

When designing the RAM disk driver, I basically followed all the
information in LDD3. Knowing that sitting down and writing out all the
source code from the book would be tedious, I opted to find a copy of
the \emph{sbull.c} file already written. This lead me to find a
simple driver which had been updated for the 3.0 kernel.

Very little had to be done in terms of design, since most of the
architecture was handled by the \emph{sbull.c} driver from the book.

% How did you ensure your solution was correct?
% Testing details, for instance.
\section*{Testing}

To test the new RAM disk driver I formated the new device
\emph{/dev/sbd0} that
was created when the sbd kernel module was loaded. Then I created a new
partition on the device, and also created an ext2 file system. I then
created a new file, wrote out some text, closed the file, and then
'cat'ed it to stdout. 

Seeing as the printk statements in the driver, along with the data I
wrote to the file was outputting correctly. I deduced that the
implementation was correct and working accordingly.

\section*{Learning}

From this assignment I learned a decent bit about how cryptography is
handled in the Kernel. Mainly that there are several different methods
of using a single algorithm, such as AES. The kernel allows you to use
any of the block modes: CBC, ECB, CTS, CTR, XTS, PCBC, and LRW. On top
of that you have your choice of algorithm from: AES, Blowfish, DES,
Anubis, Twofish, ARC4, or CAST; just to name a few.

Originally when I tried using the AES algorithm, the module would not
load. I found that this was because the AES module had to be loaded in
as well, and the Kernel was not configured to even build it!

Overall I learned a great deal from this assignment about writing a
Linux Kernel driver, encryption, and loading and compiling modules.

\end{document}
