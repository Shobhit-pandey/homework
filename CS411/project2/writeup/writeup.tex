\documentclass[letterpaper,10pt,titlepage]{article}

\usepackage{graphicx}                                        
\usepackage{amssymb}                                         
\usepackage{amsmath}                                         
\usepackage{amsthm}                                          

\usepackage{alltt}                                           
\usepackage{float}
\usepackage{color}
\usepackage{url}

\usepackage{balance}
\usepackage[TABBOTCAP, tight]{subfigure}
\usepackage{enumitem}
\usepackage{pstricks, pst-node}

\usepackage{geometry}
\geometry{textheight=8.5in, textwidth=6in}

%random comment

\newcommand{\cred}[1]{{\color{red}#1}}
\newcommand{\cblue}[1]{{\color{blue}#1}}

\usepackage{hyperref}

\def\name{Trevor Bramwell}

%% The following metadata will show up in the PDF properties
\hypersetup{
  colorlinks = true,
  urlcolor = black,
  pdfauthor = {\name},
  pdfpagemode = UseNone
}

\title{CS411: Individual Writeup - Group 13\\
    Project 2: SSTF IO Scheduler}
\date{}
\author{\name}

\begin{document}
\maketitle

\tableofcontents

\section{Introduction}

The Linux Kernel I/O Scheduling algorithms developed along this path:
Linus Elevator -> Deadline -> Anticipitory

Then jumped to CFQ.

There is also the Noop Scheduler which is used for devices that have
random access.

I/O Scheduling algorithms are briefly touched on in Modern Operating
Systems (3e) on pages 379 - 382.

\section{First Steps}

All IO Schedulers are kernel modules. The default is chosen either at
compile time (with CONFIG\_DEFAULT\_IOSCHED), or during boot with the
\emph{elevator=} option.

Though told to start off with using the FIFO IO Scheduler as a template,
no FIFO Scheduler exists in the LInux Kernel. Thus, we began by using
the noop scheduler (noop-iosched.c) as a template.

\section{Algorithm}


\end{document}
