\documentclass[letterpaper,10pt,titlepage]{article}

\usepackage{graphicx}                                        
\usepackage{amssymb}                                         
\usepackage{amsmath}                                         
\usepackage{amsthm}                                          

\usepackage{alltt}                                           
\usepackage{float}
\usepackage{color}
\usepackage{url}

\usepackage{balance}
\usepackage[TABBOTCAP, tight]{subfigure}
\usepackage{enumitem}
\usepackage{pstricks, pst-node}

\usepackage{geometry}
\geometry{textheight=8.5in, textwidth=6in}

\newcommand{\cred}[1]{{\color{red}#1}}
\newcommand{\cblue}[1]{{\color{blue}#1}}

\usepackage{hyperref}
\usepackage{geometry}

\def\name{Group 13: Brian Cox, Matthew Okazaki, Trevor Bramwell}

%pull in the necessary preamble matter for pygments output
\input{pygments.tex}
\parindent = 0.0 in
\parskip = 0.1 in

%% The following metadata will show up in the PDF properties
\hypersetup{
  colorlinks = true,
  urlcolor = black,
  pdfauthor = {\name},
  pdfkeywords = {cs411 ``operating systems'' USB Rocket Launcher Driver},
  pdftitle = {CS 411 Project 5: USB Rocket Launcher Driver} 
  pdfsubject = {CS 411 Project 5},
  pdfpagemode = UseNone
}

\def\name{Trevor Bramwell}
\title{CS411: Individual Writeup - Group 13\\
    Project 5: USB Rocket Launcher}
\author{\name}

\begin{document}
\maketitle

%What do you think the main point of this assignment is?
\section*{Introduction}

The main point of this assignment was to learn how to write a USB device
driver for the Linux kernel, understand how the USB protocol works at a
low level, and write a userspace program that interacts with a driver.

Our group achieved this primarily by using a preexisting driver and
controller program we found online written by Nick Glynn. Nick had
previously completed a project similar to this one. 

%How did you personally approach the problem?
%Design decisions, algorithm, etc.
\section*{Approach}

I approached this assignment with high hopes that completing it would be
very simple. Before I started I was already aware of the \emph{Linux
Device Drivers, 3rd Edition} (LDD3) book by Greg Kroah-Hartman, et al. which
outlines how to write device drivers for the Linux kernel. Upon actually
reading through the USB device driver chapter of LDD3 I realized that
it would not be as great of an asset as I originally though. 

LDD3 is currently completely outdated. Though the general overview on
USB was somewhat informative, the rest of the chapter read primarily as
a reference guide to USB structs in the Linux kernel. Thus, I needed to
look else where -- le Internet -- for more information.

\section*{Design}

Since Matt ended up finding a pre-written driver and controller program
for the USB Dream Cheeky Rocket Launcher, not much else was required for
me to do besides test our resources.

% How did you ensure your solution was correct?
% Testing details, for instance.
\section*{Testing}

Testing for this project consisted of building the driver against the current
kernel. Loading the module into the Linux kernel, and then running the
userspace program.

This was done by using/modifying the example \textbf{Makefile} from LDD3
for compiling modules outside the Linux tree. It was similar to this:

\input{__kern_make.sh.tex}

once that was written to a file on Squidly 13, making the module was as
simple as running \emph{make}.

After the module was made, it needed to be loaded into the kernel. This
was accomplished by running:

\begin{alltt}
sudo insmod launcher\_driver.ko
\end{alltt}

I then ran the userspace program compiled from \emph{launcher\_control.c}
to move and fired the USB rocket launcher. This worked well, but didn't
allow easy control, so I edited it to continually grab character from
the user and send those `commands' directly to the launcher.

\subsection*{launcher\_control.c}
\input{__launcher_control.c.tex}

\section*{Learning}

Since getting the USB Rocket Launcher up and running was no very
challenging, because our group didn't actually write any code for the
driver, most of our time was used reading about how USB drivers work and
how to implement them.

As with the Shortest Seek Time First elevator algorithm, there was
already a predefined interface to work with in the kernel. This
interface required that the USB device driver implement the functions:

\begin{itemize}
\item open
\item release -- remapped as close
\item read
\item write
\end{itemize}

These were implemented in the driver that we got from Nick Glynn. This
is what each of them did.

\begin{description}[style=nextline]

\item[Open]
Open the device for access. Called when an application wants access to
the device through udev or the \emph{/dev} directory. Similar to opening
a file.

\item[Release]
Release the device from being accessed. Similar to closing a file.

\item[Read]
Read from the device. This does nothing for the USB Rocket Launcher as
it does not send back any data that is useful to a user.

\item[Write]
Write to the device. This is the meat of the USB Rocket Launcher driver
as it handled sending all the commands to the device and setting flags
for responses.

\end{description}

\section*{Conclusion}

Overall this was a fun assignment as each of us learned how to work with
devices that don't have pre-existing drivers, and we got to shoot
each other.


\end{document}
