\documentclass[letterpaper,10pt,titlepage]{article}

\usepackage{graphicx}                                        
\usepackage{amssymb}                                         
\usepackage{amsmath}                                         
\usepackage{amsthm}                                          

\usepackage{alltt}                                           
\usepackage{float}
\usepackage{color}
\usepackage{url}

\usepackage{balance}
\usepackage[TABBOTCAP, tight]{subfigure}
\usepackage{enumitem}
\usepackage{pstricks, pst-node}

\usepackage{geometry}
\geometry{textheight=8.5in, textwidth=6in}

\newcommand{\cred}[1]{{\color{red}#1}}
\newcommand{\cblue}[1]{{\color{blue}#1}}

\usepackage{hyperref}
\usepackage{geometry}

\def\name{Group 13: Brian Cox, Matthew Okazaki, Trevor Bramwell}

%pull in the necessary preamble matter for pygments output
\input{pygments.tex}
\parindent = 0.0 in
\parskip = 0.1 in

%% The following metadata will show up in the PDF properties
\hypersetup{
  colorlinks = true,
  urlcolor = black,
  pdfauthor = {\name},
  pdfkeywords = {cs411 ``operating systems'' Final Exam - Android x86},
  pdftitle = {CS 411 Final: Android x86} 
  pdfsubject = {CS 411 Final},
  pdfpagemode = UseNone
}

\def\name{Trevor Bramwell}
\title{CS411: Final\\
    Comparison of Android x86 kernel and Linux stable kernel}
\author{\name}

\begin{document}
\maketitle


%
% Project Overview
%
% %Look back at your previous assignments,
% choose 2 key concepts (plus another for extra
% credit) and briefly describe what your implementations for those to
% projects were. 
%
% Then, dig into the android source (provided on 
% \textbf{os-class - /scratch/android/}) and find where that concept is
% implemented, read through the code to figure out how
% the Android developers implemented that portion of the Android-specific
% kernel, and compare and contrast implementation
% algorithms between the Android version and your implementation.
% 
% To be more clear, write a bit about how you implemented the 2 or 3
% concepts of your choice, then write about how Android
% either does things differently, or how they are the same. 
% 
% Finally, write
% about why you think the Android Developers chose
% the implementation that they did - hardware capabilities, restrictions,
% etc. 

\section{Introduction}

I will briefly describe my changes to the process scheduler,  I/O
scheduler. Following that I will compare and contrast the Android-x86's
implementations of those concepts with my own implementations. Finally I
will explain why I think the Android Developers choose to use the
implementations they did.

\section{My Designs}
\subsection*{Process Scheduling}

Since there was already a Round Robin (RR) and First In First Out (FIFO) scheduler
in the Linux kernel, nothing was actually designed in this project. We
merely replaced what was missing. This included a policy check, enabling
of priority scheduling, setting the min and max values of the
priority, and updating of the process timeslice.

\subsection*{I/O Scheduling}

To implement the Shortest Seek Time algorithm for I/O scheduling, my
solution was to add all requests to the end of the list. Then, when a
new request needs to be dispatched, the last dispatched request is
compared to all the requests in the list until the smallest difference
in sectors is found. That request is then put on the dispatch queue.

\section{Android-x86 Designs}

% compare & contrast
\subsection*{Process Scheduling}

The Android process scheduler has no differences from the Linux kernel.
This is probably because  CFS works quite well as a general purpose
scheduler. Since CFS is focused on response time, it is a good choice
for mobile devices.

\subsection*{I/O Scheduling}

The differences between the I/O Scheduler of the Android-x86 kernel and
the Linux kernel are extremely minimal (as you can see from the
following patch). These changes also are more along the lines of bug
fixes than actual implementation.

\input{__io_kernel_diff.patch.tex}

\section{Conclusion}

There is very little to no differences, that I could find, between the
Android-x86 kernel and Linux kernel in terms of process and I/O
scheduling.

I find this somewhat surprising considering that Android is specifically
targeted at mobile devices. Perhaps if we were to go back a few years
prior to smart phones, when devices had a lot fewer resources available,
we would see a much larger change.

\section*{Addendum: Group evaluation}

\subsection*{Matthew Okazaki}

Matthew contributed a lot to our group. We would not have been able to
complete most of the assignments had he not stepped up and done a decent
bit of research for each project. He did a great job finding useful and
relevant information for our group. As a group member he worked well
reminding Brian and I when deadlines were coming up, and scheduling
meeting times.

During Project 2 Matthew was instrumental in understanding the shortest
seek time first algorithm, and it's implementation.

\subsection*{Brian Cox}

Brian worked well in our group. He primarily handled the compiling of
our writeup, and submission of our documents. As a member of our group
Brian was extremely helpful with his strong knowledge of C and Unix
programming. Brian designed almost all of the user space program we used
to test our shortest seek time first I/O scheduler.

During the last assignment Brian was also extremely helpful in his work
on USB packet sniffing.


\end{document}
