\documentclass[letterpaper,10pt,titlepage]{article}

\usepackage{graphicx}                                        
\usepackage{amssymb}                                         
\usepackage{amsmath}                                         
\usepackage{amsthm}                                          

\usepackage{alltt}                                           
\usepackage{float}
\usepackage{color}
\usepackage{url}

\usepackage{balance}
\usepackage[TABBOTCAP, tight]{subfigure}
\usepackage{enumitem}
\usepackage{pstricks, pst-node}

\usepackage{geometry}
\geometry{textheight=8.5in, textwidth=6in}

%random comment

\newcommand{\cred}[1]{{\color{red}#1}}
\newcommand{\cblue}[1]{{\color{blue}#1}}

\usepackage{hyperref}

\def\name{Trevor Bramwell}

%% The following metadata will show up in the PDF properties
\hypersetup{
  colorlinks = true,
  urlcolor = black,
  pdfauthor = {\name},
  pdfpagemode = UseNone
}

\title{CS411: Individual Writeup - Group 13}
\date{}
\author{\name}

\begin{document}
\maketitle

\subsection*{What do you think the main point of this assignment is?}

The main point of this assignment was several fold.

    \begin{enumerate}
        \item For us to learn how to compile the linux kernel.
        \item For us to understand how the linux scheduler works.
        \item For us to work as a team to create a Round Robin and FIFO
            process scheduler.
        \item And for us to learn to work as a group.
    \end{enumerate}

\subsection*{How did you personally approach the problem? Design
decisions, algorithm, etc.}

I began by reading. I re-read most of Chapter 4 of Linux Kernel
Development to make sure I really understood the concepts, and then dove
into the code. Starting in \emph{sched\_rt.c}, once I discovered the
\textbf{SCHED\_RR} and \textbf{SCHED\_FIFO} constants, I grepped around
for them in all the kernel files. This lead me to the
\emph{include/linux/sched.h} and
\emph{kernel/sched.c} files, which contained the \emph{struct
sched\_class} needed to implement scheduling classes.

I let my group know that I felt we were somewhat off track by not
implementing two new scheduling classes based on the
\emph{struct sched\_class}, but they were both confident that we were
only supposed to \emph{replace} what had been taken out.

So, I set my differences aside, and went with their decision.

\subsection*{How did you ensure your solution was correct? Testing
details, for instance.}

After our group realized we had to compile the kernel \emph{on} the
Squidly, it worked quite well. We knew that during the bootup
phase of the kernel, it uses Round Robin and FIFO scheduling for
processes. Thus, we determined that if it booted, and we could login,
our solution was correct and working as intended.

\subsection*{What did you learn?}

\begin{enumerate}
    \item Compiling is hard.

    \item Depending upon how things turn out for this assignment, I
probably need to stick up more for my beliefs.

If my group isn't going the direction I feel the teacher intended for
them to go, I should probably double check with the teacher.

    \item Get things working as soon as possible

One of the reason I think we might have gotten off track is due to the
long amount of time it took to get the Squidly up and running with a new
compiled kernel.

\end{enumerate}


\end{document}
