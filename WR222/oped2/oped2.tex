\documentclass[12pt,letterpaper]{article}

\usepackage{mymla}

\begin{document}
\begin{mla}{Trevor}{Bramwell}{Delf}{WR222 -- MWF 1600}{\today}{Defining Adultery}

Ross Douthat's article titled ``Is Pornography Adultery?'' and published
in \emph{The Atlantic}, starts an interesting discussion on the 
contingencies of adultery. Jumping off from a slew of divorce court cases 
in 2008, which involved pornography in someway, Douthat dives into an
examination of how pornography has affected American society. From statistics
on sexual violence to a survey of university student's thoughts on pornography,
Douthat provides the reader with enough evidence to support his claim that:
\begin{blocks}
The man who uses porn is cheating sexually, but he isn't involving himself in a relationship.
He's cheating in a way that carries non of the risks of intercourse, from pregnancy to 
venereal disease. And he's cheating with women who may be trading sex for money, but are 
doing so in vastly safer situations than streetwalkers or even high-end escorts. (Douthat)
\end{blocks}

The title of the article easily summarizes the word Douthat is defining: Adultery.
Douthat provides several alternative definitions of adultery before presenting his own.
Starting with Jesus of Nazareth's saying: ```I tell you that anyone who looks at a woman 
lustfully has already committed adultery with her in his heart.'',' he explains that 
Americans make a distinction between lustful thoughts and actions, and prefer the
\emph{Merriam-Webster} definition of adultery being ```voluntary sexual intercourse between a man 
and a someone other than his wife or between a married woman and someone other than her husband'' (Douthat).'
These alternative definitions help to frame Douthat's definition, and provide a framework around 
which he can build his argument.

Douthat's definition is primarily an operational one. He is defining the way in which
pornography is used as adultery. Although most of his definitional method is focused
on definition by example. This is due to the blurred line between how and if a person
other than a man/woman's spouse is involved. He uses the example of ``If it's cheating
on your wife to watch while another woman performs sexually in front of you, then why isn't
it cheating to watch while the same sort of spectacle unfolds on your laptop or TV?'' to
sharpen this line that though there is not a physical presence of another woman/man, the
situation should still be treated the same way.

There is a strong an engaging voice prevalent throughout this article. Douthat does a great
job of providing well researched and relevant statistics, along with strong logical 
correlations, that lead the reader to think more deeply about the meaning of adultery. 
A stronger argument may have been made by better contextualization of some of his lesser
known quotations, as readers who are not a part of his audience may not understand them.
While staying away from an enticing set of logical fallacies he moves the discussion,
in his conclusion, towards the societal impact of what his new definition will mean.

\begin{workscitedinline}
\bibent
Douthat, Ross. ``Is Pornography Adultery?'' \emph{The Atlantic}. Web. 10 Feb. 2012. 
\end{workscitedinline}
\end{mla}
\end{document}
