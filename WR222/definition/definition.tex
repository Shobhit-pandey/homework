\documentclass[12pt,letterpaper]{article}

\usepackage{mymla}
\usepackage[colorlinks=true]{hyperref}
\begin{document}
\begin{mla}{Trevor}{Bramwell}
{Delf}{WR222 -- MWF 1600}{\today}{Oh the Zombanity!}

It's 6am. You wake up and hear moaning coming from the other side of your
bedroom door. Afraid of what it could be, you pick up the trusty shotgun
leaning against your bed as you pull away the covers. Climbing out of bed
you bravely open the door. Standing before you is your mother, now a zombie.
Shuffling towards you and trying to eat your brains. Your natural instincts
kick in and you raise up your shotgun as you back away.
Would you pull the trigger?

Contrary to pop culture, taking the life of a zombie is not a simple matter.
Mention zombies to anyone and their mind will juxtapose images of
chainsaws, shotguns, and katanas. American culture has deeply ingrained within
us the will to kill all zombies, but killing zombies is morally and ethically
wrong because zombies are humans too.\label{thesis}
Zombies are humans because they are merely humans who have been infected with
a virus.

Humans did not evolve or mutate into zombies. 
Zombies are just very sick humans, and humans who are sick with a virus can
be cured though the use of antiviral drugs. Therefore, if zombies can be returned
to their original state, they should be considered as human.

Several people have commented that zombies can't be human since they
have died and then been brought back to life, but this is based on the false
assumption that I am referring to director John Romero's zombies from his
cult-classic movie \emph{Night of the Living Dead}. The zombies I am actually
referring to are known as infected zombies, those created through a viral
infection. This definition comes from Max Brooks' book
\emph{The Zombie Survival Guide}. 

Although Brooks helps to define what
zombies are, his guide itself contains large sections
devoted to the extermination of zombies and very little information on peaceful
detention or the seek for a cure.
In the second chapter titled `Weapons and Combat Techniques,'
Brooks provides his readers with a large amount of
information regarding how to properly kill zombies such as
``The advantage of slicing over bludgeoning is that
it can make killing a zombie unnecessary. In some cases, simply chopping off a
limb or severing the spine is enough to disable an undead assailant'' (Brooks).
This information is focused much more towards fighting than self defense.
Though by their nature zombies want to kill people, that does not entitle
humans to retaliate against them. Though some people may be placed into a
position where they must defend themselves, there are other non-violent ways
to deal with zombies.

If humanity doesn't take time to think about how we might treat zombies in the
future, then we will inevitably massacre a large portion of our species.
Because of their lacking mental capacity and communication skills, zombies
would provide humans with an excuse to exercise their base desire for murder.

\label{conclusion}
When a zombie outbreak does occur, zombies  be treated with the same respect
as outlined by the
United Nations in Article 3 of \emph{The Universal Declaration of Human Rights}
: ``Everyone has the right to life, liberty and security of person''
(United Nations).
Stacey Forsythe, an advocate of zombie rights and respected speaker for the
Citizens for Undead Rights and Equality (C.U.R.E), supports this claim with
her statement that 
``...the living dead share our planet and are not ours to be used for
entertainment or experimentation, regardless of their level of decomposition
or state of decay – just as living humans are granted right regardless of
their intelligence or personal likeability'' (Forsythe). Treating zombies with
the same rights as humans is the ethical thing to do, because zombies are
humans too.

\pagebreak
{\centering Rhetorical Analysis\\}
I will be writing this for \emph{The New York Times} because it is a respected
publication and has a large audience both within and outside the U.S., and
would be read by those in densely populated areas where a zombie outbreak is
most likely to occur first. My audience for this essay is Americans from 16 to
32; those people interested in zombies and zombie related material. Because
\emph{The New York Times} is a well known and respected publication I focused
on making my voice clear and understandable. Straying from large and
genera specific words so as to include the large readership it has.
Trying to find a way to define zombies as human and not humans as zombies
was the most challenging part of this essay since most of the examples that
I came up with fit the definition of zombies more than humans.

%\paragraph{References}
%\begin{itemize}
%\item
%\href{http://zombie.wikia.com/wiki/Zombies\_(Max\_Brooks)}{Max Brooks' Zombies}
%\item
%\href{http://zombie.wikia.com/wiki/Infected}{Infected Zombies}
%\item
%\href{http://www.citizensfortheundead.com/about.html}{CURE}
%\end{itemize}

\begin{workscited}
\bibent
Forsythe, Stacey. ``About Cure and Frequently Asked Questions.'' \emph{Citizensfortheundead.com}. Capcom Inc., n. d. Web. 15 Feb. 2012.

\bibent
Brooks, Max. \emph{The Zombie Survival Guide}. New York: Three Rivers Press,
2003. Print.

\bibent
United Nations. Human Rights Division. \emph{The Universal Declaration of Human Rights.} 
\end{workscited}
\end{mla}
\end{document}
