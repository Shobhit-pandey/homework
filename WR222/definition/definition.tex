\documentclass[12pt,letterpaper]{article}

\usepackage{mymla}
\usepackage[colorlinks=true]{hyperref}
\begin{document}
\begin{mla}{Trevor}{Bramwell}{Delf}{WR222 -- MWF 1600}{\today}{Oh the Zombanity!}

It's 6am. You wake up and hear moaning coming from the other side of your
bedroom door. Afraid of what it could be, you pick up the trusty shotgun
leaning against your bed as you pull away the covers. Climbing out of bed
you bravely open the door. Standing before you is your mother, now a zombie.
Shuffling towards you and trying to eat your brains. Your natural instincts
kick in and you raise up your shotgun as you back away.
Would you pull the trigger?

Contrary to pop culture, taking the life of a zombie is no simple matter.
Mention zombies to anyone and their mind will juxtapose images of
chainsaws, shotguns, and katanas. Our culture has deeply ingrained within us
the will to kill all zombies. This is morally and ethically wrong
because zombies are humans too. Zombies who came to be through a viral
transformation of a human, are still humans.\label{thesis}  
Humans who have been infected with a virus and turned into zombies should
still be considered human, because viruses can be cured. Therefore, with the
right medical treatment, they could be returned to their original human state.

Several people have commented that zombies can't be human since they have died
and then been brought back to life, but this is based on the false assumption
that I am referring to Romero's zombies from \underline{Night of the
Living Dead}. The zombies I am referring to are known as the modern infected
zombie.

Although my definition of zombies comes from Max Brooks'
\emph{The Zombie Survival Guide}, the guide itself contains large sections
devoted to exterminating zombies and very little information on peaceful
detention.  It promotes violence against zombies.
In the second chapter titled `Weapons and Combat Techniques,'
Brooks provides his readers with a large amount of
information regarding how to properly kill zombies. Though it is understood
that due to the nature of the zombie thought process and wanting to kill people
and all, a time will inevitably come when people must defend themselves. The
information given, such as ``The advantage of slicing over bludgeoning is that
it can make killing a zombie unnecessary. In some cases, simply chopping off a
limb or severing the spine is enough to disable an undead assailant.'' (Brooks)
, is worded much more towards fighting than self defense. 

If humanity doesn't take time to think about how we might treat zombies in the
future, how will humankind treat them when they already walk about? Any
American can attest to this truth. People with headphones in while walking to
their next class. Store clerks who ring up your purchase and return your
change. Cinema ticket takers. File clerks. Government employees.
All of them humans, and all of them zombies. Are coma patients slaughtered
because of their mental state? Absolutely not.
Even those some people refer to as vegetables, who have severe brain damage
and are no longer able to function on their own, do not get put to death, but
spend the rest of their days in a mental institution. 



\paragraph{Conclusion}
\subparagraph{Why does this matter?}
When and if a zombie outbreak does occur, zombies should not be outright
massacred, but be treated with the same respect as humans. Zombies should be
treated with the same respect all humans are given.
According to \emph{The Universal Declaration of Human Rights}, Article 3:
``Everyone has the right to life, liberty and security of person.''

\paragraph{Notes}
I'm defining humans, and how zombies fit in the category of humans.
I am not defining zombies with the exception of what zombies I am referring to.

Zombies are Humans. Not: Humans are Zombies.
Both have a body
Both have a mind
Both have a virus
Both eat
Both die
Both are violent
Both communicate


\subparagraph{Human Rights}
Human Rights are defined by The Universal Human Rights (United Nations).

\subparagraph{Publication}
I will be writing this for \emph{The New York Times} because it is a respected
publication and has a large audience both within and outside the U.S., and
would be read by those in densely populated areas where a zombie outbreak is
most likely to occur first.



% Works Cited
\paragraph{References}
\begin{itemize}
\item
\href{http://zombie.wikia.com/wiki/Zombies\_(Max\_Brooks)}{Max Brooks' Zombies}
\item
\href{http://zombie.wikia.com/wiki/Infected}{Infected Zombies}
\item
\href{http://www.citizensfortheundead.com/about.html}{CURE}
\end{itemize}

\begin{workscited}
\bibent
Forsythe, Stacey. ``About Cure and Frequently Asked Questions.'' \emph{Citizensfortheundead.com}. Capcom Inc., n. d. Web. 15 Feb. 2012.

\bibent
Brooks, Max. \emph{The Zombie Survival Guide}. New York: Three Rivers Press,
2003. Print.

\bibent
United Nations. Human Rights Division \emph{The Universal Declaration of Human RIghts.} 
\end{workscited}
\end{mla}
\end{document}
