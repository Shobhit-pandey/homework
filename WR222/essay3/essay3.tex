%--------------
% Title: WR222 - Essay 3
% Author: Trevor Bramwell
% Date: March 10th, 2012
%--------------
\documentclass[12pt,letterpaper]{article}
\usepackage{mymla}
\usepackage[colorlinks=true]{hyperref}

\begin{document}
\begin{mla}{Trevor}{Bramwell}
{Delf}{WR222 -- MWF 1600}{\today}{The History of The Oregon
    State Open Source Lab}

%\section*{Hook}
On August 25th, 1991, Linus Torvalds changed the world with one email. 
\begin{quotation}
Hello everybody out there using minix - 
I'm doing a (free) operating system (just a hobby, won't be big and 
professional like gnu) for 386(486) AT clones.  This has been brewing 
since april, and is starting to get ready.  I'd like any feedback on 
things people like/dislike in minix, as my OS resembles it somewhat 
(same physical layout of the file-system (due to practical reasons) 
among other things). (Torvalds) 
\end{quotation}
As those at \emph{Linux Weekly News} know, Linus had
began working on a free operating system which would be known today as
GNU/Linux. It served as a catalyst for the open source movement. Without
GNU/Linux, or the open source movement, we wouldn't have some of the most 
widely used and know software projects in the world: Mozilla Firefox, Gentoo, 
the Apache Web Server, Debian, and Drupal.

%\section*{Introduction/Thesis}
Where do all these globally known open source project reside today? 
Corvallis, OR.
Wait. It's not Cupertino, CA, home of Apple; Redmond, WA, home of Microsoft;
or Mountain View, CA, home of Google? Nope! All of them, with the exception of
Firefox, reside in Corvallis, OR, home of the Oregon State Open Source Lab 
(OSUOSL).
The OSUOSL hosts the largest collection of high impact open source software in 
the world. Though a lot of people don't know this because they do not directly
interact with the OSUOSL to download Drupal or the Apache Web Server.
The Open Source Lab did not evolve overnight night, but through a series of 
interrelated events. %\label{thesis}
Founded in
late 2003, the Oregon State Open Source Lab came about due to the hosting 
needs of the
Gentoo community, a financial disaster in the Oregon State Information Services
department, a large increase of available network bandwidth, and a domain name 
for the worst education.

%\section*{Open Source Movement}
For those who are unaware of what the open source movement here is a brief
overview of it's inception.
Steven Weber explains in his book how Richard Stallman started the open source
movement. ``He founded the Free Software Foundation as a non-profit 
organization to support the work. The goal was the produce an entirely free
operating system that anyone could download, use, modify, and distribute freely
.''
Back in the 1990's there were no free operating systems available.
GNU/Linux 
was the first free and open source operating system. It was based 
on MINIX, which was created by Professor Andrew Tanenbaum of Vrije 
Universitait, Amsterdam as a teaching tool to understand how the UNIX 
operating system worked. At the same time Linus was creating GNU/Linux, the GNU
Project, out of the Free Software Foundation, had the same goal in mind: to 
developing a free and open source 
operating system. Richard Stallman released the
GNU Project's source code under the GNU General Public License, the first open 
source license. Now with a full operating system and public license which
helped frame their ideals, the open source movement was well underway. 

%\section*{Gentoo Hosting}
One of the projects that came out of the open source movement was the Gentoo 
linux distribution.
Unlike other GNU/Linux distributions, Gentoo is special because of it's
involvement in the creation of the Oregon State Open Source Lab.
``It's not an overstatement to say that Gentoo was integral to the Open Source
Lab’s foundation. The Linux-based operating system was one of the OSUOSL's 
first
projects – it even preceded the existence of the lab – and was instrumental in
building the buzz that put open source at OSU on the map'' (Reciprocity and
Gentoo). As Gentoo began to grow it needed more hosting. The Gentoo community
reached out to Oregon State and a partnership was formed that laid a
foundation for the mission of the OSUOSL. But before it was hosted by the
OSUOSL it was originally hosted by Oregon State's Information Services 
department (IS), which after a series of events underwent some 
economic strees.

%\section*{Oregon State Budget}
Scott Kventon, one of the founders of the OSUOSL, explained this stress to a
member of the KernelTrap project in a blog post in 2005. ```OSU's information
services group had a bit of a financial disaster 8 years
ago [1997]; our former CIO went a little crazy writing checks, etc. After the 
dust settled, we were left with a huge debt, and increasing user base and the 
need to provide a lot more with a lot less. We had no choice but to turn to 
open source''' (Andrews).
When Curt Pederson came to Oregon State University as the new Chief Information
Officer he had no idea that the IS deparment was \$6,000,000
in debt. He set up a six year plan that would cut \$1,000,000 every year. A
large part of this plan involved leveraging open source software over
proprietary software which costs a large amount due to licensing fees.
Once Information Services was back in the black, and with the the goal of
enabling open source to grow at Oregon State, Pederson reinvested
some of the money into increasing the network bandwidth available to the 
university (Pederson).

%\section*{Dark Fiber}
With the extra money available from the cuts, Pederson funded the laying
of unused, also know as 'dark', fiber optic cable to Interstate 5.
Having been around at the time, Scott Kventon told Kernel Trap: ```Our CIO 
(Curt Pederson) chose to fund a fiber build-out to I-5 and the Open
Source Lab. The strategic funds are seed money to get the OSL off the ground
and we're well on our way to cost recovery''' (Andrews).
The laying of this dark fiber was key to the OSL success. TDS Telecom saw that
The Open Source Lab was putting out a lot of bandwidth, and wanted to partner
with them to save money. With funding from TDS Telecom and Oregon State 
University, 21 miles of fiber optic cable
were laid between OSU and Interstate 5. This construction cost the university
\$500,000, but provided them with 2 Gibibytes of bandwidth from TDS at no 
charge; that amount of bandwidth is worth \$300,000 a year. 
TDS also provided the OSUOSL with three mirror hosting sites for downloads.

%\section*{Worst Domain Name}
Not too long after this extra bandwidth was available, the former Oregon State
University President Paul Risser saw a sign on the highway advertising Oregon
State's website at `www.orst.edu'. He didn't want Oregon State to be know as
the University with the `worst' education website (Pederson).
Read that domain name again: `ww-worst-edu'. To fix that problem Risser hired
Scott Kventon to change the domain name. When Kventon came to Oregon State,
Pederson had just requesitioned a server room in the Kerr Administration
building from holding Banner, Oregon State's administrative website, to be used
as the home of the Oregon State Open Source Lab.

%\section*{Conclusion}
It would be wrong to say that a single event was responsible for the creation
of the Oregon State Open Source Lab. Gentoo's need for hosting, the financial
mishaps of the Oregon State University Information Services department, the
laying of fiber to Interstate 5, and a domain name for the `worst' education,
all contributed to the creation of the OSUOSL. As supporters of the open source
community at large, the lab has helped to filled a gap in the needs of open
source projects all over the world. 
Without Richard Stallman creation of the Free Software Foundation and GPL,
or Linus' work on the GNU/Linux operating system, open source software
and the open source movement would exist today.  
But without the Oregon State Open Source Lab, that open source software 
would not continue to be free to download, use, or modify.

%\pagebreak
%{\centering Notes\\}
\begin{description}
    \item[Thesis]
        Several events led to the creation of The Oregon State Open Source Lab 
("OSL"). Fifteen miles of dark-fiber were laid out to I-5 to help load balance
web traffic to the university, and the previous CIO cause the university's IS
department to be in debt, leading to the university moving towards open source
technologies.
    \item[Stakes]
        There is not definite understanding of how the open source lab was
created. Quotes from several different old employees conflict on this topic.
    \item[Publication]
        This essay is being written for Linux World News (lwn.net).
    \item[Alternative Cause/Effect]
        www.orst.edu domain seen by Ed Ray. Didn't want to have 'worst' domain.
This lead to a focus on web presence.
    \item[Speakers]
        Joe Brockmeier, a writer for linux.com, wrote an overview of the OSL
that includes information regarding it's history.
        Jeremy Andrews, a writer for Kerneltrap.org, blogged about Kernel
Trap's relocation to the OSL.
\end{description}

\noindent\textbf{Question}: What caused the OSL to be created?
\begin{itemize}
    \item Open Source Movement \\
        The open source movement starting with linux and progressed from there.
        It is hard to talk about open source in general without metioning the
        movement.

    \item Dark fiber \\
        ```Our CIO (Curt Pederson) chose to fund a fiber build-out to I-5 and the Open
        Source Lab. The strategic funds are seed money to get the OSL off the ground
        and we're well on our way to cost recovery.''' [Scott Kveton] (Andrews).

    \item Forced into open source due to budget \\
        ```OSU's information services group had a bit of a financial disaster 8 years
        ago; our former CIO went a little crazy writing checks, etc. After the dust
        settled, we were left with a huge debt, and increasing user base and the need
        to provide a lot more with a lot less. We had no choice but to turn to open
        source.''' [Scott Kveton] (Andrews).

    \item Worst domain name \\
        "Oregon State University president saw a sign with "www.orst.edu," and didn't
        want to have the "worst" edu domain. (Read it again.) That led OSU's president
        to get in touch with the folks running the Web site. Having drawn the attention
        of the university leadership to the Web presence, OSU paid more attention to
        hosting and eventually got into the business of hosting for open source
        projects." (Brockmeier).

    \item Gentoo hosting \\
        \tab One of the primary causes for the OSL's creation was Oregon State University's
        hosting of a Gentoo server. Gentoo is a linux distribution. ```Web services
        donated an old Dell server, and from there Gentoo just grew and grew,' says
        Lance Albertson'' (http://osuosl.org/about/stories/gentoo).
\end{itemize}

%\pagebreak 
\pagebreak
{\centering Rhetorical Analysis\\}

This article is directed at the audience of \emph{Linux Weekly News} (lwn.net).
To reflect this audience I didn't take as much time to provide information on 
what linux is, or how it evolved, as non-users of linux might require. I
focused on using clear and understandable language, evidence relative to open
source software, and a direct tone. To not bore the reader I tried to vary my
sentence structure which can be evident in the second paragraph. ``Corvallis, 
OR.  Wait. It's not Cupertino, CA, home of Apple; Redmond, WA, home of 
Microsoft; or Mountain View, CA, home of Google? Nope!'' Since most of my 
audience is either in the
computer science field, or a hard science, I focused most of my attention on
logical appeals; there is very little use of ethos since this article
is focused on history, and writing that relates to history should be unbiased.
To better contextify why I wrote this article and why the subject is important
to me, I could have taken some time to explain my employment at the Open Source
Lab and my involvement in open source development, giving my article a needed
dose of pathos.

\begin{workscited}
\bibent
Andrews, Jeremy. ``Kernel Trap: New Home At The Open Source Lab''.
    \emph{Kerneltrap.org}. 6 May 2005. Web. 10 Mar. 2012.
% http://kerneltrap.org/node/5083

\bibent
Brockmeier, Joe. ``A Look At Oregon State University's Open Source Lab''.
    \emph{Linux.com}. 19 Aug. 2011. Web. 10 Mar. 2012.
% https://www.linux.com/news/enterprise/biz-enterprise/486133-
%   a-look-at-oregon-state-universitys-open-source-lab

\bibent
Pederson, Curt. Personal Interview. 9 Mar. 2012

\bibent
``Reciprocity and Gentoo.'' \emph{osusol.org}. Open Source Lab, n.d., Web.
    16 Mar. 2012.

\bibent
Torvalds, Linus. ``What would you like to see most in minix?''. \emph{Usenet}.
    25 Aug. 1991. Web 12 Mar. 2012

\bibent
Weber, Steven. \emph{The Success of Open Source}. Harvard University Press, 
    2004. 
Cambridge, Massachusetts.
London, England.
\end{workscited}

\end{mla}

\end{document}
