%--------------
% Title: WR222 - Essay 3
% Author: Trevor Bramwell
% Date: March 10th, 2012
%--------------
\documentclass[12pt,letterpaper]{article}
\usepackage{mymla}
\usepackage[colorlinks=true]{hyperref}

\begin{document}
\begin{mla}{Trevor}{Bramwell}
{Delf}{WR222 -- MWF 1600}{\today}{The History of The Oregon
    State Open Source Lab}

% Linux/Open Source Movement
On August 25th, 1991, Linus Torvalds changed the world with one email. 
(http://groups.google.com/group/comp.os.minix/msg/b813d52cbc5a044b). Linus had
began working on a free operating system which would later be known as
GNU/Linux, commonly referred to today as just Linux. Back in the 1990's there
were really no free operating systems available, with the exception of UNIX,
which was freely distribution to universities and government agencies. Linux 
was the first free, and open source alternative operating system. It was based 
on MINIX, which was created by Professor Andrew Tanenbaum of Vrije 
Universitait, Amsterdam as a teaching tool to understand how the UNIX 
operating system worked. At the same time Linus was creating Linux, the GNU
Project was developing a free operating system of their own. Richard Stallman
was the leader of the project and also created the first open source license:
the GNU General Public License. Both the GNU and Linux worked in cohesion to
being the open source movement. Without either Richard Stallman ideas of free
softare, or Linus' work on the Linux operating system, open source software
would not be where it is today.  This movement is not something of the past. 
It is alive and well today.

Where do all the leaders in open source technology reside? Corvallis, Oregon. 
Wait. It's not Cupertino, CA home of Apple, Redmond, WA home of Microsoft,
or Mountain View, CA home of Google, but Corvallis, OR: home of the Open Source
Lab. The Oregon State Open Source Lab hosts the largest collection of high
impact open source software in the world. Most people don't know this wonderful
fact though because the primary thing the Open Source Lab does is hosting. It
is not actually home to any large open source community, so it make sense why
it is largely unheard of outside of the open source world. Yet the history of
the Open Source Lab is shrouded in a thick web of events. None of which singly
brought about it's creation. The Open Source lab was founded after Curt
Pederson became Chief Information Officer of Oregon State University.

% Finances
```OSU's information services group had a bit of a financial disaster 8 years
ago; our former CIO went a little crazy writing checks, etc. After the dust
settled, we were left with a huge debt, and increasing user base and the need
to provide a lot more with a lot less. We had no choice but to turn to open
source.''' [Scott Kveton] (Andrews).
When Curt Pederson came to Oregon State University as the new Chief Information
Officer he had no idea that the Information Services deparment was \$6,000,000
in debt. He set up a six year plan that would cut \$1,000,000 every year. Once
Information Services had more money, Pederson choose to reinvest it in
increasing the network bandwidth available to the university.

% Dark Fiber
```Our CIO (Curt Pederson) chose to fund a fiber build-out to I-5 and the Open
Source Lab. The strategic funds are seed money to get the OSL off the ground
and we're well on our way to cost recovery.''' [Scott Kveton] (Andrews).
The laying of this dark fiber was key to the OSL success. TDS Telecom saw that
The Open Source Lab was putting out a lot of bandwidth, and wanted to partner
with them to save money. As a partnership
between TDS Telecom and Oregon State University, 21 miles of fiber optic cable
were laid between OSU and Interstate 5. This construction cost the university
\$500,000, but provided them with 2gbs of bandwidth from TDS at no charge; that
amount of bandwidth is worth \$300,000 a year. TDS also provided the Open
Source Lab with three mirror sites for download.


\pagebreak
{\centering Notes\\}
\begin{description}
    \item[Thesis]
        Several events led to the creation of The Oregon State Open Source Lab 
("OSL"). Fifteen miles of dark-fiber were laid out to I-5 to help load balance
web traffic to the university, and the previous CIO cause the university's IS
department to be in debt, leading to the university moving towards open source
technologies.
    \item[Stakes]
        There is not definite understanding of how the open source lab was
created. Quotes from several different old employees conflict on this topic.
    \item[Publication]
        This essay is being written for Linux World News (lwn.net).
    \item[Alternative Cause/Effect]
        www.orst.edu domain seen by Ed Ray. Didn't want to have 'worst' domain.
This lead to a focus on web presence.
    \item[Speakers]
        Joe Brockmeier, a writer for linux.com, wrote an overview of the OSL
that includes information regarding it's history.
        Jeremy Andrews, a writer for Kerneltrap.org, blogged about Kernel
Trap's relocation to the OSL.
\end{description}
\pagebreak
                        
\noindent\textbf{Question}: What caused the OSL to be created?
\begin{itemize}
    \item Open Source Movement \\
        The open source movement starting with linux and progressed from there.
        It is hard to talk about open source in general without metioning the
        movement.

    \item Dark fiber \\
        ```Our CIO (Curt Pederson) chose to fund a fiber build-out to I-5 and the Open
        Source Lab. The strategic funds are seed money to get the OSL off the ground
        and we're well on our way to cost recovery.''' [Scott Kveton] (Andrews).

    \item Forced into open source due to budget \\
        ```OSU's information services group had a bit of a financial disaster 8 years
        ago; our former CIO went a little crazy writing checks, etc. After the dust
        settled, we were left with a huge debt, and increasing user base and the need
        to provide a lot more with a lot less. We had no choice but to turn to open
        source.''' [Scott Kveton] (Andrews).

    \item Worst domain name \\
        "Oregon State University president saw a sign with "www.orst.edu," and didn't
        want to have the "worst" edu domain. (Read it again.) That led OSU's president
        to get in touch with the folks running the Web site. Having drawn the attention
        of the university leadership to the Web presence, OSU paid more attention to
        hosting and eventually got into the business of hosting for open source
        projects." (Brockmeier).

    \item Gentoo hosting \\
        \tab One of the primary causes for the OSL's creation was Oregon State University's
        hosting of a Gentoo server. Gentoo is a linux distribution. ```Web services
        donated an old Dell server, and from there Gentoo just grew and grew,' says
        Lance Albertson'' (http://osuosl.org/about/stories/gentoo).
\end{itemize}

\pagebreak
{\centering Rhetorical Analysis\\}

\begin{workscited}
\bibent
Joe Brockmeier. ``A Look At Oregon State University's Open Source Lab''.
    \emph{Linux.com}. August 19, 2011
% https://www.linux.com/news/enterprise/biz-enterprise/486133-
%   a-look-at-oregon-state-universitys-open-source-lab

\bibent
Jeremy Andrews. ``Kernel Trap: New Home At The Open Source Lab''.
    \emph{Kerneltrap.org}. May 6, 2005.
% http://kerneltrap.org/node/5083
\end{workscited}
\end{mla}
\end{document}
