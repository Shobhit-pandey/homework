%--------------
% Title: WR222 - Essay 3
% Author: Trevor Bramwell
% Date: March 10th, 2012
%--------------
\documentclass[12pt,letterpaper]{article}
\usepackage{mymla}
\usepackage[colorlinks=true]{hyperref}

\begin{document}
\begin{mla}{Trevor}{Bramwell}
{Delf}{WR222 -- MWF 1600}{\today}{An Overview of My Writing for WR222}

My journey through Written English 222 at Oregon State has not been an easy
one. After having taken the class four times, two withdraws and one failure, I
have come to a deeper and clearer understaning of myself as a writer. I have
found one great weekness, of which I already knew, and one great strength, of
which I was unaware. 

My weekness is fear. I have an unrelenting fear of
writing. Coming from the world of computer science where we use version control
to maintain a lot of our code, it is easy to go back and edit what we have
already written, or ammend a piece of functionality. Writing English is not as 
forgiving. After turning in a paper I have no hope of returning to it to fix
errors or revise the structure. It is as if each paper I turn in is written in
stone. This causes me great fear as I know even before I sit down to write a
paper the final draft must be perfect.

My strength is outlining. Throughout this class I have developed a great
proccess of outlining that works well for me. Even before I began writing this
essay in \LaTeX, I sat down and defined a structure in which I can easily
plug in chunks of text and move them around as need be. That way after I have
developed a thesis, and have a firm understanding of what I want to say, each
part of the essay can be worked on independently. It helps greatly to
compartmentalize my thoughts.

Apart from this weekness and strength I have developed in several areas of my
writting over this past term. The areas in which I have improved the most are 
my attention to my audience, my framing of quotes, and my creation of engaging
introductions.

When I first began this class I had never even thought about who my audience
was. I found that for each successive writing assignment, I was progressively
narrowing down my audience. This helped me to focus on a smaller group of
people, constraining my voice to a smaller vocabulary, and incidentily 
freeing me to focus more on my argument than word choice. My first essay was
aimed at a vague user base: the average Windows user. Frankly I had no idea
who the average Windows user was. Had I taken the time I could have found some
research study done at Microsoft on that exact topic.
My second essay was directed at readers of \emph{The New York Times}, which was
a larger and even more ambiguous audience than the former; which, having
actually taken the time to research, I was unable to identify an age or
occupational rage for.
For the third and final essay I choose a publication I was actually familiar
with, and therefore able to identify it's audience. This can be seen by my
second sentence in the essay:``As those at Linux Weekly News know, Linus had 
began working on a free 
operating system which would be known today as GNU/Linux.''

The second area in which I found I had improved the most was my framing of
quotations. Before this class I would have not have thanken twice about how
I framed a quote, before placing it directly in an essay. Being new to the idea
I fumbled around in my first essay trying to put in quotes: ``In his article `
If this suite's a success, why is it so buggy?' 
published in the Guardian in December of 2005, Brown raises the point that 
as...'' In my second Op-Ed piece I handled framing much better, ``From 
statistics on sexual violence to a survey of university student’s 
thoughts on pornography, Douthat provides the reader
with enough evidence to support his claim that...'' and continued to do so into
my second and third essays.

Probably my most varied and creative area of improvement was in my 
introductions. In my first Op-Ed piece I wrote, ``In the New York Times 
article `Proof to Life as You Knew It' published 
December 26, 2011, Jeannette Catsoulis provides the reader with a well informed
review of the movie `The Darkest Hour','' which, looking back, seems very 
bland. In the first essay I had my weakest introduction: ``Everybody likes to 
save money and everybody likes things that are 
free. What if you could save money for free? If you currently use Microsoft 
Office, you can do just that,'' Though it sounded good in my head, it
was hard to see the connection I was trying to make to free software. The
strongest introduction I had was in my second essay. I gave an
anacdote that, had this essay been about a moral issue, done even better, but
I used it soely to catch the audiences attention with my use of descriptive 
languge. The second essay's introduction was this: ``It’s 6am. You wake up and 
hear moaning coming from the other side of 
your bedroom door. Afraid of what it could be, you pick up the trusty shotgun 
leaning against your bed as you pull away the covers. Climbing out of bed you 
bravely open the door. Standing before you is your mother, now a zombie. 
Shuffling towards you and trying to eat your brains. Your natural instincts
kick in and you raise up your shotgun as you back away. Would you pull the 
trigger?''

WR222 has helped me to become a better writer. It has taught me the benefit of
writing for a specific audience, how to properly frame quotations, and the best
way to craft an introduction that hooks an audience. These areas of writing
will continue to grow and benefit the future courses I take here at Oregon
State and abroad. Hopefully I will get to the point in my writing where I am
able to write with courage, and one day be published.
\end{mla}

\end{document}
